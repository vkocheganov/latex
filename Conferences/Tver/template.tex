\documentclass[a4paper,twoside]{article}

\usepackage[utf8x]{inputenc}
%\usepackage[cp1251]{inputenc}
\usepackage[T2A]{fontenc}
\usepackage{array}
\usepackage{amssymb}
\usepackage{amsmath}
\usepackage{amsthm}
\usepackage{latexsym}
\usepackage{indentfirst}
\usepackage{bm}
\usepackage{enumerate}
\usepackage{graphicx}
\usepackage{epsf}
%\usepackage{epsfig}
%\DeclareGraphicsExtensions{.pdf,.png,.jpg}
\usepackage{wrapfig}
\usepackage{euscript}
\usepackage{indentfirst}
\usepackage[english,russian]{babel}


\sloppy\unitlength=.24mm
\renewcommand{\thefootnote}{\arabic{footnote}}

\textwidth=132mm
\headheight=7mm
\headsep=5mm
\textheight=200mm
\oddsidemargin=0mm
\evensidemargin=0mm
\topmargin=0mm

\newcommand{\firstheader}[1]{\noindent\textbf{#1}\nopagebreak\bigskip}
\newcommand{\header}[1]{\bigskip\medskip\noindent\textbf{#1}\nopagebreak\bigskip}
\newcommand{\subheader}[1]{\bigskip\medskip\noindent\emph{#1}\nopagebreak\bigskip}
\newcommand{\subsub}[1]{\bigskip\medskip\noindent\emph{#1}\nopagebreak\bigskip}

\theoremstyle{theorem}
\newtheorem{theorem}{Теорема}
\newtheorem{Theorem}{Теорема}
\newtheorem{definition}{Определение}
\newtheorem{Def}{Определение}
\newtheorem{corollary}{Следствие}
\newtheorem{proposition}{Предложение}
\newtheorem{prop}{Предположение}
\newtheorem{lemma}{Лемма}
\newtheorem{assumption}{Предположение}
\newtheorem{Lemma}{Лемма}
\newtheorem{Cons}{Следствие}
\newtheorem{Proposition}{Предложение}
\newtheorem{Statement}{Утверждение}
\newtheorem{statement}{Утверждение}
\theoremstyle{remark}
\newtheorem{remark}{Замечание}
\newtheorem{Remark}{Замечание}
\newtheorem{example}{Пример}
\newtheorem{Example}{Пример}
\newtheorem{notation}{Замечание}
\newtheorem{teo}{Теорема}
\newtheorem{sled}{Следствие}
\newtheorem{sublemma}[theorem]{\indent\bf Подлемма}
\newtheorem{problem}{\indent\bf Проблема}
\newtheorem{hypothesis}{\indent\bf Гипотеза}
\newtheorem{denotation}[theorem]{\indent\bf Обозначение}
\newtheorem{thr}{Теорема}
\newtheorem{crl}[thr]{Следствие}
\newtheorem{lmm}[thr]{Лемма}
\newtheorem{qu}{Вопрос}
\newtheorem{dfn}{Определение}
\newtheorem{approval}{Утверждение}

% нумерацию можно оставить как есть
\newcommand{\pages}{1-8}

\begin{document}
\pagestyle{headings}
\makeatletter
\renewcommand{\@evenhead}{\raisebox{0pt}[\headheight][0pt]{\vbox{\hbox to\textwidth{\thepage\hfill \strut {\small БЕНИНГ В.Е., САВУШКИН В.А.}}\hrule}}}
\renewcommand{\@oddhead}{\raisebox{0pt}[\headheight][0pt]{\vbox{\hbox to\textwidth{{\small О ДЕФЕКТЕ ВЫБОРОЧНОЙ МЕДИАНЫ В СЛУЧАЕ ВЫБОРОК... }\hfill \strut\thepage}\hrule}}}
\makeatother

%%%%%%%%%%%%%%%%%%%%%%%%%%%%%%%%%%%%%%%%%%%%%%%%%%%%%%%%%%%%%%%%%%%%%%%%%%%%%%%%%%%%%%%%%%%%%%%%%%%%%%%%
% Заголовок
%%%%%%%%%%%%%%%%%%%%%%%%%%%%%%%%%%%%%%%%%%%%%%%%%%%%%%%%%%%%%%%%%%%%%%%%%%%%%%%%%%%%%%%%%%%%%%%%%%%%%%%%
\thispagestyle{plain}
УДК 510.676, 519.7
\begin{center}
{\bf О ДЕФЕКТЕ ВЫБОРОЧНОЙ МЕДИАНЫ В СЛУЧАЕ ВЫБОРОК СЛУЧАЙНОГО ОБЪЕМА\footnote{Работа выполнена при финансовой поддержке РФФИ (проект № 15 - 07 - 02652).}}
\vspace{4mm}\par
{\bf Бенинг В.Е.$^{*,**}$, Савушкин В.А.$^{***}$}\\
$^{*}$МГУ имени М.В. Ломоносова, г. Москва\\
$^{**}$Институт проблем информатики РАН, г. Москва\\
$^{***}$Международный университет природы,\\ общества и человека <<Дубна>>, г. Дубна 
\end{center}
\vspace{2mm}\par

\begin{center}
\renewcommand{\arraystretch}{0}
\begin{tabular}{c}
\hline
\rule{0pt}{2mm}\\
\small\it
Поступила в редакцию 28.04.2016,
после переработки 15.05.2016.
\\
\rule{0pt}{2mm}\\
\hline
\end{tabular}
\end{center}

\begin{quote}
В работе доказаны теоремы, позволяющие находить асимптотический дефект выборочной медианы, основанной на выборках случайного объема. Это делает возможным сравнивать в терминах добавочного числа наблюдений качество выборочной медианы, основанной на выборках случайного и неслучайного объемов. Рассмотрены случаи биномиального и распределения Пуассона.
\end{quote}

\begin{quote}
{\bf Ключевые слова:} выборочная медиана, асимптотический дефект, выборка случайного объема, биномиальное и распределение Пуассона.
\end{quote}
{\small {\it Вестник ТвГУ. Серия: Прикладная математика. 2016. № 2. С.~\pages.}}
\vspace{5mm}


%%%%%%%%%%%%%%%%%%%%%%%%%%%%%%%%%%%%%%%%%%%%%%%%%%%%%%%%%%%%%%%%%%%%%%%%%%%%%%%%%%%%%%%%%%%%%%%%%%%%%%%%
% Статья
%%%%%%%%%%%%%%%%%%%%%%%%%%%%%%%%%%%%%%%%%%%%%%%%%%%%%%%%%%%%%%%%%%%%%%%%%%%%%%%%%%%%%%%%%%%%%%%%%%%%%%%%

\firstheader{Введение}

Здесь идет ваше введение.

\header{Часть 1}

Здесь идет текст первой главы. Здесь идет текст первой главы. Здесь идет текст первой главы. 
Здесь идет текст первой главы. Здесь идет текст первой главы. Здесь идет текст первой главы. 
Здесь идет текст первой главы. Здесь идет текст первой главы. Здесь идет текст первой главы. 
Здесь идет текxст первой главы. Здесь идет текст первой главы. Здесь идет текст первой главы. 
Здесь идет текст первой главы. Здесь идет текст первой главы. Здесь идет текст первой главы. 
Здесь идет текст первой главы. Здесь идет текст первой главы. Здесь идет текст первой главы. 
Сслыка на литературу \cite{author:ref1}


\header{Часть 2}

Здесь идет текст второй главы. Здесь идет текст второй главы. Здесь идет текст второй главы.
Здесь идет текст второй главы. Здесь идет текст второй главы. Здесь идет текст второй главы.
Здесь идет текст второй главы. Здесь идет текст второй главы. Здесь идет текст второй главы.
Здесь идет текст второй главы. Здесь идет текст второй главы. Здесь идет текст второй главы.
\begin{definition}
Возможностная мера $\pi:\Gamma\rightarrow E^1$ -- есть функция множества, обладающая следующими свойствами:
\begin{enumerate}
\item $\pi(\oslash) = 0$, $\pi(\Gamma) = 1$,
\item $\displaystyle\pi\left(\bigcup_{i\in I}A_i\right) = \sup_{i\in I}\pi(A_i), \forall A_i\in {\mathbb P}(\Gamma), \forall I$.
\end{enumerate}
\end{definition}

Здесь идет текст второй главы. Здесь идет текст второй главы. Здесь идет текст второй главы.
Здесь идет текст второй главы. Здесь идет текст второй главы. Здесь идет текст второй главы.
\begin{lemma}
Пусть $a(\gamma)$ и $b(\gamma)$ -- взаимно $T_W$-связанные нечеткие величины, характеризующиеся квазивогнутыми полунепрерывными сверху строго унимодальными функциями распределения $\mu_a(x)$ и $\mu_b(x)$. Тогда возможностное неравенство
$$
\pi\{a(\gamma) = b(\gamma)\} >= \alpha
$$
эквивалентно следующей совокупности систем детерминированных неравенств:
$$
\left[\begin{array}{l}
a^-(\alpha) \leq \hat b \leq a^+(\alpha),\\
b^-(\alpha) \leq \hat a \leq b^+(\alpha),
\end{array}\right.
$$
где $a^-(\alpha)$ и $b^-(\alpha)$ -- левые границы, $a^+(\alpha)$ и $b^+(\alpha)$ -- правые границы $\alpha$-уровневых множеств, $\hat a$ и $\hat b$ -- модальные значения нечетких величин $a(\gamma)$ и $b(\gamma)$, соответственно.
\end{lemma}
\begin{remark}
Совокупность неравенств понимается в обычном смысле -- как дизъюнкция неравенств, в которой хотя бы одно должно быть истинным.
\end{remark}
\begin{proof}
Доказательство очевидно.
\end{proof}
Здесь идет текст второй главы. Здесь идет текст второй главы. Здесь идет текст второй главы.
Сслыка на литературу \cite{author:ref2}
\begin{theorem}
Условие теоремы
\end{theorem}
\begin{proof}
Доказательство теоремы.
\end{proof}
Здесь идет текст второй главы. Здесь идет текст второй главы. Здесь идет текст второй главы.
Здесь идет текст второй главы. Здесь идет текст второй главы. Здесь идет текст второй главы.
\begin{example}
Если в качестве левой и правой форм взять кусочно-линейную функцию $L(x) = R(x) = \max\{0, 1 - x\},\;x\geq 0$,
то для случая нечетких чисел мы получаем так называемые {\it триангулярные нечеткие числа}.
\end{example}
Здесь идет текст второй главы. Здесь идет текст второй главы. Здесь идет текст второй главы.


\subheader{Часть 2.1}

Здесь идет текст третьей главы. Здесь идет текст третьей главы. Здесь идет текст третьей главы.
Здесь идет текст третьей главы. Здесь идет текст третьей главы. Здесь идет текст третьей главы.
Здесь идет текст третьей главы. Здесь идет текст третьей главы. Здесь идет текст третьей главы.
Здесь идет текст третьей главы. Здесь идет текст третьей главы. Здесь идет текст третьей главы.
Здесь идет текст третьей главы. Здесь идет текст третьей главы. Здесь идет текст третьей главы.
Здесь идет текст третьей главы. Здесь идет текст третьей главы. Здесь идет текст третьей главы.
Сслыка на литературу \cite{author:ref3}

\subheader{Часть 2.2}

Здесь идет текст третьей главы. Здесь идет текст третьей главы. Здесь идет текст третьей главы.
Здесь идет текст третьей главы. Здесь идет текст третьей главы. Здесь идет текст третьей главы.
Здесь идет текст третьей главы. Здесь идет текст третьей главы. Здесь идет текст третьей главы.
%\begin{figure}[htp]
%\centering
%\includegraphics[width=0.8\textwidth]{image.pdf}
%\it
%\caption{Схема нагружения тела с включением} \label{f1}
%\end{figure}
Здесь идет текст третьей главы. Здесь идет текст третьей главы. Здесь идет текст третьей главы.
Здесь идет текст третьей главы. Здесь идет текст третьей главы. Здесь идет текст третьей главы.
Здесь идет текст третьей главы. Здесь идет текст третьей главы. Здесь идет текст третьей главы.
Сслыка на литературу \cite{author:ref3}

\header{Часть 3}

Здесь идет текст третьей главы. Здесь идет текст третьей главы. Здесь идет текст третьей главы.
Здесь идет текст третьей главы. Здесь идет текст третьей главы. Здесь идет текст третьей главы.
Здесь идет текст третьей главы. Здесь идет текст третьей главы. Здесь идет текст третьей главы.
Здесь идет текст третьей главы. Здесь идет текст третьей главы. Здесь идет текст третьей главы.
Здесь идет текст третьей главы. Здесь идет текст третьей главы. Здесь идет текст третьей главы.
Здесь идет текст третьей главы. Здесь идет текст третьей главы. Здесь идет текст третьей главы.
Сслыка на литературу \cite{author:ref3}

\header{Заключение}

В работе рассмотрен случай статистического оценивания неизвестного параметра сдвига, основанный на выборочной медиане, построенной по выборкам случайного объема. С помощью понятия дефекта проведено асимптотическое сравнение качества такой оценки. Получены явные формулы для асимптотического дефекта, имеющего смысл необходимого добавочного числа наблюдений. Подробно рассмотрен случай, когда случайный объем выборки имеет биномиальное, пуассоновское и трехточечное симметричное распределения.


\bigskip\bigskip\def\refname{\centerline{Список литературы}}
\begin{thebibliography}{99}
\bibitem{author:ref1} Hodges J.L., Lehmann E.L. Deficiency // Annals of Mathematical Statistics. 1970. Vol. 41, № 5. Pp. 783--801.
\bibitem{author:ref2} Крамер Г. Математические методы статистики. М.: Мир, 1976. 648 с.
\bibitem{author:ref3} Леман Э. Теория точечного оценивания. М.: Наука, ФизМатЛит, 1991. 444 с.
\bibitem{} Bening V.E. Asymptotic theory of testing statistical hypotheses: efficient statistics, optimality, power loss, and deficiency. Utrecht: VSP, 2000. 277 p.
\bibitem{} Бенинг В.Е. О дефекте некоторых оценок, основанных на выборках случайного объема // Вестник ТвГУ. Серия: Прикладная математика. 2015. № 1. С.~5--14.
\bibitem{} Бурнашев М.В. Асимптотические разложения для медианной оценки параметра // Теория вероятностей и ее применения. 1996. Т. 41, № 4. С. 738--753.
\bibitem{} Гнеденко Б.В. Об оценке неизвестных параметров распределения при случайном числе независимых наблюдений // Труды Тбилисского Математического Института. 1989. Т. 92. С. 147--150.
\bibitem{} Lehmann E.L. Elements of Large -- Sample Theory. Springer, 1999. 631 p.
\bibitem{} Бенинг В.Е. Об асимптотическом поведении обратных моментов некоторых случайных величин // Вестник ТвГУ. Серия: Прикладная математика. 2015. № 2. С. 47--65.
\bibitem{} Бенинг В.Е., Галиева Н.К., Королев В.Ю. Асимптотические разложения для функций распределения статистик, построенных по выборкам случайного объема // Информатика и ее применения. 2013. Т. 7, № 2. С. 75--91.
\bibitem{} Бенинг В.Е., Савушкин В.А. Об аппроксимации распределений статистик, основанных на выборках случайного объема // Вестник ТвГУ. Серия: Прикладная математика. 2014. № 1. С. 91--111.
\bibitem{} Бенинг В.Е., Королев В.Ю. Об использовании распределения Стьдента в задачах теории вероятностей и математической статистики // Теория вероятностей и ее применения. 2008. Т. 2, № 2. С. 19--34.
\bibitem{} Бенинг В.Е., Королев В.Ю., Соколов И.А., Шоргин С.Я. Рандомизированные модели и методы теории надежности информационных и технических систем. М.: ТОРУС ПРЕСС, 2007.
\bibitem{9} Бенинг В.Е., Королев В.Ю. Некоторые статистические задачи, связанные с распределением Лапласа // Информатика и ее применения.
2008. Т. 2, № 2. С. 19--34.
\bibitem{11} Лямин О.О. О скорости сходимости распределений некоторых статистик к распределению Лапласа и Стьюдента // Вестник Московского университета. Серия 15: Вычислительная математика и кибернетика. 2011. № 1. С. 39--47.
\bibitem{13} Двайт Г.Б. Таблицы интегралов и другие математические формулы. М.: Наука, 1977.
\bibitem{26} Wilks S.S. Recurrence of extreme observations // Journal of American Mathematical Society. 1959. Vol. 1, № 1. Pp. 106--112.
\bibitem{27} Невзоров В.Б. Рекорды. Математическая теория. М.: Фазис, 2000.
\bibitem{} Бенинг В.Е., Савушкин В.А. Асимптотические разложения для функции распределения выборочной медианы в случае выборок случайного объема // Вестник ТвГУ. Серия: Прикладная математика. 2015. № 4. С. 39--54.
\end{thebibliography}

\bigskip\medskip{\centerline{\bf Библиографическая ссылка}}\medskip
{Бенинг В.Е., Савушкин В.А. О дефекте выборочной медианы в случае выборок случайного объема // Вестник ТвГУ. Серия: Прикладная математика. 2016. №~2. С.~\pages.}

\bigskip\medskip{\centerline{\bf Сведения об авторах}}
\begin{enumerate}[1.]
\item {\bf Бенинг Владимир Евгеньевич}\\
профессор кафедры математической статистики факультета вычислительной математики и кибернетики Московского 
государственного университета им. М.В. Ломоносова, старший научный сотрудник ИПИ РАН.

\vspace{1mm}
{\it Россия, 119992, г. Москва, ГСП-1, Воробьевы горы, МГУ им. М.В. Ломоносова. E-mail: ivanov@yandex.ru.} 
\item {\bf Савушкин Владислав Андреевич}\\
ассистент кафедры прикладной математики и информатики факультета естественных  и инженерных наук государственного университета Природы, Общества и Человека <<Дубна>>.

\vspace{1mm}
{\it Россия, 141982, Московская область, г. Дубна, улица Университетская, д. 19, Университет Дубна. E-mail: sidorov@mail.ru.} 
\end{enumerate} 

\newpage
\thispagestyle{plain}
%%%%%%%%%%%%%%%%%%%%%%%%%%
\begin{center}
{\bf ON THE DEFICIENCY OF SAMPLE MEDIANE\\ BASED ON THE SAMPLE WITH RANDOM SIZE}
\vspace{4mm}\par
{\bf Bening Vladimir Evgenyevich}\\
Professor at Mathematical Statistics department,\\ Lomonosov Moscow State University\\
{\it Russia, 119992, Moscow, GSP-1, Vorobyovi gory, Lomonosov MSU.\\ E-mail: ivanov@yandex.ru}\\ \vspace{2mm}
{\bf Savushkin Vladislav Andreevich}\\
Assistant at Applied Mathematics and Computer Science department,\\ International University of Nature, Society and Man <<Dubna>>\\
{\it Russia, 141980, Moscow region, Dubna, 19 Universitetskaya str.,\\ International University of Nature, Society and Man <<Dubna>>.\\ E-mail: sidorov@mail.ru}  
\end{center}
\vspace{2mm}\par

\begin{center}
\renewcommand{\arraystretch}{0}
\begin{tabular}{c}
\hline
\rule{0pt}{2mm}\\
\small\it
Received 28.04.2016,
revised 15.05.2016.
\\
\rule{0pt}{2mm}\\
\hline
\end{tabular}
\end{center}

\begin{quote}
In the paper general theorems concerning the asymptotic deficiencies of sample mediane based on the sample of random size are proved.
\end{quote}
\begin{quote}
{\bf Keywords:} sample mediane, asymptotic deficiency, sample with random size, binomial and Poisson distributions.
\end{quote}
\vspace{5mm}

\bigskip{\centerline{\bf Bibliographic citation}}\medskip
{Bening V.E., Savushkin V.A. On the deficiency of sample mediane based on the sample with random size. {\it Vestnik TvGU. Seriya: Prikladnaya Matematika} [Herald of Tver State University. Series: Applied Mathematics], 2016, no.~2, pp. \pages.} (in Russian) 


\bigskip\bigskip\def\refname{\centerline{References}}
\begin{thebibliography}{99}
\bibitem{} Hodges J.L., Lehmann E.L. Deficiency. {\it Annals of Mathematical Statistics}, 1970, vol. 41(5), pp. 783--801.
\bibitem{} Kramer G. {\it Matematicheskie Metody Statistiki} [Mathematical Methods of Statistics]. Mir Publ., Moscow, 1976. 648 p. (in Russian)
\bibitem{} Leman E. {\it Teoriya Tochechnogo Ocenivaniya} [Theory of Point Estimation]. Nauka, FizMatLit Publ., Moscow, 1991. 444 p. (in Russian)
\bibitem{} Bening V.E. {\it Asymptotic theory of testing statistical hypotheses: efficient statistics, optimality, power loss, and deficiency}. VSP, Utrecht, 2000. 277 p.
\bibitem{} Bening V.E. On deficiencies of some estimators based on samples of random size. {\it Vestnik TvGU. Seriya: Prikladnaya Matematika} [Herald of Tver State University. Series: Applied Mathematics], 2015, no.~1, pp. 5--14. (in Russian)
\bibitem{} Burnashev M.V. The asymptotic expansions for median estimate parameter. {\it Teoriya Veroyatnostei i ee Primeneniya} [Probability Theory and its Applications], 1996, vol. 41(4), pp. 738--753. (in Russian)
\bibitem{} Gnedenko B.V. An estimate of the distribution of the unknown parameters with a random number of independent observations. {\it Trudy Tbilisskogo Matematicheskogo Instituta} [Proceedings of the Tbilisi Institute of Mathematics], 1989, vol. 92, pp. 147--150. (in Russian)
\bibitem{} Lehmann E.L. {\it Elements of Large -- Sample Theory}. Springer, 1999. 631 p.
\bibitem{} Bening V.E. Asymptotic behavior of inverse moments of some discrete random variables. {\it Vestnik TvGU. Seriya: Prikladnaya Matematika} [Herald of Tver State University. Series: Applied Mathematics], 2015, no.~2, pp. 47--65. (in Russian)
\bibitem{} Bening V.E., Galieva N.K., Korolev V.Yu. Asymptotic expansions for the distribution of the statistics functions, constructed from samples with random sizes. {\it Informatika i ee Primeneniya} [Informatics and its Applications], 2013, vol. 7(2), pp. 75--91. (in Russian)
\bibitem{} Bening V.E., Savushkin V.A. On the approximation of the distributions of statistics based on the samples with random size. {\it Vestnik TvGU. Seriya: Prikladnaya Matematika} [Herald of Tver State University. Series: Applied Mathematics], 2014, no.~1, pp. 91--111. (in Russian)
\bibitem{} Bening V.E., Korolev V.Yu. On the use of Student distribution in the problems of probability theory and mathematical statistics. {\it Teoriya Veroyatnostei i ee Primeneniya} [Probability Theory and its Applications], 2008, vol. 2(2), pp. 19--34. (in Russian)
\bibitem{} Bening V.E., Korolev V.Yu., Sokolov I.A., Shorgin S.Ya. {\it Randomizirovannye Modeli i Metody Teorii Nadezhnosti Informatsionnykh i Tekhnicheskikh Sistem} [Randomized Models and Methods of the Theory of Reliability of Information and Technical Systems]. TORUS PRESS Publ., Moscow, 2007. (in Russian)
\bibitem{9} Bening V.E., Korolev V.Yu. Some statistical problems related to the Laplace distribution. {\it Informatika i ee Primeneniya} [Informatics and its Applications], 2008, vol. 2(2), pp. 19--34. (in Russian)
\bibitem{11} Lyamin O.O. On the rate of convergence of the distributions of some statistics to the Laplace and Student distributions. {\it Vestnik Moskovskogo Universiteta. Seriya 15: Vychislitel'naya Matematika i Kibernetika} [Bulletin of Moscow University. Series 15: Computational Mathematics and Cybernetics], 2011, no. 1, pp. 39--47. (in Russian)
\bibitem{13} Dwait G.B. {\it Tablitsy Integralov i Drugie Matematicheskie Formuly} [Tables of Integrals and Other Mathematical Formulas]. Nauka Publ., Moscow, 1977. (in Russian)
\bibitem{26} Wilks S.S. Recurrence of extreme observations. {\it Journal of American Mathematical Society}, 1959, vol. 1(1), pp. 106--112.
\bibitem{27} Nevzorov V.B. {\it Rekordy. Matematicheskaya Teoriya} [Records. Mathematical Theory]. Fazis Publ., Moscow, 2000. (in Russian)
\bibitem{} Bening V.E., Savushkin V.A. Asymptotic expansions for the distribution function of sample median based on the sample with random size. {\it Vestnik TvGU. Seriya: Prikladnaya Matematika} [Herald of Tver State University. Series: Applied Mathematics], 2015, no.~4, pp. 39--54. (in Russian)
\end{thebibliography} 
\end{document}