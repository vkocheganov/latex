\documentclass[10pt]{beamer}

\usepackage{color}
\usepackage[T2A]{fontenc}
\usepackage[russian]{babel}
\usepackage{amsmath}
\newcommand{\No}{\textnumero}

\usepackage{beton}

\usepackage{tikz}
\usepackage[utf8]{inputenc}

\defbeamertemplate{footline}{centered page number}
{%
  \hspace*{\fill}%
  \usebeamercolor[fg]{page number in head/foot}%
  \usebeamerfont{page number in head/foot}%
  \insertpagenumber\,/\,\insertpresentationendpage%
  \hspace*{\fill}\vskip2pt%
}
\setbeamertemplate{footline}[centered page number]
%\mode<presentation>{
%\usetheme{Rochester}
%}
\newcommand{\backupbegin}{
   \newcounter{framenumberappendix}
   \setcounter{framenumberappendix}{\value{framenumber}}
}
\newcommand{\backupend}{
   \addtocounter{framenumberappendix}{-\value{framenumber}}
   \addtocounter{framenumber}{\value{framenumberappendix}} 
}
\mode<presentation>{
%  \usetheme{Madrid}
  \usetheme{AnnArbor}
  \usefonttheme{serif}
}
\makeatletter
\setbeamertemplate{footline}
{
  \leavevmode%
  \hbox{%
  \begin{beamercolorbox}[wd=.333333\paperwidth,ht=2.25ex,dp=1ex,center]{author in head/foot}%
    \usebeamerfont{author in head/foot}\insertshortauthor%~~\beamer@ifempty{\insertshortinstitute}{}{(\insertshortinstitute)}
  \end{beamercolorbox}%
  \begin{beamercolorbox}[wd=.333333\paperwidth,ht=2.25ex,dp=1ex,center]{title in head/foot}%
    \usebeamerfont{title in head/foot}\insertshorttitle
  \end{beamercolorbox}%
  \begin{beamercolorbox}[wd=.333333\paperwidth,ht=2.25ex,dp=1ex,right]{date in head/foot}%
    \usebeamerfont{date in head/foot}\insertshortdate{}\hspace*{2em}
    \insertframenumber{} / \inserttotalframenumber\hspace*{2ex} 
  \end{beamercolorbox}}%
  \vskip0pt%
}
\makeatother


\begin{document}

\title[ПОТОКИ ПЕРВИЧНЫХ ТРЕБОВАНИЙ...]{\normalsize \color{blue} ПОТОКИ ПЕРВИЧНЫХ ТРЕБОВАНИЙ В ТАНДЕМЕ СИСТЕМ ОБСЛУЖИВАНИЯ С ЦИКЛИЧЕСКИМ АЛГОРИТМОМ С ПРОДЛЕНИЕМ}

\author[В.М.~Кочеганов, А.В.~Зорин (ННГУ)]{\textbf{В.М.~Кочеганов}, А.В.~Зорин}
\institute[ННГУ]{\normalsize Нижегородский государственный университет\\
им.~Н.И.~Лобачевского}
\date[25-29.09.2017]{
XX Международная конференция \\ 
Распределенные Компьютерные и Телекоммуникационные Сети: Управление, Вычисление и Связь\\
25-29 сентября 2017 г., Москва
}

\begin{frame}
\titlepage
\end{frame}



\begin{frame}{Похожие исследования}
    \begin{itemize}
    \item \textbf{Yamada K., Lam T.N.} Simulation analysis of two adjacent traffic signals // Proceedings of the 17th winter simulation conference. --- New York: ACM, 1985. --- P.~454–464.
    \item \textbf{Afanasyeva L.~G., Bulinskaya E.~V. } Mathematical models of transport systems based on queueing theory~// Works of Moscow Institute of Physics and Technology. 2010. No.~4. P.6--21. 
\item \textbf{Zorine A.V.} Stability of a tandem of queueing systems with Bernoulli noninstantaneous transfer of
  customers~// Theory of Probability and Mathematical Statistics. 2012. V.~84. P.~173-188.
    \end{itemize}
\end{frame}


\begin{frame}{Тандем перекрестков}
  \begin{figure}[h]
    \centering
    \pgfimage[height=5cm]{Crossroads}
    \caption{Тандем перекрёстков}
    \label{VK:fig:1}
  \end{figure}
\end{frame} 


\begin{frame}{Постановка задачи. Управляющая СМО}
  \begin{figure}[h]
    \centering
    \pgfimage[height=6cm]{SystemScheme}
    \caption{Структурная схема рассматриваемой СМО}
    \label{VK:fig:2}
  \end{figure}
\end{frame} 

\begin{frame} {Выбор моментов наблюдения}
  \begin{figure}[h]
    \centering
    \pgfimage[height=2.5cm]{timings}
    \caption{Шкала моментов наблюдения}
    \label{VK:fig:3}
  \end{figure}
\end{frame}

   \begin{figure}[h]
    \centering
    \pgfimage[height=7cm]{GraphScheme3}
    \caption{Пример. Граф переходов.}
    \label{VK:fig:4}
  \end{figure}




\begin{frame}{Параметры системы}
\begin{itemize}
    \item 
$\color{blue}\lambda_1>0$, $\color{blue}\lambda_3>0$ --- интенсивности поступления групп требований по потокам  $\Pi_1$, $\Pi_3$ соответственно.
%Входные потоки $\Pi_1$ и $\Pi_3$ --- \textbf{неординарные пуассоновские потоки} групп требований с интенсивностями поступления групп требований по потоку $\lambda_1$ и $\lambda_3$ соотвественно.
  \item 
{\color{blue}Распределение числа заявок в группе} по потоку $\Pi_j$, $j \in \{1,3\}$, имеет производящую функцию:
\begin{equation}
f_j(z) = \sum_{\nu=1}^{\infty} p_{\nu}^{(j)} z ^{\nu}, \quad |z|<(1+\varepsilon), \varepsilon>0.
    \end{equation}
  \item 
$\color{blue}T^{(k,r)}>0$ --- неслучайное время нахождения обслуживающего устройства в состоянии $\Gamma^{(k,r)}$, $k\in \{0, 1, \ldots, d\}$, $r \in \{1, 2, \ldots, n_k\}$.  
  \item 
$\color{blue}\ell(k,r,j)\geqslant 0$ --- количество требований содержащихся в потоке насыщения $\Pi^{\text{нас}}_j$ в состоянии  $\Gamma^{(k,r)}$.
  \item 
$\color{blue}L > 0$ --- порог числа требований в очереди $O_3$, при превышении которого начинается обслуживание очереди $O_3$.
\end{itemize}

\end{frame}



\begin{frame}{Необходимые случайные величины}
  \begin{itemize}
    \item $\tau_i \in {\mathbb R}_+$, $i=0$, $1$, \ldots --- момент смены состояния
    обслуживающего устройства;
    \item $\eta_{j,i} \in Z_+$ --- число требований потока $\Pi_j$, поступивших за
    промежуток $(\tau_i, \tau_{i+1}]$;
    \item $\xi_{j,i} \in Z_+$ --- число требований потока насыщения $\Pi^{\mbox{\scriptsize{нас}}}_j$ на промежутке $(\tau_i, \tau_{i+1}]$;
    \item $\varkappa_{j,i}$ --- число требований в
    очереди $O_j$ в момент $\tau_i$;
  \item $\Gamma_i\in\Gamma$ --- состояние обслуживающего устройства в момент $\tau_i$;
  \item $\overline{\xi}_{j,i} \in Z_+$ --- число требований
    выходного потока $\Pi^{\mbox{\scriptsize{вых}}}_j$ на промежутке
    $(\tau_i, \tau_{i+1}]$.
  \end{itemize}
  $j=1$,  $2$, $3$, $4$.
\end{frame}



\begin{frame}{Функциональные соотношения}
Функционирование системы подчиняется следующим функциональным соотношениям:
\begin{align}
\overline{\xi}_{j,i}&=\min\{\varkappa_{j,i}+\eta_{j,i},\xi_{j,i}\}, \quad & j\in \{1,2,3\},\\
\varkappa_{j,i+1}&=\varkappa_{j,i}+\eta_{j,i}-\overline{\xi}_{j,i}, \quad & j\in \{1,2,3\},\\
\varkappa_{j,i+1}&=\max\{{0,\varkappa_{j,i}+\eta_{j,i}-\xi_{j,i}}\}, \quad & j\in \{1,2,3\},\\
\varkappa_{4,i+1}&=\varkappa_{4,i}+\eta_{4,i}-\eta_{2,i}, \quad &\\
\xi_{4,i} & = \varkappa_{4,i}, & \\
\eta_{4,i} & = \min\{ \varkappa_{1,i} + \eta_{1,i}, \xi_{1,i}\}.
\end{align}
\end{frame}


\begin{frame}{Аналитическое задание графа переходов}
  \begin{equation}
  \Gamma = \bigl( \bigcup_{k=1}^d C_k \bigr) \bigcup \{\Gamma^{(0,1)}, \Gamma^{(0,2)}, \ldots, \Gamma^{(0,n_0)}\}, \quad C_k = C_k^{\mathrm{I}} \cup C_k^{\mathrm{O}}  \cup C_k^{\mathrm{N}}.
  \end{equation}
  \begin{equation}
h(\Gamma^{(k,r)},y) = 
\begin{cases}
\Gamma^{(k,r\oplus_k 1)},& \quad \text{ если } \Gamma^{(k,r)}\in C_k\setminus C_k^{\mathrm{O}};\\
\Gamma^{(k,r\oplus_k 1)},& \quad \text{ если } \Gamma^{(k,r)}\in C_k^{\mathrm{O}} \text{ и } y>L;\\
\Gamma^{(0,h_1(\Gamma^{(k,r)}))},& \quad \text{ если } \Gamma^{(k,r)}\in C_k^{\mathrm{O}} \text{ и } y\leqslant L;\\
\Gamma^{(0,h_2(r))},& \quad \text{ если } k=0 \text{ и } y\leqslant L;\\
h_3(r),& \quad \text{ если } k=0 \text{ и } y > L.
\end{cases}
\end{equation}
где 
$$h_1(\cdot)\colon \bigcup_{k=1}^d C_k^{\mathrm{O}}\to N_0, \quad h_2(\cdot)\colon N_0\to N_0, \quad h_3(\cdot)\colon N_0 \to\bigcup_{k=1}^d C_k^{\mathrm{I}},$$ и $N_0=\{1,2, \ldots, n_0\}$.

Тогда 
\begin{equation}
\Gamma_{i+1} = h(\Gamma_i, \varkappa_{3,i}).
\end{equation}
\end{frame}

\begin{frame}{Потоки первичных требований}
Рассмотрим последовательность
\begin{equation}
\label{eq:theMC}
\{(\Gamma_i, \varkappa_{1,i},\varkappa_{3,i}); i =0, 1, \ldots\},
\end{equation}
 
\begin{block}
    {\bf Теорема 1.} {\it 
    Пусть $$\Gamma_0=\Gamma^{(k_0,r_0)} \in \Gamma,  (\varkappa_{1,0}, \varkappa_{3,0})=(x_1,x_3)\in \mathbb{Z}_+^2$$ фиксированы. 
    
    Тогда последовательность $$\{(\Gamma_i, \varkappa_{1,i},\varkappa_{3,i}); i =0, 1, \ldots\}$$ является счетной цепью Маркова.}
\end{block}
\end{frame}


\begin{frame}[allowframebreaks]{Потоки первичных требований. Обозначения}
Определим для  $\gamma \in \Gamma$ и $x_3 \in Z_+$ величины
\begin{equation}
Q_{1,i}(\gamma,x_1,x_3) = {\mathbf P}(\{\omega\colon \Gamma_{i}(\omega)=\gamma, \varkappa_{1,i}(\omega)=x_1, \varkappa_{3,i}(\omega)=x_3\}).
\label{distr}
\end{equation}
и для $\Gamma^{(k,r)}\in \Gamma$
\begin{align}
    &q^{(1,i)}(k,r, v_1) = v_1^{-\ell(k,r,1)}\sum_{w=0}^{\infty} \varphi_1(w,T^{(k,r)})v_1^w,\\
&q^{(3,i)}(k,r, v_3) = q_{k,r} (v_3) = v_3^{-\ell(k,r,3)}\sum_{w=0}^{\infty} \varphi_3(w,T^{(k,r)})v_3^w,
\end{align}
Тогда определим частичные производящие функции распределения \eqref{distr} следующим образом
\begin{align}
&\mathfrak{M}^{(1,i)}(k,r,v_1,v_3) = \sum_{w_1=0}^{\infty}\sum_{w_3=0}^{\infty} Q_{1,i}(\Gamma^{(k,r)},w_1,w_3) v_1^{w_1} v_3^{w_3}
\end{align}

\framebreak
Пусть $\Gamma^{(k,r)} \in C_{k}^{\mathrm{I}}$:
\begin{multline}
    g^{(1,i)}(k,r,v_1,v_3) =\sum_{x_3=L+1}^{\infty} \sum_{x_1=\ell(k,r,1)+1}^{\infty}  [  Q_{1,i}(\Gamma^{(k, r\ominus_{k}1)},x_1, x_3) +\\+Q_{1,i}(\Gamma^{(0, r_2)},x_1, x_3)]v_1^{x_1} v_3^{ x_3}, \quad 
\end{multline}
Пусть $\Gamma^{(k,r)} \in C_{k}^{\mathrm{O}} \cup C_{k}^{\mathrm{N}}$
\begin{multline}
g^{(1,i)}(k,r,v_1,v_3) =\\
    =\!\!\sum_{x_3=\ell(k,r,3)+1}^{\infty} \sum_{x_1=\ell(k,r,1)+1}^{\infty}
    \!\! Q_{1,i}(\Gamma^{(k, r\ominus_{k}1)}\!,x_1, x_3) v_1^{x_1} v_3^{ x_3}, \; 
\end{multline}

\end{frame}

\begin{frame}[allowframebreaks]{Основные результаты}
\begin{block}{\bf Теорема 2.}
Имеют место следующие рекуррентные соотношения для частичных производящих функций:
\begin{enumerate}
\item $ \Gamma^{(0,\tilde{r})} \in \Gamma$, $\tilde{r} = \overline{1,n_0}$ 
$$
\mathfrak{M}^{(1,i+1)}(0,\tilde{r},v) = \alpha^{(1,i)}(0,\tilde{r},v_1,v_3);
$$
\item $\Gamma^{(\tilde{k},\tilde{r})} \in C_{\tilde{k}}^{\mathrm{I}} \cup C_{\tilde{k}}^{\mathrm{O}} \cup C_{\tilde{k}}^{\mathrm{N}}$
\begin{multline*}
\mathfrak{M}^{(1,i+1)}(\tilde{k},\tilde{r},v) = q^{(1,i)}(\tilde{k},\tilde{r},v_1) q^{(3,i)}(\tilde{k},\tilde{r},v_3) \times g^{(1,i)}(\tilde{k},\tilde{r},v_1,v_3)
     +\\+ \alpha^{(1,i)}(\tilde{k},\tilde{r},v_1,v_3);
\end{multline*}
\end{enumerate}
\label{theorem:gen}
\end{block}

\framebreak
\begin{block}{\bf Теорема 3.}
Для того, чтобы марковская цепь $$\{(\Gamma_, \varkappa_{1,i},\varkappa_{3,i}); i =0, 1, \ldots\},$$ имела стационарное распределение $Q_{1,i}(\gamma,x_1,x_3)$, $(\gamma,x_1,x_3)\in \Gamma \times {\mathbb Z}_+^2$, необходимо выполнение следующего неравенства
$$
\min_{\substack{k=\overline{1,d}\\ j=1,3}} { \frac{\sum_{r = 1}^{n_k} \ell(k,r,j) }{\lambda_j f_j'(1) \sum_{r=1}^{n_k} T^{(k,r)} }}>1.
$$
\label{theorem:nec}
\end{block}
\end{frame}

\begin{frame}
\Huge{\centerline{\color{blue} Спасибо за внимание!}}
\end{frame}

\appendix
\section{Приложение}
\backupbegin

\begin{frame}[allowframebreaks]{Потоки первичных требований. Обозначения}
Пусть $\Gamma^{(k,r)}\in \Gamma$ и $x_3 \in Z_+$. Обозначим 
$$
{\mathbb H}_{-1}(\Gamma^{(k,r)}, x_3) = \{\gamma \in \Gamma \colon h(\gamma, x_3) = \Gamma^{(k,r)}\}.
$$
Вид отображения $h(\cdot,\cdot)$ позволяет записать явный вид множества ${\mathbb H}_{-1}(\Gamma^{(k,r)}, x_3)$:
\begin{equation*}
H_{-1}(\Gamma^{(k,r)}, x_3) = 
\begin{cases}
\bigl\{\Gamma^{(k_1,r_1)}, \Gamma^{(0,r\ominus_0 1)}\bigr\},&  \text{ если  $(k=0) \wedge (x_3 \leqslant L)$,}\\
\bigl\{\Gamma^{(k,r\ominus_k 1)}, \Gamma^{(0,r_2)}\bigr\},&  \text{ если  $(\Gamma^{(k,r)}\in C_k^{\mathrm{I}})
  \wedge (x_3>L)$,}\\ 
\bigl\{\Gamma^{(k,r\ominus_k 1)}\bigr\},&  \text{ если  $(\Gamma^{(k,r)}\in C_k^{\mathrm{O}}) \vee (\Gamma^{(k,r)}\in C_k^{\mathrm{N}})$;}\\
\varnothing,&  \text{ если  $(k = 0)\wedge  (x_3>L)$}\\
 & \qquad \text{ или $(\Gamma^{(k,r)}\in C_k^{\mathrm{I}}) \wedge (x_3\leqslant L)$}
\end{cases}
\end{equation*}
где $h_1(\Gamma^{(k_1,r_1)})=r$ и $h_3(r_2)=\Gamma^{(k,r)}$.
\framebreak

\end{frame}

\begin{frame}{Кодирование информации}
Пусть $Z_+$ --- множество целых неотрицательных чисел
  \begin{itemize}
  \item $\{e^{(1)}\}$ --- множество состояний \textbf{внешней среды} (одно состояние);
  \item $Z^4_+$ --- множество состояний \textbf{входных полюсов};
  \item $Z^4_+$ --- множество состояний \textbf{выходных полюсов};
 \item $\Gamma=\{\Gamma^{(k,r)} \colon k=0,1,\ldots,d; r=1,2,\ldots n_k\}$ --- множество состояний \textbf{внутренней памяти};
   \item $Z^4_+$ --- множество состояний \textbf{внешней памяти};
   \item $\{r^{(1)}\}$ --- множество состояний \textbf{устройства по переработке информации во внешней памяти} (одно состояние)
   \item граф переходов (будет описан ниже) описывает устройство по переработке информации во внутренней памяти
   \end{itemize}
\end{frame}

\begin{frame}[allowframebreaks]{Свойства условных распределений}
Определим функции $\varphi_j(\cdot,\cdot)$, $j\in \{1,3\}$, и $\psi(\cdot, \cdot, \cdot)$ из разложений:
\begin{equation*}
\sum_{\nu=0}^{\infty} z^\nu\varphi_j(\nu,t) = \exp\{\lambda_j t (f_j(z)-1)\}, \quad \psi(k;y,u)=C_y^k u^k (1-u)^{y-k}.	
\end{equation*}

Пусть $a=(a_1, a_2, a_3, a_4) \in \mathbb{Z}_+^4$ и $x=(x_1, x_2, x_3, x_4) \in \mathbb{Z}_+^4$.
%\begin{block}{Предположение 1}

Тогда вероятность $\varphi(a,k,r,x)$ одновременного выполнения равенств $\eta_{1,i}=a_1$, $\eta_{2,i}=a_2$, $\eta_{3,i}=a_3$, $\eta_{4,i}=a_4$ при условии  $\nu_i=(\Gamma{(k,r)}; x)$ есть 
\begin{equation}
\!\!\varphi_1(a_1,h_T(\Gamma^{({k},{r})},x_3)) \times \psi(a_2,x_4, p_{\tilde{k},\tilde{r}}) \times \varphi_3(a_3,h_T(\Gamma^{({k},{r})},x_3))
\times \delta_{a_4,\min{\{\ell(\tilde{k},\tilde{r},1), x_1+a_1}\}},
\end{equation}
%\end{block}
где
\begin{equation*}
\Gamma^{(\tilde{k},\tilde{r})}=h(\Gamma^{(k,r)},x_3), \quad \delta_{i,j}=\begin{cases} 1, \quad \text{ если }i=j\\0, \quad \text{ если } i\neq j,
\end{cases}.
\end{equation*}
и 
$$
T_{i+1}=h_T(\Gamma_i,\varkappa_{3,i})= T^{(k,r)},\quad  \Gamma^{(k,r)}=\Gamma_{i+1}=h(\Gamma_i,\varkappa_{3,i}).
$$
\framebreak

Пусть $b=(b_1, b_2, b_3, b_4) \in \mathbb{Z}_+^4$. 

Тогда вероятность $\zeta(b, k, r, x)$ одновременного выполнения равенств $\xi_{1,i}=b_1$, $\xi_{2,i}=b_2$, $\xi_{3,i}=b_3$, $\xi_{4,i}=b_4$ при фиксированном значении метки $\nu_i=(\Gamma{(k,r)}; x)$ есть
\begin{equation}
\delta_{b_1,\ell(\tilde{k},\tilde{r},1)} \times \delta_{b_2,\ell(\tilde{k},\tilde{r},2)} \times 
\delta_{b_3,\ell(\tilde{k},\tilde{r},3)} \times \delta_{b_4,x_4}.
\end{equation}
где $\tilde{k}$ и $\tilde{r}$ такие, что $\Gamma^{(\tilde{k},\tilde{r})}=h(\Gamma^{(k,r)},x_3)$.
\end{frame}
\begin{frame}[allowframebreaks]{Полученные ранее результаты}

\begin{block}
    {\bf Теорема 1.} {\it 
    Пусть $$\Gamma_0=\Gamma^{(k_0,r_0)} \in \Gamma,  \varkappa_{3,0}=x_3\in \mathbb{Z}_+$$ фиксированы. 
    
    Тогда последовательность $$\{(\Gamma_i, \varkappa_{3,i}); i \geqslant 0\}$$ является счетной цепью Маркова.}
\end{block}
\framebreak
\begin{block}
    {\bf Теорема 2.}{\it Пусть 
    $$x_3, \tilde{x}_3~\in \mathbb{Z}_+, \quad \Gamma^{(k,r)}\in \Gamma, \Gamma^{(\tilde{k},\tilde{r})}= h(\Gamma^{(k,r)},x_3).$$ Тогда условная вероятность 
$${\mathbf P}(\{\Gamma_{i+1}=\Gamma^{(\tilde{k},\tilde{r})},\varkappa_{3,i+1}=\tilde{x}_3\}|\{\Gamma_{i}=\Gamma^{(k,r)},\varkappa_{3,i}=x_3\})$$ равна
$$\delta_{\tilde{x}_3,0}\!\sum_{a=0}^{\ell(\tilde{k},\tilde{r},3)-x} \varphi_3(a,h_T(\Gamma^{(k,r)},x_3)) + (1-\delta_{\tilde{x}_3,0}) \varphi_3(\tilde{x}_3 + \ell(\tilde{k},\tilde{r},3)-x_3,h_T(\Gamma^{(k,r)},x_3)).$$}
\end{block}
\framebreak
\begin{block}
    {\bf Теорема 3.}{\it Пусть для $r=\overline{1,n_0}$ определено множество $$S^3_{0,r} =\bigl\{(\Gamma^{(0,r)},x_3)  \colon x_3\in Z_+,  L \geqslant x_3 > L - \max\bigl\{\sum_{t=0}^{n_k} \ell_{k,t,3}\colon k=\overline{1,d}\bigr\}\,\bigr\}
$$ и  для $k=\overline{1,d}$, $r=\overline{1,n_k}$ обозначено $$S^3_{k,r} =  \{(\Gamma^{(k,r)},x_3) \colon x_3\in Z_+, x_3 > L - \sum_{t=0}^{r-1} \ell_{k,t,3}\}.$$ Тогда множество существенных состояний марковской цепи $\{(\Gamma_i, \varkappa_{3,i}); i \geqslant 0\}$ есть $$S=\bigcup_{k=0}^d \bigl(\bigcup_{r=1}^{n_k} S^3_{k,r}\bigr).$$}
\end{block}

\end{frame}
\backupend\end{document}
