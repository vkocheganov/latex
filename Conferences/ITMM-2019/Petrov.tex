%Äàòà ïîñëåäíåãî èçìåíåíèÿ ôàéëà 26-05-2017
%\nonstopmode
\documentclass[12pt]{book}
%\usepackage[cp1251]{inputenc}
\usepackage{amsthm}
\usepackage{amsmath}
\usepackage{amssymb}
\usepackage{amsfonts}
\usepackage{mathtext}
\usepackage{mathrsfs}
\usepackage{cite}
\ifx\pdfoutput\undefined
\usepackage{graphicx}
\else
\usepackage[pdftex]{graphicx}
\fi

\usepackage{xypic}
\usepackage{epic}
%\usepackage{urwcyr}
%\usepackage{bng}
%\usepackage{pscyr}
%\usepackage[english,russian]{babel}
\usepackage[russian,english]{babel}
\usepackage[T2A]{fontenc}
\usepackage[utf8x]{inputenc}

\usepackage{multicol}
\premulticols=5pt \postmulticols=0mm \multicolsep=0mm
\usepackage{longtable}
%
%
%\usepackage{array}
%\usepackage{multirow}
%\renewcommand\multirowsetup{\centering}
%\usepackage{dcolumn}
\usepackage{graphicx}

% LAYOUT ----------------------------------------------------------
\usepackage{geometry}% Меняем поля страницы
\geometry{left=3.5cm}% левое поле
\geometry{right=1.5cm}% правое поле
\geometry{top=3cm}% верхнее поле
\geometry{bottom=2cm}% нижнее поле
\headsep=5mm%расстояние от верхнего колонтитула до текста
%
%Колонтитулы-----------------------------------------------------------
\usepackage{fancyhdr}%загрузим пакет
\pagestyle{fancy}%применим колонтитул
\fancyfoot{}\fancyhead{}% очистим футер (снизу) и хидер (сверху) на всякий случай
%\fancyhead[CE,CO]{\thepage}%номер страницы снизу по центру
\fancyfoot[LE,RO]{\thepage}%E - нечетные,  O - четные,  L - слева, R - справа, C - по центру
%\fancyhead[CO]{}%{текст-центр-нечетные}
%\fancyhead[LO]{Левый колонтитул}%
%\fancyhead[RE]{Правый колонтитул}%

%------------------------------------------------------------------

\makeatletter
\renewcommand{\@biblabel}[1]{#1.}

\renewcommand{\thesection}{\arabic{section}.}

\renewcommand{\section}{\@startsection
                                {section} %name
                                {1}       %level
                                {\z@}     %indent
                                {12pt}    %before skip
                                {10pt}    %after skip
                                {\reset@font\normalfont\bfseries\sffamily}}
                                          %font style

\newcommand{\titler}[1]{%
\vspace*{5mm}%

\noindent%
{\large\bf #1}%
\vspace*{5mm}%
}

\makeatother
\allowdisplaybreaks     % Разрешение переносить на другую страницу часть
                        % многострочной формулы
\sloppy                 %

\newcommand{\di}{\displaystyle}

%% аналог \- для внутритекстовых формул
%% пример: $y(x) \hm= R_\lambda f(x)
%% или : $Y(x)  \hm{:=} y_1(x) + y_2(x)
\newcommand{\hm}[1]{#1\nobreak\discretionary{}{\hbox{\ensuremath{#1}}}{}}
\relpenalty=10000 \binoppenalty=10000

%Обнулить все счетчики
\newcommand{\zerosetcounter}{
\setcounter{footnote}{0}\setcounter{section}{0}\setcounter{equation}{0}%
\setcounter{theorem}{0}\setcounter{lemma}{0}\setcounter{corollary}{0}\setcounter{prop}{0}
\setcounter{propos}{0}\setcounter{problem}{0}\setcounter{Theorem}{0}%
\setcounter{theoremnn}{0}\setcounter{lemmann}{0}\setcounter{corollarynn}{0}
\setcounter{propnn}{0}\setcounter{proposnn}{0}%
\setcounter{definitionn}{0}\setcounter{remarknn}{0}\setcounter{examplenn}{0}%
\setcounter{definition}{0}\setcounter{remark}{0}\setcounter{example}{0}%
\setcounter{hypothesis}{0}\setcounter{hypothesisnn}{0}\setcounter{theoremb}{0}
}

%%%%%%%%%%%%%%%%%%%%%%%%%%%%%%%%%%%%%%%%%%%%%%%%%%
%УДК
%%%%%%%%%%%%%%%%%%%%%%%%%%%%%%%%%%%%%%%%%%%%%%%%%%
\newcommand{\UDC}[1]{%
\vspace*{5mm}
\noindent\textsf{УДК %
#1}%
}

%%%%%%%%%%%%%%%%%%%%%%%%%%%%%%%%%%%%%%%%%%%%%%%%%%
%Заголовки статей
%%%%%%%%%%%%%%%%%%%%%%%%%%%%%%%%%%%%%%%%%%%%%%%%%%
\newcommand{\Rtitle}[1]{%
\begin{center}%
{ \bf\fontsize{14pt}{16pt}\sffamily%
#1}%
\end{center}%
}

\newcommand{\Etitle}[1]{%
\newpage
\begin{center}%
{ \bf\fontsize{12pt}{14pt}\sffamily%
#1}%
\end{center}%
}

%%%%%%%%%%%%%%%%%%%%%%%%%%%%%%%%%%%%%%%%%%%%%%%%%%%%%%%
%Авторы
%%%%%%%%%%%%%%%%%%%%%%%%%%%%%%%%%%%%%%%%%%%%%%%%%%%%%%%
\newcommand{\Rauthor}[1]{%
\centerline{%
\bf\fontsize{11pt}{14pt}
\sffamily%
#1}%
}

\newcommand{\Eauthor}[1]{%
\centerline{%
\bf\fontsize{11pt}{14pt}
\sffamily%
#1}%
}

%%%%%%%%%%%%%%%%%%%%%%%%%%%%%%%%%%%%%%%%%%%%%%%%%%%%%%%
%Сведения об авторах %affiliation
%%%%%%%%%%%%%%%%%%%%%%%%%%%%%%%%%%%%%%%%%%%%%%%%%%%%%%%
\newcommand{\Raffil}[1]{%
\begin{center}
\begin{minipage}{150mm}{%
\small\sffamily%
#1}%
\end{minipage}%
\end{center}
}

\newcommand{\Eaffil}[1]{%
\begin{center}%
\begin{minipage}{150mm}{%
\small\sffamily%
#1}%
\end{minipage}%
\end{center}
}

%%%%%%%%%%%%%%%%%%%%%%%%%%%%%%%%%%%%%%%%%%%%%%%%%%%%%%%
%E-mail
%%%%%%%%%%%%%%%%%%%%%%%%%%%%%%%%%%%%%%%%%%%%%%%%%%%%%%%
\newcommand{\Email}[1]{%
\vspace*{-3mm}
\centerline{%
\small\textit{%
E-mail: #1}}%
\vspace*{3mm}%
}

%%%%%%%%%%%%%%%%%%%%%%%%%%%%%%%%%%%%%%%%%%%%%%%%%%%%%
%Аннотация
%%%%%%%%%%%%%%%%%%%%%%%%%%%%%%%%%%%%%%%%%%%%%%%%%%%%%
\newcommand{\Rabstract}[1]{%
\centerline{%
\begin{minipage}{150mm}%
{%\setlength{\parindent}{4mm}
\small\sffamily%
#1}%
\end{minipage}}%
}

\newcommand{\Eabstract}[1]{%
\centerline{%
\begin{minipage}{150mm}%
{%\setlength{\parindent}{4mm}
\small\sffamily%
#1}%
\end{minipage}}%
}

%%%%%%%%%%%%%%%%%%%%%%%%%%%%%%%%%%%%%%%%%%%%%%%%%%%%%
%Ключевые слова
%%%%%%%%%%%%%%%%%%%%%%%%%%%%%%%%%%%%%%%%%%%%%%%%%%%%%
\newcommand{\Rkeywords}[1]{%
\vspace*{3mm}%
\centerline{%
\begin{minipage}{150mm}%
{%\setlength{\parindent}{4mm}
\small\sffamily%
\textit{Ключевые слова:} #1.}%
\end{minipage}}%
\vspace*{3mm}%
}

\newcommand{\Ekeywords}[1]{%
\vspace*{3mm}%
\centerline{%
\begin{minipage}{150mm}%
{%\setlength{\parindent}{4mm}
\small\sffamily%
\textit{Key words:} #1.}%
\end{minipage}}%
\vspace*{3mm}%
}

\newlength{\realparindent}%


%%%%%%%%%%%%%%%%%%%%%%%%%%%%%%%%%%%%%%%%%%%%%%%%%%%%%
%Библиографический список
%%%%%%%%%%%%%%%%%%%%%%%%%%%%%%%%%%%%%%%%%%%%%%%%%%%%%
\makeatletter
\renewcommand{\chapter}{\@startsection{chapter}{1}{0em}%
{3.5ex plus 1ex minus .2ex}{.9ex plus.2ex}%
{\zerosetcounter}}
%Стили нумерации формул, списка литературы
\addto\captionsrussian{%
\def\bibname{}%Меняем заголовок литературы
}%
\renewcommand{\@biblabel}[1]{#1.}%Оформление номера в списке литературы
\makeatother

\newenvironment{Rtwocolbib}
{%
\vspace*{3mm} %
\noindent
{\normalfont\bfseries\sffamily Библиографический список}%
\def\bibname{}
\small
%\begin{multicols}{2}%
\vspace*{-12mm}%
\begin{thebibliography}{99}
\setlength{\itemsep}{-4pt}
}{%
\end{thebibliography}
%\end{multicols}%
\normalsize}%

\newenvironment{Etwocolbib}
{%
\vspace*{1mm} %
\noindent
{\normalfont\bfseries\sffamily References}%
%\begin{otherlanguage}{english}
\small
%\begin{multicols}{2}%
%\vspace*{-45pt}%
\begin{enumerate}
\setlength{\itemsep}{-4pt}
}{%
\end{enumerate}
%\end{multicols}%
%\end{otherlanguage}
\normalsize
}%
%%%%%%%%%%%%%%%%%%%%%%%%%%%%%%%%%%%%%%%%%%%%%%%%%%%%%%%%%%%
\renewcommand{\arraystretch}{1.1}

%%%%%%%%%%%%%%%%%%%%%%%%%%%%%%%%%%%%%%%
%Благодарности
%%%%%%%%%%%%%%%%%%%%%%%%%%%%%%%%%%%%%%%
\renewcommand{\thanks}[1]{%
\vspace*{3mm}%
\noindent\textit{Благодарности. #1.}%
\vspace{2mm}%
}%

\newcommand{\ethanks}[1]{%
%\vspace*{3mm}%
\small
\noindent\textit{Acknowledgements: #1.}%
\vspace{3mm}%
\normalsize
}%

%-------------------------------------------------
\theoremstyle{plain}
%Íóìåðîâàííûå
\newtheorem{theorem}{\indent Òåîðåìà}
\newtheorem{lemma}{\indent Ëåììà}
\newtheorem{corollary}{\indent Ñëåäñòâèå}
\newtheorem{prop}{\indent Óòâåðæäåíèå}
\newtheorem{propos}{\indent Ïðåäëîæåíèå}
\newtheorem{problem}{\indent Çàäà÷à}
\newtheorem{Theorem}{\indent Theorem}
\newtheorem{hypothesis}{\indent Ïðåäïîëîæåíèå}
\newtheorem{Etheorem}{\indent Theorem}
\newtheorem{Eproposition}{\indent Proposition}
\newtheorem{Corollary}{\indent Corollary}
\newtheorem{Lemma}{\indent Lemma}
\newtheorem{theoremb}{\indent Òåîðåìà}
\renewcommand{\thetheoremb}{\Alph{theoremb}}%Íóìåðàöèÿ òåîðåì A, B, C

\theoremstyle{remark}
\newtheorem{Example}{\indent Example}


%Äâîéíàÿ íóìåðàöèÿ
\newtheorem{theoremnn}{\indent Òåîðåìà}[section]
\newtheorem{lemmann}{\indent Ëåììà}[section]
\newtheorem{corollarynn}{\indent Ñëåäñòâèå}[section]
\newtheorem{propnn}{\indent Óòâåðæäåíèå}[section]
\newtheorem{proposnn}{\indent Ïðåäëîæåíèå}[section]
\newtheorem{hypothesisnn}{\indent Ïðåäïîëîæåíèå}[section]

\theoremstyle{plain}
%Íåíóìåðîâàííûå îêðóæåíèÿ
\newtheorem*{theorem*}{\indent Òåîðåìà}%Ïðîñòî òåîðåìà (áåç íîìåðà)
\newtheorem*{lemma*}{\indent Ëåììà}
\newtheorem*{corollary*}{\indent Ñëåäñòâèå}
\newtheorem*{Corollary*}{\indent Corollary}
\newtheorem*{prop*}{\indent Óòâåðæäåíèå}
\newtheorem*{propos*}{\indent Ïðåäëîæåíèå}
\newtheorem*{hypothesis*}{\indent Ïðåäïîëîæåíèå}
%-------------------------------------------------
\theoremstyle{definition}
\newtheorem{definitionn}{\indent Îïðåäåëåíèå}[section]
\newtheorem{remarknn}{\indent Çàìå÷àíèå}[section]
\newtheorem{examplenn}{\indent Ïðèìåð}[section]

\newtheorem{definition}{\indent Îïðåäåëåíèå}
\newtheorem{remark}{\indent Çàìå÷àíèå}
\newtheorem{example}{\indent Ïðèìåð}
\newtheorem{Remark}{\indent Remark}
\newtheorem{Question}{\indent Question}
\newtheorem{Definition}{\indent Definition}

\newtheorem*{definition*}{\indent Îïðåäåëåíèå}
\newtheorem*{remark*}{\indent Çàìå÷àíèå}
\newtheorem*{example*}{\indent Ïðèìåð}

\renewenvironment{proof}{\indent\textbf{Äîêàçàòåëüñòâî. }}{\hfill$\Box$}

\newcommand{\const}{\mathrm{const}}
\newcommand{\Span}{\mathrm{Span}\,}
\renewcommand{\Re}{\,\mathrm{Re}\,}
\renewcommand{\Im}{\,\mathrm{Im}\,}
\newcommand{\sgn}{\mathrm{sgn}\,}
\newcommand{\diag}{\mathrm{diag}\,}
%\numberwithin{equation}{section}%Äâîéíàÿ íóìåðàöèÿ ôîðìóë
%Åñëè Âû ïîäêëþ÷àåòå íîâûé ïàêåò, òî îáÿçàòåëüíî ñîîáùèòå îá ýòîì â êîììåíòàðèÿõ ê òåêñòó ñòàòüè.
\begin{document}
%%%%%%%%%%%%%%%%%%%%%%%%%%%%%%%%%%%%%%%%%%%%%%%%%%%%%%%%%%%%%%%%%%%%
\fancyhead[CE]{ÌÀÒÅÌÀÒÈÊÀ}%Ðàçäåë æóðíàëà
\fancyhead[CO]{È.~Î.~Ïåòðîâ, Ñ.~Â.~Ñèäîðîâ. Íàèìåíîâàíèå ñòàòüè }%

\UDC{501.1}

\Rtitle{ÍÀÈÌÅÍÎÂÀÍÈÅ ÑÒÀÒÜÈ ÍÀ ÐÓÑÑÊÎÌ ßÇÛÊÅ}%

%\thispagestyle{izsc}%íå óáèðàòü!!!
\Rauthor{È.~Î.~Ïåòðîâ, Ñ.~Â.~Ñèäîðîâ}%

%Ôàìèëèÿ Èìÿ îò÷åñòâî, ó÷åíàÿ ñòåïåíü, äîëæíîñòü (ñ óêàçàíèåì êàôåäðû, îòäåëà), ìåñòî ðàáîòû (ïîëíîå
%îôèöèàëüíîå íàçâàíèå ó÷ðåæäåíèÿ, ïî÷òîâûé àäðåñ), e-mail;
\Raffil{Ïåòðîâ Èãîðü Îëåãîâè÷, êàíäèäàò ôèçèêî-ìàòåìàòè÷åñêèõ
íàóê, äîöåíò êàôåäðû ìàòåìàòè÷åñêîãî àíàëèçà, Ñàðàòîâñêèé
íàöèîíàëüíûé èññëåäîàòåëüñêèé ãîñóäàðñòâåííûé óíèâåðñèòåò
èìåíè~Í.~Ã.~×åðíûøåâñêîãî, 410012, Ðîññèÿ, Ñàðàòîâ, Àñòðàõàíñêàÿ, 83, PetrovIO@info.sgu.ru\\
Ñèäîðîâ Ñåðãåé Âèêòîðîâè÷, äîêòîð òåõíè÷åñêèõ íàóê, âåäóùèé
íàó÷íûé ñîòðóäíèê, Èíñòèòóò ïðîáëåì òî÷íîé ìåõàíèêè è óïðàâëåíèÿ
ÐÀÍ, 410024, Ðîññèÿ, Ñàðàòîâ, Ðàáî÷àÿ, 24, SidorovOV@mail.mipt.ru}


\Rabstract{Àííîòàöèÿ íà ðóññêîì ÿçûêå. Òðåáîâàíèÿ ê àííîòàöèè:\\
--  Îïòèìàëüíûé îáúåì 200--250 ñëîâ.\\
-- Àííîòàöèÿ íå äîëæíà ñîäåðæàòü ñëîæíûå ôîðìóëû, ññûëêè íà
áèáëèîãðàôè÷åñêèé ñïèñîê, ïî ñîäåðæàíèþ ïîâòîðÿòü íàçâàíèå ñòàòüè,
áûòü íàñûùåíà îáùèìè ñëîâàìè, íå èçëàãàþùèìè ñóòè èññëåäîâàíèÿ;\\
Àííîòàöèÿ äîëæíà îòðàæàòü ñîäåðæàíèå ñòàòüè, ñîõðàíÿÿ åå ñòðóêòóðó
-- ââåäåíèå, öåëè è çàäà÷è, ìåòîäû èññëåäîâàíèÿ, ðåçóëüòàòû,
çàêëþ÷åíèå (âûâîäû).}

\Rkeywords{êëþ÷åâûå ñëîâà ÷åðåç çàïÿòóþ}

%%%%%%%%%%%%%%%%%%%%%%%%%%%%%%%%%%%%%%%%%%%%%%%
%Òåêñò ñòàòüè
%%%%%%%%%%%%%%%%%%%%%%%%%%%%%%%%%%%%%%%%%%%%%%%

\section{ÍÀÈÌÅÍÎÂÀÍÈÅ ÐÀÇÄÅËÀ}\label{Petr_sec1}

 ðåäàêöèþ ïðåäñòàâëÿþòñÿ ôàéëû ñòàòüè â ôîðìàòå tex è pdf, à òàêæå ôàéë Ñâåäåíèÿ.doc.

Îáúåì ïóáëèêóåìîé ñòàòüè íå äîëæåí ïðåâûøàòü 12 ñòðàíèö, îôîðìëåííûõ ñîãëàñíî äàííîìó ñòèëåâîìó
ôàéëó. Ñòàòüè áîëüøåãî îáúåìà ïðèíèìàþòñÿ òîëüêî ïî ñîãëàñîâàíèþ ñ ðåäêîëëåãèåé æóðíàëà.


\begin{theorem}\label{Petr_thm1}  Íóìåðîâàòü ñëåäóåò òîëüêî ôîðìóëû, íà êîòîðûå åñòü ññûëêè. Ñëåäóåò
èñïîëüçîâàòü àâòîìàòè÷åñêóþ íóìåðàöèþ ôîðìóë (\textrm{îêðóæåíèÿ equation, gather, align}):
\begin{equation}\label{Petr_eq1}
\vert \langle f,g\rangle\vert \leq [\langle f,f\rangle]^{\frac 12}
[\langle g,g\rangle]^{\frac 12},\qquad f,g\in C^k(X),
\end{equation}
×òîáû èñêëþ÷èòü ïåðåêðåñòíûå ññûëêè ïðè ñáîðêå ìàêåòà æóðíàëà,
ïðîñèì â ìåòêàõ (label) èñïîëüçîâàòü 3--4 áóêâû ôàìèëèè ïåðâîãî
àâòîðà è íîìåð îáúåêòà ïî ïîðÿäêó, íàïðèìåð, äëÿ ôîðìóëû 1
èñïîëüçîâàòü ìåòêó Petr-eq1.
\end{theorem}

\begin{proof}
Òåêñò äîêàçàòåëüñòâà òåîðåìû (ëåììû).
\end{proof}

\section{ÐÈÑÓÍÊÈ È ÒÀÁËÈÖÛ}\label{Petr_sec2}

Ðèñóíêè ñëåäóåò ïðåäñòàâëÿòü ôàéëàìè â ôîðìàòå eps, ïðè ýòîì ôàéë ðèñóíêà äîëæåí îáåñïå÷èâàòü
ÿñíîñòü ïåðåäà÷è âñåõ äåòàëåé.

Ðàñïå÷àòûâàåìûå ïî ôàéëó ðèñóíêè äîëæíû áûòü â íàòóðàëüíóþ
âåëè÷èíó (ãîðèçîíòàëüíûé ðàçìåð æóðíàëüíîãî ïîëÿ ðàâåí 160 ìì,
øðèôò íàäïèñåé íà ðèñóíêå è ïîäïèñè ê íåìó --- 10--11 Times New
Roman (øðèôò small)).

Ïîäðèñóíî÷íàÿ ïîäïèñü äîëæíà áûòü ñàìîäîñòàòî÷íîé áåç àïåëëÿöèè ê
òåêñòó. Åñëè èëëþñòðàöèÿ ñîäåðæèò äîïîëíèòåëüíûå îáîçíà÷åíèÿ, èõ
ñëåäóåò ðàñøèôðîâàòü.

Êàæäàÿ òàáëèöà äîëæíà áûòü ïðîíóìåðîâàíà àðàáñêèìè öèôðàìè è èìåòü òåìàòè÷åñêèé çàãîëîâîê, êðàòêî
ðàñêðûâàþùèé å¸ ñîäåðæàíèå. Âñå ñòîëáöû äîëæíû èìåòü ìàêñèìàëüíî êðàòêèå è èíôîðìàòèâíûå
ïîäçàãîëîâêè. Åäèíèöû èçìåðåíèÿ óêàçûâàþòñÿ ïîñëå çàïÿòîé.

Ïîäðèñóíî÷íûå ïîäïèñè, îáîçíà÷åíèÿ íà ðèñóíêàõ, çàãîëîâêè òàáëèö è
èõ ñîäåðæèìîå îáÿçàòåëüíî ïðèâîäÿòñÿ íà äâóõ ÿçûêàõ --- ðóññêîì è
àíãëèéñêîì.

Íà êàæäûé ðèñóíîê è êàæäóþ òàáëèöó â òåêñòå ñòàòüè îáÿçàòåëüíî äîëæíà áûòü ññûëêà.

\section{ÁÈÁËÈÎÃÐÀÔÈ×ÅÑÊÈÉ ÑÏÈÑÎÊ}\label{Petr_sec3}

Áèáëèîãðàôèÿ äîëæíà îòðàæàòü êà÷åñòâî ïðîðàáîòêè àâòîðàìè
àêòóàëüíûõ ïóáëèêàöèé ïî òåìàòèêå ñòàòüè, â òîì ÷èñëå çàðóáåæíûõ
èñòî÷íèêîâ.

Ñàìîöèòèðîâàíèå íå äîëæíî ïðåâûøàòü 20\% îò îáúåìà
áèáëèîãðàôè÷åñêîãî ñïèñêà.


 áèáëèîãðàôè÷åñêîì ñïèñêå äîëæíû áûòü óêàçàíû òîëüêî
ïðîöèòèðîâàííûå â ñòàòüå ðàáîòû. Íóìåðàöèÿ èñòî÷íèêîâ äîëæíà
ñîîòâåòñòâîâàòü î÷åðåäíîñòè ññûëîê íà íèõ â òåêñòå. Ññûëêè íà
íåîïóáëèêîâàííûå ðàáîòû íå äîïóñêàþòñÿ.

Ïðèìåðû îôîðìëåíèÿ ññûëîê íà ëèòåðàòóðó: \cite{Petr1},
\cite{Petr2,Petr3,Petr4}, \cite[ñ.~17]{Petr1}, \cite[òåîðåìà
1]{Petr1}.

%%%%%%%%%%%%%%%%%%%%%%%%%%%%%%%%%%%%%%%%%%%%%%%%%%%%%%%%%%%%%%%
%Ññûëêà íà èñòî÷íèê ôèíàíñèðîâàíèÿ (åñëè åñòü)
%%%%%%%%%%%%%%%%%%%%%%%%%%%%%%%%%%%%%%%%%%%%%%%%%%%%%%%%%%%%%%%
\thanks{Ðàáîòà âûïîëíåíà ïðè ôèíàíñîâîé ïîääåðæêå ÐÔÔÈ
(ïðîåêò ¹ 00-00-00000)}

%%%%%%%%%%%%%%%%%%%%%%%%%%%%%%%%%%%%%%%%%%%%%%%%%%%%%%%%%%%%%%%%%%%%%%
%Áèáëèîãðàôè÷åñêèé ñïèñîê
%%%%%%%%%%%%%%%%%%%%%%%%%%%%%%%%%%%%%%%%%%%%%%%%%%%%%%%%%%%%%%%%%%%%%%
\begin{Rtwocolbib}%Íå óáèðàòü!!!
%Äëÿ êíèã: Ôàìèëèè È.Î. âñåõ àâòîðîâ. Çàãëàâèå èçäàíèÿ.
%Ãîðîä~: Èçäàòåëüñòâî, ãîä èçäàíèÿ. Êîëè÷åñòâî ñòðàíèö.
%Íàïðèìåð:
\bibitem{Petr1}
\textit{Èëüèí~Â.~À.} Èçáðàííûå òðóäû~: â 2~ò. Ò.~2. Ì.~: ÌÀÊÑ
Ïðåññ, 2008. 692~ñ.

%Äëÿ ñòàòåé èç æóðíàëîâ: Ôàìèëèè È. Î. âñåõ àâòîðîâ. Íàçâàíèå
%ñòàòüè // Çàãëàâèå èçäàíèÿ. Ãîä èçäàíèÿ. Òîì, Íîìåð. Ñòðàíèöû. DOI (åñëè åñòü)
%Íàïðèìåð:
\bibitem{Petr2}
\textit{Áóðëóöêàÿ~Ì.~Ø., Õðîìîâ~À.~Ï.} Ðåçîëüâåíòíûé ïîäõîä â
ìåòîäå Ôóðüå~// Äîêë. ÀÍ. 2014. Ò.~458, ¹~2. Ñ.~138--140. DOI:
10.7868/S0869565214260041.

\bibitem{Petr3}
\textit{Õðîìîâ~À.~Ï.} Î êëàññè÷åñêîì ðåøåíèè îäíîé ñìåøàííîé
çàäà÷è äëÿ âîëíîâîãî óðàâíåíèÿ~// Èçâ. Ñàðàò. óí-òà. Íîâ. ñåð.
Ñåð. Ìàòåìàòèêà. Ìåõàíèêà. Èíôîðìàòèêà. 2015. Ò.~15, âûï.~1.
Ñ.~56--66. DOI: 10.18500/1816-9791-2015-15-1-56-66.

%Äëÿ ìàòåðèàëîâ êîíôåðåíöèé, ñáîðíèêîâ òðóäîâ è ò.ä.:
%Ôàìèëèè  È. Î. âñåõ àâòîðîâ. Íàçâàíèå ñòàòüè
%// Çàãëàâèå èçäàíèÿ. Ìåñòî, ãîä èçäàíèÿ. Òîì, Íîìåð (âûïóñê). Ñòðàíèöû.
\bibitem{Petr4}
\textit{Áóëàíîâ~À.~Ï.} Îá èíâàðèàíòàõ íà ñîâîêóïíîñòè ïîêàçàòåëåé
âçàèìíî îáðàòíûõ ôóíêöèé Ëàìáåðòà, ïðåäñòàâëåííûõ öåïíûìè
ýêñïîíåíòàìè~// Ñîâðåìåííûå ìåòîäû òåîðèè ôóíêöèé è ñìåæíûå
ïðîáëåìû~: ìàòåðèàëû Âîðîíåæ. çèì. ìàòåì. øê. Âîðîíåæ, 2013.
Ñ.~295--303.
\end{Rtwocolbib}%Íå óáèðàòü!!!

\language=0%English

\Etitle{Íàèìåíîâàíèå Ñòàòüè íà Àíãëèéñêîì ßçûêå\\
(âñå ñëîâà, êðîìå ïðåäëîãîâ, ñ áîëüøîé áóêâû)}

\Eauthor{I.~O.~Petrov, S.~V.~Sidorov}%

%Íà àíãë. ÿçûêå: Èìÿ Î. Ôàìèëèÿ, îôèöèàëüíîå íàçâàíèå îðãàíèçàöèè, ïî÷òîâûé àäðåñ
%îðãàíèçàöèè, e-mail àâòîðà
\Eaffil{Igor O. Petrov, orcid.org/0000-0002-7926-1347, Saratov
State University, 83, Astrakhanskaya Str.,  Saratov, Russia, 410012, PetrovIO@info.sgu.ru\\
Sergei V. Sidorov, orcid.org/0000-0002-0426-2046, Institute of
Precision Mechanics and Control, Russian Academy of Sciences, 24,
Rabochaya Str., Saratov, Russia, 410028, SidorovOV@mail.mipt.ru}%

\Eabstract{Àííîòàöèÿ íà àíãëèéñêîì ÿçûêå (200--250 ñëîâ).\\
Êà÷åñòâåííûé ïåðåâîä àííîòàöèè íà àíãëèéñêèé ÿçûê ïîçâîëÿåò:\\
-- çàðóáåæíîìó ó÷åíîìó îçíàêîìèòüñÿ ñ ñîäåðæàíèåì ñòàòüè è îïðåäåëèòü èíòåðåñ ê íåé, íåçàâèñèìî îò
ÿçûêà ñòàòüè è íàëè÷èÿ âîçìîæíîñòè ïðî÷èòàòü åå ïîëíûé òåêñò;\\
-- ïðåîäîëåâàòü ÿçûêîâûé áàðüåð ó÷åíîìó, íå çíàþùåìó ðóññêèé ÿçûê;\\
-- ïîâûñèòü âåðîÿòíîñòü öèòèðîâàíèÿ ñòàòüè çàðóáåæíûìè êîëëåãàìè.}%

\Ekeywords{êëþ÷åâûå ñëîâà íà àíãëèéñêîì ÿçûêå}

%%%%%%%%%%%%%%%%%%%%%%%%%%%%%%%%%%%%%%%%%%%%%%%%%%%%%%%%%%%%%%%
%Ññûëêà íà èñòî÷íèê ôèíàíñèðîâàíèÿ (åñëè åñòü)
%%%%%%%%%%%%%%%%%%%%%%%%%%%%%%%%%%%%%%%%%%%%%%%%%%%%%%%%%%%%%%%
\ethanks{This work was supported by the Russian Foundation for
Basic Research (projects no.~00-00-00000)}

%%%%%%%%%%%%%%%%%%%%%%%%%%%%%%%%%%%%%%%%%%%%%%%%%%%%%%%%%%%%%%%%%%%%%%
%References - Áèáëèîãðàôè÷åñêèé ñïèñîê íà àíãë. ÿçûêå
%%%%%%%%%%%%%%%%%%%%%%%%%%%%%%%%%%%%%%%%%%%%%%%%%%%%%%%%%%%%%%%%%%%%%%

Îñíîâíûå ïðàâèëà îôîðìëåíèÿ  ðóññêîÿçû÷íûõ èñòî÷íèêîâ â References:

--  åñëè ñóùåñòâóåò ïåðåâîäíàÿ âåðñèÿ ñòàòüè (êíèãè), òî ñëåäóåò
ïðåä-ñòàâèòü èìåííî åå; ïåðåâîäíàÿ âåðñèÿ ìîæåò áûòü òàêæå îïèñàíà
êàê äîïîëíèòåëüíûå ñâåäåíèÿ (â ñêîáêàõ);

--  åñëè ïåðåâîäíîé âåðñèè íå ñóùåñòâóåò, òî ìîæíî èñïîëüçîâàòü
òðàíñëèòåðàöèþ (http://translit.ru, âàðèàíò LC), â êâàäðàòíûõ
ñêîáêàõ îáÿçàòåëüíî ïðåäñòàâëÿåòñÿ ïåðåâîä íà àíãë. ÿçûê íàçâàíèÿ
ñòàòüè (êíèãè), ïîñëå îïèñàíèÿ äîáàâëÿåòñÿ ÿçûê ïóáëèêàöèè (in
Russian);

--  åñëè îïèñûâàåìàÿ ïóáëèêàöèÿ èìååò DOI, åãî îáÿçàòåëüíî íàäî
óêàçûâàòü.


\smallskip

Ðåêîìåíäóåì èñïîëüçîâàòü ñàéòû http://www.mathnet.ru è
http://elibrary.ru.

\begin{Etwocolbib}%Íå óáèðàòü!!!
\bibitem{ePetr1}Il'in~V.~A. \textit{Izbrannye trudy} [Chosen works]. Vol.~2.
Moscow, MAKS Press, 2008. 692~p. (in Russian).

\bibitem{ePetr2}
Burlutskaya~M.~S.,
Khromov~A.~P. Rezolventny approach in the Fourier method.
\textit{Doklady Math.}, 2014, vol.~90, no.~2, pp.~545--548. DOI:
10.1134/ S1064562414060076.

\bibitem{ePetr3}
Khromov~A.~P. About the classical solution of the mixed problem
for the wave equation. \textit{Izv. Saratov Univ. (N. S.), Ser.
Math. Mech. Inform.}, 2015, vol.~15, no.~1, pp.~56--66 (in
Russian). DOI: 10.18500/1816-9791-2015-15-1-56-66.

\bibitem{ePetr4}
Bulanov~A.~P. Ob invariantakh na sovokupnosti pokazatelei vzaimno
obratnykh funktsii Lamberta, predstavlennykh tsepnymi eksponentami
[On invariants on the set of indicators of mutually inverse
functions Lambert submitted chain exhibitors]. \textit{Sovremennye
metody teorii funktsii i smezhnye problemy~: materialy Voronezh.
zim. matem. shk.} [Modern Methods of Function Theory and Related
Problems~: Proc. Voronezh Winters School], Voronezh, 2013,
pp.~295--303 (in Russian).
\end{Etwocolbib}%Íå óáèðàòü!!!
\language=1%Ðóññêèé

\newpage

\begin{center}
\textbf{ÏÐÈÌÅÐÛ}
\end{center}

\textbf{Îïèñàíèå êíèãè (ìîíîãðàôèè,
ñáîðíèêè)}

\textit{Àãàåâ~Ã.~Í., Âèëåíêèí~Í.~ß., Äæàôàðëè~Ã.~Ì.,
Ðóáèíøòåéí~À.~È.} Ìóëüòèïëèêàòèâíûå ñèñòåìû ôóíêöèé è
ãàðìîíè÷åñêèé àíàëèç íà íóëüìåðíûõ ãðóïïàõ. Áàêó~: Ýëì, 1981.
180~c.

Agaev~G.~N., Vilenkin~N.~Ya., Dzafarli~G.~M., Rubinstein~A.~I.
\textit{Mul'tiplikativnye sistemy funkcij i garmonicheskij analiz
na nul'mernyh gruppah} [Multiplicative Systems of Functions and
Harmonic Analysis on Zero-Dimensional Groups]. Baku, Elm, 1981,
180~p.  (in Russian).

\textbf{Îïèñàíèå ïåðåâîäíîé
êíèãè}

\textit{Êàðãàïîëîâ~Ì.~È., Ìåðçëÿêîâ~Þ.~È.} Îñíîâû òåîðèè ãðóïï.
Ì.~: Íàóêà, 1982. 288~ñ.

Kargapolov~M.~I., Merzljakov~Ju.~I. \textit{Fundamentajs of
the Theory of Groups}. New York, Springer-Verlag, 1979, 203~p.
(Rus. ed.~: Kargapolov~M.~I., Merzljakov~Ju.~I. Osnovy teorii
grupp. Moscow, Nauka, 1982,  288~p.)


\textbf{Îïèñàíèå ñòàòüè èç æóðíàëà}

\textbf{\textit{Ïåðåâîäíûå}}

\textit{Áóðëóöêàÿ~Ì.~Ø., Êóðäþìîâ~Â.~Ï., Ëóêîíèíà~À.~Ñ.,
Õðîìîâ~À.~Ï.} Ôóíêöèîíàëüíî-äèôôåðåíöèàëüíûé îïåðàòîð ñ
èíâîëþöèåé~// ÄÀÍ. 2007. Ò.~414. ¹~4. Ñ.~443--446.

Burlutskaya~M.~Sh., Kurdyumov~V.~P., Lukonina~A.~S.,
Khromov~A.~P. A func\-tio\-nal-differential operator with
involution. \textit{Doklady Math.}, 2007,  vol.~75, no.~3,
pp.~399--402.


{\textbf{\textit{Íåïåðåâîäíûå}}}

\textit{Õðîìîâ~À.~Ï.} Ñìåøàííàÿ çàäà÷à äëÿ äèôôåðåíöèàëüíîãî
óðàâíåíèÿ ñ èíâîëþöèåé è ïîòåíöèàëîì ñïåöèàëüíîãî âèäà~// Èçâ.
Ñàðàò. óí-òà. Íîâ. ñåð. Ñåð. Ìàòåìàòèêà. Ìåõàíèêà. Èíôîðìàòèêà.
2010. Ò.~10, âûï.~4. Ñ.~17--22.

Khromov~A.~P. The mixed problem for the differential equation
with involution and potential of the special kind. \textit{Izv.
Saratov Univ. (N.\,S.), Ser. Math. Mech. Inform.}, 2010,  vol.~10,
iss.~4, pp.~17--22 (in Russian).

{\textbf{Îïèñàíèå ñòàòüè ñ doi}}

\textit{Õðîìîâ~À.~Ï.} Îá îáðàùåíèè èíòåãðàëüíûõ îïåðàòîðîâ ñ
ÿäðàìè, ðàçðûâíûìè íà äèàãîíàëÿõ~// Ìàò. çàìåòêè. 1998. Ò.~64,
¹~6. Ñ.~932--949. DOI: 10.4213/mzm1472.

Khromov~A.~P. Inversion of integral operators with kernels
discontinuous on the diagonal. \textit{Math. Notes}. 1998,
vol.~64, no.~6, pp.~804--813. DOI: 10.4213/mzm1472.

\textbf{Îïèñàíèå ñòàòüè â ýëåêòðîííîì
æóðíàëå}

\textit{Îðëÿíñêàÿ~È.~Â.} Ñîâðåìåííûå ïîäõîäû ê ïîñòðîåíèþ
ìåòîäîâ ãëîáàëüíîé îïòèìèçàöèè // Ýëåêòðîííûé æóðíàë <<Èññëåäîâàíî
â Ðîññèè>>. Ñ.~2097--2108. URL: http://zhurnal.ape.relarn.ru/
articles/2002/189.pdf  (äàòà îáðàùåíèÿ: 02.12.2011).

Orlyanskaya I. V. Modern approaches to global optimization
methods building. \textit{Online journal <<Issledovano v
Rossii>>}, pp. 2097--2108. Available at:
http://zhurnal.ape.relarn.ru/articles/ 2002/189.pdf (Accessed 02,
December, 2012).

\textbf{Îïèñàíèå ñòàòüè èç ñáîðíèêà òðóäîâ}

\textit{×åëíîêîâ~Þ.~Í.} Îïòèìàëüíàÿ ïåðåîðèåíòàöèÿ îðáèòû
êîñìè÷åñêîãî àïïàðàòà ïîñðåäñòâîì ðåàêòèâíîé òÿãè, îðòîãîíàëüíîé
ïëîñêîñòè îðáèòû // Ìàòåìàòèêà. Ìåõàíèêà~: ñá. íàó÷. òð. Ñàðàòîâ~:
Èçä-âî Ñàðàò. óí-òà, 2006. Âûï.~8. Ñ.~231--234.

Chelnokov~Yu.~N. Optimal'naja pereorientacija orbity
kosmicheskogo apparata posredstvom reaktivnoj tjagi, ortogonal'noj
ploskosti orbity [Optimal reorientation of spacecraft's orbit
through thrust orthogonal to the plane of orbit].
\textit{Matematika. Mehanika} [Mathematics. Mechanics]. Saratov,
Saratov Univ. Press, 2006, iss.~8, pp.~231--234 (in Russian).

\textbf{Îïèñàíèå ìàòåðèàëîâ êîíôåðåíöèé}

\textit{Àíäðååâ~À.~À.} Î êîððåêòíîñòè êðàåâûõ çàäà÷ äëÿ
íåêîòîðûõ óðàâíåíèé â ÷àñòíûõ ïðîèçâîäíûõ ñ êàðëåìàíîâñêèì
ñäâèãîì~// Äèôôåðåíöèàëüíûå óðàâíåíèÿ è èõ ïðèëîæåíèÿ~: òð. 2-ãî
ìåæäóíàð. ñåìèíàðà. Ñàìàðà, 1998. Ñ.~5--18.

Andreev~A.~A. About the correctness of boundary problems for
some equations with calimanesti shift. \textit{Differentsial'nye
uravneniia i ikh prilozheniia~: trudy 2-go mezhdunarodnogo
seminara} [Differential equations and their applications~:
proceedings of the 2nd international workshop]. Ñàìàðà, 1998,
pp.~5--18 (in Russian).


\textbf{Îïèñàíèå Èíòåðíåò-ðåñóðñà}

Illumina, Inc. URL: http://www. illumina.com/  (äàòà
îáðàùåíèÿ: 18.05.2012).

\textit{Illumina, Inc.} Available at: http://
www.illumina.com/  (Accessed  18, May, 2012).

\textbf{Îïèñàíèå äèññåðòàöèè èëè àâòîðåôåðàòà äèññåðòàöèè}

\textit{Òåðåõèí~Ï.~À.} Àôôèííûå ñèñòåìû ôóíêöèé è ôðåéìû â
áàíàõîâîì ïðîñòðàíñòâå : äèñ. ... ä-ðà ôèç.-ìàò. íàóê. Ñàðàòîâ,
2010. 230~ñ.

Terekhin~P.~A. \textit{Affinnye sistemy funkcij i frejmy v
banahovom prostranstve}. Diss. dokt. fiz.-mat. nauk [Affine
systems of functions and frames in Banach space : Dr. phys. and
math. sci. diss.]. Saratov, 2010. 230 p. (in Russian).

\textbf{Îïèñàíèå ÃÎÑÒà}

ÃÎÑÒ 27.002-89. Íàäåæíîñòü â òåõíèêå. Îñíîâíûå ïîíÿòèÿ.
Òåðìèíû è îïðåäåëåíèÿ. Ì., 1990. 24~ñ.

State Standard 27.002-89. Industrial product dependability.
General concepts. Terms and definitions. Moscow, Standartinform,
1990. 24~p. (in Russian).
\end{document}
