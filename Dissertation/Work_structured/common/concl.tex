%% Согласно ГОСТ Р 7.0.11-2011:
%% 5.3.3 В заключении диссертации излагают итоги выполненного исследования, рекомендации, перспективы дальнейшей разработки темы.
%% 9.2.3 В заключении автореферата диссертации излагают итоги данного исследования, рекомендации и перспективы дальнейшей разработки темы.
\begin{enumerate}
  \item Построена и исследована математическая модель потока неоднородных требований в виде неординарного пуассоновского потока с ограниченным количеством требований в группе. Найдены основные вероятностные характеристики такого потока. 
  \item Для обоснования корректности построенной модели потока разработана компьютерная программа. Такая программа позволяет 1)~получить нелокальное описание реальных потоков требований, 2)~проверить, может ли построенная математическая модель быть использована при описании реальных потоков, 3)~получить оценку неизвестных параметров распределений, возникающих в указанной модели.
  \item Построена математическая модель для двух систем обслуживания требований и управления потоками: 1)~циклическое управление потоками неоднородных заявок, 2)~управление разнородными потоками с помощью адаптивного алгоритма с пороговым приоритетом и возможностью продления обслуживания. В обоих случаях модель представляет собой многомерную однородную управляемую цепь Маркова. 
  \item Произведена классификация состояний указанных цепей Маркова, получены рекуррентные соотношения для одномерных распределений таких марковских процессов и их производящих функций. Итеративно-мажорантным методом получены условия существования в системах стационарного режима. Найденные условия являются легко проверяемыми ограничениями на значения параметров и характеристик системы. 
  \item Для указанных систем построены компьютерные имитационные модели. Разработан алгоритм определения момента достижения системами квазистационарного режима. Предложен способ получения численных оценок основных показателей качества функционирования системы.
  \item Для рассмотренных систем управления предложены алгоритмы поиска квазиоптимальных значений управляющих параметров, при которых достигается минимальное значение оценки среднего времени ожидания начала обслуживания произвольной заявкой в квазистационарном режиме.
\end{enumerate}
