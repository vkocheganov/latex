{\aim} Целями данной работы являются: 1)~построение и исследование математической модели тандема управляющих систем обслуживания по циклическому алгоритму с продлением; 2)~построение, реализация и анализ имитационной модели систем, осуществляющих циклическое управление с продлением тандемом перекрестков.

Для~достижения поставленных целей решаются следующие задачи:

1. Построение строгой вероятностной модели тандема управляющих систем с помощью явного построения вероятностного пространства и поточечного задания необходимых для исследования случайных величин и элементов.

2. Анализ построенной вероятностной модели, получение условий существования стационарного режима в различных подсистемах тандема.

3. Разработка имитационной модели тандема, определение момента достижения системы квазистационарного режима, анализ зависимости условий стационарности от управляющих параметров.



{\novelty} Результаты диссертации являются новыми и заключаются в следующем:

1. Впервые построена вероятностная модель тандема управляющих систем с немгновенным перемещением требований между ними, управление в которых осуществляется по циклическому алгоритму и алгоритму с продлением. В этой модели требования сначала поступают в первую систему на обслуживание по циклическому алгоритму, а затем немгновенно поступают во вторую систему на обслуживание по циклическому алгоритму с продлением. Немгновенность перемещения моделируется при помощи биномиальной случайной величины с параметром $p$, имеющим смысл вероятности перехода требования из одной системы в другую за определенный промежуток времени.
%Построена строгая вероятностная модель тандема управляющих систем с немгновенным перемещением требований между ними, управление в которых осуществляется по циклическому алгоритму и алгоритму с продлением. 

2. Впервые применен аппарат абстрактных управляющих систем Ляпунова--Яблонского для изучения указанной выше системы. Построенная по принципам кибернетического подхода вероятностная модель позволила провести разносторонний анализ системы. В частности, была проведена классификация состояний марковской цепи, описывающей динамику системы, найдены рекуррентные соотношения для соответствующих производящих функций и были изучены эргодические свойства системы. Также, благодаря этому подходу, была построена и реализована имитационная модель для численного анализа системы.
%Изучены эргодические свойства построенной модели, найдены условия существования стационарного режима для очередей первичных требований, а также для промежуточной очереди.

3. Впервые применен итеративно-мажорантный метод для нахождения достаточных условий существования стационарного распределения в указанной выше модели. Благодаря итеративно-мажорантному методу были найдены условия существования стационарного режима для очередей первичных требований, а также для промежуточной очереди.

%3. Разработана и реализована имитационная модель для тандема

%4. Проведено исследование вероятностной и имитационной моделей, и определена расширенная область стационарности системы при алгоритме с продлением.




{\influence} Научная значимость работы заключается в построении строгой вероятностной модели 
для качественно нового вида управляющей системы и в последовательном исследовании ее эргодических свойств. Успешно примененный в работе метод нелокального описания процессов существенно расширяет множество поддающихся исследованию реальных систем массового обслуживания. Строгая математическая модель позволяет оперировать существующим, хорошо разработанным вероятностным аппаратом для нахождения условий стационарности и нахождения оптимального управления системой. 
 Разработанные модели дают базу для изучения более комплексных тандемных систем, систем с более сложными входными потоками и алгоритмами управления.

Практическая значимость исследования состоит в том, что изученная управляющая система является адекватным описанием реальной системы тандема перекрестков, а также других сетей, состоящих из двух узлов с перемещающимися между ними требованиями и циклическими алгоритмами обслуживания с продлением на узлах.




{\methods} 
В диссертации применяется аппарат теории вероятностей, теории массового обслуживания, исследования операций, теории управляемых марковских процессов. Также применяются методы теории линейных отображений, математической статистики и теории функций комплексного переменного.
 При реализации имитационной модели на компьютере использовались языки программирования C++, Python.

Методология диссертации основывается на   представлении стохастических систем массового обслуживания в виде абстрактных управляющих систем Ляпунова--Яблонского. Использование данной методологии  позволяет разделить исследуемые системы на составные части (блоки), описать эти части математически,  задать правила их функционирования и взаимодействия между собой.
Для описания входных потоков было примено нелокальное описание.
%, что сделало возможным более глубокое математическое изучение рассматриваемых объектов.

\pagebreak
{\defpositions}

1. Методика построения вероятностного пространства для тандема систем обслуживания по циклическому алгоритму с продлением и задержкой требований между ними.

2. Методика определения условий существования стационарного режима в системах управления неординарными пуассоновскими потоками требований с использованием циклического алгоритма и алгоритма с продлением.

3. Методика определения фазы квазистационарного режима  управляющей системы обслуживания тандемного типа.





{\probation} Достоверность полученных результатов обеспечивается строгим применением используемого математического аппарата, проведением и сравнением статистических и численных исследований. Результаты работы находятся в соответствии с результатами, полученными ранее другими авторами при исследовании управляющих систем обслуживания.

Основные результаты диссертации докладывались и обсуждались на следующих  конференциях.
\begin{enumerate}
    \item Международная научная конференция <<Теория вероятностей, случайные процессы, математическая статистика и приложения>> (Минск, Республика Беларусь, 2015 г.).
    \item IX Международная конференция <<Дискретные модели в теории управляющих систем>> (Москва и Подмосковье, 2015 г.).
\item 8-я международная научная конференция <<Распределенные компьютерные и коммуникационные сети: управление, вычисление, связь>> DCCN-2015 (Москва, 2015 г.).
\item Международная научная конференция <<Distributed Computer and Communication Networks>> DCCN 2016 (Москва, 2016 г.).
\item XVIII Международная конференция <<Проблемы теоретической кибернетики>> (Пенза, 2017 г.).
\item XVI Международная конференция имени А.Ф. Терпугова <<Информационные технологии и математическое моделирование>> ИТММ-2017 (Казань, 2017 г.).
\item  20-я международная научная конференция <<Распределенные компьютерные и телекоммуникационные сети: управление, вычисление, связь>> DCCN-2017 (Москва, 2017 г.).
\item IX Московская международная конференция по исследованию операций (Москва, 2018~г.).
\item Четвертая международная конференция по стохастическим методам МКСМ-4 (пос.~Дивноморское, г.~Новороссийск).
\item XVIII Международная конференция имени А.Ф.~Терпугова <<Информационные технологии и математическое моделирование>> ИТММ-2019 (Саратов, 2019 г.).
\end{enumerate}


{\contribution} В совместных публикациях научному руководителю принадлежит постановка задачи и общее редактирование работ. Все исследования выполнены автором диссертации лично, все полученные результаты принадлежат автору. 

{\passport} Диссертационная работа выполнена в соответствии с паспортом специальности 01.01.09 «Дискретная математика и математическая кибернетика» и включает оригинальные результаты в области дискретной математики и математической кибернетики. 

Исследование, приведенное в работе, соответствует следующим
разделам паспорта специальности: 
\begin{itemize}
    \item пункт~4 (Математическая теория исследования операций и теория игр)~--- построена и изучена математическая модель для новой системы массового обслуживания методами теории исследования операций;
    \item пункт~2 (Теория управляющих систем)~--- математическая модель исследуемой системы представлена в виде абстрактной управляющей системы Ляпунова--Яблонского и построена на базе принципов кибернетического подхода.
\end{itemize} 

\ifnumequal{\value{bibliosel}}{0}{% Встроенная реализация с загрузкой файла через движок bibtex8
    \publications\ Основные результаты по теме диссертации изложены в XX печатных изданиях, 
    X из которых изданы в журналах, рекомендованных ВАК, 
    X "--- в тезисах докладов.%
}{% Реализация пакетом biblatex через движок biber
%Сделана отдельная секция, чтобы не отображались в списке цитированных материалов
    %\begin{refsection}%
        %\printbibliography[heading=countauthornotvak, env=countauthornotvak, keyword=biblioauthornotvak, section=1]%
        %\printbibliography[heading=countauthorvak, env=countauthorvak, keyword=biblioauthorvak, section=1]%
        %\printbibliography[heading=countauthorconf, env=countauthorconf, keyword=biblioauthorconf, section=1]%
        %\printbibliography[heading=countauthor, env=countauthor, keyword=biblioauthor, section=1]%
        %\publications\ Основные результаты по теме диссертации изложены в \arabic{citeauthor} печатных изданиях\nocite{bib1,bib2}, 
        %\arabic{citeauthorvak} из которых изданы в журналах, рекомендованных ВАК\cite{vestnikUNN,vestnikVGAVT1,vestnikVGAVT2,vestnikTGU}, 
        %\nocite{DCCN2010,Minsk2011,Novgorod2011,Novosibirsk2011,DCCN2013,Kazan,Minsk2015,RachinskayaStatistics,Dm2015,DCCN2015,Dm2016,Penza2017,DCCN2017,Tomsk2017,Soloviev2017}\arabic{citeauthorconf} "--- в тезисах докладов  \cite{DCCN2010,Minsk2011,Novgorod2011,Novosibirsk2011,DCCN2013,Kazan,Minsk2015,RachinskayaStatistics,Dm2015,DCCN2015,Dm2016,Penza2017,DCCN2017,Tomsk2017,Soloviev2017}.
        	\publications\ Основные результаты по теме диссертации изложены в 17 работах, 
	5 из них "--- в журналах, рекомендованных ВАК (\cite{vestnikTvGU,UBS,vestnikVGAVT2, vestnikSGU, MKSM2019}),
	2 "--- в библиографической базе Scopus, 2 "--- в библиографической базе Web of Science, 13 "--- в библиографической базе РИНЦ (\cite{ Dm2015, Penza2017, Tomsk2017, ITMM2019, DCCN2016:3, DCCN2016:1, DCCN2016:2,  DCCN2017, vestnikTvGU, vestnikVGAVT2, UBS, vestnikSGU, MKSM2019}),
	10 "--- в тезисах докладов (\cite{Minsk2015, Dm2015, DCCN2015, DCCN2016:3, Penza2017, Tomsk2017, DCCN2017, ORM2018, ITMM2019, MKSM2019}). Получено $1$ свидетельство о государственной регистрации программы для ЭВМ~\cite{RID}.
}

% \underline{\textbf{Объем и структура работы.}} Диссертация состоит из~введения, четырех глав, заключения и~приложения. Полный объем диссертации \textbf{ХХХ}~страниц текста с~\textbf{ХХ}~рисунками и~5~таблицами. Список литературы содержит \textbf{ХХX}~наименование.
