\chapter{Анализ первичных и промежуточной очередей системы}	

В главе~3 более подробно изучаются случайные последовательности,  содержащие состояния только части очередей из последовательности $\Mark$: очереди $O_1$,  $O_3$ и $O_4$. Исключение из рассмотрения нескольких компонент пятмерной марковской цепи позволяет найти достаточное,  а в некоторых случаях  необходимое условия существования стационарного распределения. Для нахождения условий стационарности используется хорошо зарекомендовавший себя в схожих задачах итеративно-мажорантный метод,  в котором последовательность математических ожиданий компонент цепи ограничивается числовой последовательностью более простого вида.
Ограниченность математических ожиданий,  в следствие особенностей цепи,  влечет существование стационарного распределения.

\section{Условия ограниченности очереди $O_4$ потока $\Pi_4$}
\begin{theorem}
Для того,  чтобы последовательность 
$$
\{\varkappa_{4, i}(\omega); i =0,  1,  \ldots\}, 
$$ была ограничена,  достаточно выполнения неравенства
\begin{equation*}
   % \min_{\substack{k=\overline{1, d}\\ j=1, 3}} {\{p_{k, r}\}} > 0.
    \min_{k=\overline{0, d},  r=\overline{1, n_k}} {\{p_{k, r}\}} > 0.
\end{equation*}
\end{theorem}
\begin{proof}
Пусть $(\gamma,  x_3) \in \Gamma \times {\mathbb Z}_+$ и  $\Gamma^{(\tilde{k}, \tilde{r})}=h(\gamma, x_3)$. Учитывая выражения \eqref{FourthFunc},  получим
\begin{multline*}
    E[\varkappa_{4, i+1} | \varkappa_{1, i}=w_1, \varkappa_{3, i}=w_3,  \varkappa_{4, i}=w_4,  \Gamma_i=\gamma] = \\
    =
    E[w_4 - \eta_{2, i} +  \min{\{\xi_{1, i},  w_1 + \eta_{1, i} \}} | \varkappa_{1, i}=w_1, \varkappa_{3, i}=w_3,  \varkappa_{4, i}=w_4,  \Gamma_i=\gamma] \leqslant \\
    \leqslant
     E[w_4 - \eta_{2, i} +  \xi_{1, i} | \varkappa_{1, i}=w_1, \varkappa_{3, i}=w_3,  \varkappa_{4, i}=w_4,  \Gamma_i=\gamma] =\\ 
     = 
     E[w_4 - \eta_{2, i} +  \ell(\tilde{k}, \tilde{r}, 1) | \varkappa_{1, i}=w_1, \varkappa_{3, i}=w_3,  \varkappa_{4, i}=w_4,  \Gamma_i=\gamma] = \\ 
    =
   w_4 + \ell(\tilde{k}, \tilde{r}, 1)  -  E[\eta_{2, i} | \varkappa_{1, i}=w_1, \varkappa_{3, i}=w_3,  \varkappa_{4, i}=w_4,  \Gamma_i=\gamma].
\end{multline*}
Из соотношений \eqref{ProbablititiesToProve} следует
\begin{multline*}
  E[\eta_{2, i} | \varkappa_{1, i}=w_1, \varkappa_{3, i}=w_3,  \varkappa_{4, i}=w_4,  \Gamma_i=\gamma]
= \\
=
\sum_{a=0}^{w_4}  a \psi(a;w_4,  p_{\tilde{k}, \tilde{r}}) = 
\sum_{a=0}^{w_4} a C_{w_4}^{a} p_{\tilde{k}, \tilde{r}}^a (1-p_{\tilde{k}, \tilde{r}})^{w_4-a} = w_4 p_{\tilde{k}, \tilde{r}}.
\end{multline*}
Таким образом, верна оценка 
\begin{equation*}
     E[\varkappa_{4, i+1} | \varkappa_{1, i}=w_1, \varkappa_{3, i}=w_3,  \varkappa_{4, i}=w_4,  \Gamma_i=\gamma] \leqslant
     w_4 (1-p_{\tilde{k}, \tilde{r}}) + \ell(\tilde{k}, \tilde{r}, 1).
\end{equation*}
Далее по формуле полного математического ожидания:
\begin{multline*}
    E[\varkappa_{4, i+1}] =  \\
    =\sum_{w_1=0}^{\infty} \sum_{w_3=0}^{\infty}  \sum_{w_4=0}^{\infty} \sum_{\gamma \in \Gamma}  E[\varkappa_{4, i+1} | \varkappa_{1, i}=w_1, \varkappa_{3, i}=w_3,  \varkappa_{4, i}=w_4,  \Gamma_i=\gamma]  \times \\ 
    \times
    \Pr(\Gamma_{i}=\gamma,  \varkappa_{1, i}=w_1,  \varkappa_{3, i}=w_3,  \varkappa_{4, i}=w_4)
    \leqslant \\ 
    \leqslant
    \sum_{w_1=0}^{\infty} \sum_{w_3=0}^{\infty}  \sum_{w_4=0}^{\infty} \sum_{\gamma \in \Gamma} 
    ( w_4 (1-p_{\tilde{k}, \tilde{r}}) + \ell(\tilde{k}, \tilde{r}, 1) )     \times \\
    \times
    \Pr(\Gamma_{i}=\gamma,  \varkappa_{1, i}=w_1,  \varkappa_{3, i}=w_3,  \varkappa_{4, i}=w_4).
\end{multline*}
Избавимся в последнем выражении от суммирования по $w_1$,  а также вынесем суммирование по $w_4$:
    \begin{multline*}
    E[\varkappa_{4, i+1}]\leqslant\\
    \leqslant
    \sum_{w_3=0}^{\infty}  \sum_{w_4=0}^{\infty} \sum_{\gamma \in \Gamma} 
    ( w_4 (1-p_{\tilde{k}, \tilde{r}}) + \ell(\tilde{k}, \tilde{r}, 1) ) \times 
    \Pr(\Gamma_{i}=\gamma,  \varkappa_{3, i}=w_3,  \varkappa_{4, i}=w_4)  \leqslant\\
    \leqslant
    \sum_{w_4=0}^{\infty}  ( w_4 (1-\min_{\substack{\tilde{k}=\overline{0, d}, \\ \tilde{r}=\overline{1, n_{\tilde{k}}}}}{\{p_{\tilde{k}, \tilde{r}}\}}) + \max_{\substack{\tilde{k}=\overline{0, d}, \\ \tilde{r}=\overline{1, n_{\tilde{k}}}} } {\{\ell(\tilde{k}, \tilde{r}, 1)\}} )    \times \\
   \times \sum_{w_3=0}^{\infty}   \sum_{\gamma \in \Gamma} 
 \Pr(\Gamma_{i}=\gamma,  \varkappa_{3, i}=w_3,  \varkappa_{4, i}=w_4),
\end{multline*}
и после упрощений получим
\begin{multline*}
        E[\varkappa_{4, i+1}]\leqslant\\
    \leqslant
         (1-\min_{\substack{\tilde{k}=\overline{0, d}, \\ \tilde{r}=\overline{1, n_{\tilde{k}}}}}{\{p_{\tilde{k}, \tilde{r}}\}}) \sum_{w_4=0}^{\infty}   w_4 \Pr(\varkappa_{4, i}=w_4) + \max_{\substack{\tilde{k}=\overline{0, d}, \\ \tilde{r}=\overline{1, n_{\tilde{k}}}}}{\{\ell(\tilde{k}, \tilde{r}, 1)\}}    \sum_{w_4=0}^{\infty}\Pr(\varkappa_{4, i}=w_4) = \\
     =
      (1-\min_{\substack{\tilde{k}=\overline{0, d}, \\ \tilde{r}=\overline{1, n_{\tilde{k}}}}}{\{p_{\tilde{k}, \tilde{r}}\}})  E[\varkappa_{4, i}] + \max_{\substack{\tilde{k}=\overline{0, d}, \\ \tilde{r}=\overline{1, n_{\tilde{k}}}}}{\{\ell(\tilde{k}, \tilde{r}, 1)\}} .
\end{multline*}
Последовательность $\{M_i; i\geqslant 0\}$,  где 
$$
M_0=E[\varkappa_{4, 0}],  \quad M_{i+1}= (1-\min_{\substack{\tilde{k}=\overline{0, d}, \\ \tilde{r}=\overline{1, n_{\tilde{k}}}}}{\{p_{\tilde{k}, \tilde{r}}\}}) M_{i} + \max_{\substack{\tilde{k}=\overline{0, d}, \\ \tilde{r}=\overline{1, n_{\tilde{k}}}}}{\{\ell(\tilde{k}, \tilde{r}, 1)\}}, 
$$
ограничивает сверху последовательность $E[\varkappa_{4, i+1}]$ и в условиях теоремы ограничена. Это означает,  что размер $\varkappa_{4, i}$  очереди $O_4$ также ограничен равномерно по всем $i\geqslant 0$. Что и следовало доказать.
\end{proof}


\section[Рекуррентные соотношения для производящих функций последовательности\\ ${\MarkThree}$]%
{Рекуррентные соотношения для производящих \\ функций последовательности  $\boldsymbol{\MarkThree}$}
Пусть $\Gamma^{(k, r)}\in \Gamma$ и $x_3 \in {\mathbb Z}_+$. Обозначим 
\begin{equation*}
{\mathbb H}_{-1}(\Gamma^{(k, r)},  x_3) = \{\gamma \in \Gamma \colon h(\gamma,  x_3) = \Gamma^{(k, r)}\}.
\end{equation*}
Тогда из определения \eqref{hLaw} находим явный вид множества ${\mathbb H}_{-1}(\Gamma^{(k, r)},  x_3)$ для различных $\Gamma^{(k, r)}$ и $x_3$:
\begin{equation}
{\mathbb H}_{-1}(\Gamma^{(k, r)},  x_3) = 
\begin{cases}
\{\Gamma^{(k_1, r_1)},  \Gamma^{(0, r\ominus_0 1)}\}, & \quad \text{ если } (k=0\text{ \& } x_3 \leqslant L);\\
\{\Gamma^{(k, r\ominus_k 1)},  \Gamma^{(0, r_2)}\}, & \quad \text{ если } (\Gamma^{(k, r)}\in C_k^{\mathrm{I}} \text{ \& } x_3>L);\\ 
\{\Gamma^{(k, r\ominus_k 1)}\}, & \quad \text{ если } (\Gamma^{(k, r)}\in C_k^{\mathrm{O}}) \text{ или } (\Gamma^{(k, r)}\in C_k^{\mathrm{N}});\\
\varnothing, & \quad \text{ если $(k = 0 \:\&\: x_3>L)$}\\ 
 & \text{ \phantom{1}\qquad или $(\Gamma^{(k, r)}\in C_k^{\mathrm{I}} \:\&\: x_3\leqslant L)$}, 
\end{cases}
\end{equation}
где $k_1, r_1$ таковы,  что $h_1(\Gamma^{(k_1, r_1)})=r$,  и $r_2$ таково,  что $h_3(r_2)=\Gamma^{(k, r)}$.
Обозначим для $\gamma \in \Gamma$ и $x_3 \in {\mathbb Z}_+$
\begin{equation}
Q_{3, i}(\gamma, x) = \Pr(\Gamma_{i}=\gamma,  \varkappa_{3, i}=x_3).
\end{equation}

Для вывода рекуррентных соотношений для частичных производящих функций марковской цепи  ${\MarkThree}$ полезными будут рекуррентные соотношения для вероятностей $\{Q_{3, i}(\gamma, x)\}_{i\geqslant 0, \gamma \in \Gamma, x \in {\mathbb Z}_+}$. Этому посвящена теорема~\ref{prob:rek:theorem}.
\begin{theorem}
Пусть $\tilde{\gamma} =\Gamma^{(\tilde{k}, \tilde{r})}\in \Gamma$ и $\tilde{x}_3 \in {\mathbb Z}_+$. Тогда для переходных вероятностей $\{Q_{3, i}(\cdot, \cdot)\}$,  ${i\geqslant 0}$, марковской цепи $\MarkThree$ имеют место следующие рекуррентные соотношения:
\begin{multline}
Q_{3, i+1}(\tilde{\gamma}, \tilde{x}_3) =\\
=(1-\delta_{\tilde{x}_3, 0}) \sum_{x_3=0}^{\tilde{x}_3 +  \ell(\tilde{k}, \tilde{r}, 3)}\sum_{\gamma \in {\mathbb H}_{-1}(\tilde{\gamma}, x_3)} Q_{3, i}(\gamma, x_3) \times 
\varphi_3(\tilde{x}_3 + \ell(\tilde{k}, \tilde{r}, 3) - x_3, T^{(\tilde{k}, \tilde{r})}) + \\
+ \delta_{\tilde{x}_3, 0} \sum_{x_3=0}^{\ell(\tilde{k}, \tilde{r}, 3)}\sum_{\gamma \in {\mathbb H}_{-1}(\tilde{\gamma}, x_3)} Q_{3, i}(\gamma, x_3) \sum_{a=0}^{\ell(\tilde{k}, \tilde{r}, 3) - x_3} \varphi_3(a, T^{(\tilde{k}, \tilde{r})}).
\label{prob:rek}
\end{multline}
\label{prob:rek:theorem}
\end{theorem}
\begin{proof}
По формуле полной вероятности имеем
\begin{multline*}
Q_{3, i+1}(\tilde{\gamma}, \tilde{x}_3) = \Pr(\Gamma_{i+1}=\tilde{\gamma},  \varkappa_{3, i+1}=\tilde{x}_3) = \\
= \sum_{x_3=0}^{\infty}\sum_{\gamma \in \Gamma} \Pr(\Gamma_{i}=\gamma,  \varkappa_{3, i}=x_3) \times \Pr(\Gamma_{i+1}=\tilde{\gamma},  \varkappa_{3, i+1}=\tilde{x}_3 | \Gamma_{i}=\gamma,  \varkappa_{3, i}=x_3) =  \\ 
= \sum_{x_3=0}^{\infty}\sum_{\gamma \in \Gamma} Q_{3, i}(\gamma, x_3) \times \delta_{\tilde{\gamma}, h(\gamma, x_3)}\times
\Pr(\varkappa_{3, i+1}=\tilde{x}_3 | \Gamma_{i}=\gamma,  \varkappa_{3, i}=x_3).
\end{multline*}
Тогда из определения $ {\mathbb H}_{-1}(\tilde{\gamma}, x_3)$ следует,  что 
\begin{equation*}
Q_{3, i+1}(\tilde{\gamma}, \tilde{x}_3) =\sum_{x_3=0}^{\infty}\sum_{\gamma \in {\mathbb H}_{-1}(\tilde{\gamma}, x_3)} Q_{3, i}(\gamma, x_3) \times 
\Pr(\varkappa_{3, i+1}=\tilde{x}_3 | \Gamma_{i}=\gamma,  \varkappa_{3, i}=x_3),
\end{equation*}
и, учитывая равенство \eqref{kappa:3:conditional}, продолжаем цепочку выкладок:
\begin{multline*}
Q_{3, i+1}(\tilde{\gamma}, \tilde{x}_3)= \sum_{x_3=0}^{\infty}\sum_{\gamma \in {\mathbb H}_{-1}(\tilde{\gamma}, x_3)} Q_{3, i}(\gamma, x_3) \times 
\tilde{\varphi}_3(\tilde{k}, \tilde{r},  T^{(\tilde{k}, \tilde{r})}, x_3, \tilde{x}_3) = \\
= \sum_{x_3=0}^{\infty}\sum_{\gamma \in {\mathbb H}_{-1}(\tilde{\gamma}, x_3)} Q_{3, i}(\gamma, x_3) \times 
[ (1-\delta_{\tilde{x}_3, 0})\varphi_3(\tilde{x}_3 + \ell(\tilde{k}, \tilde{r}, 3) - x_3, T^{(\tilde{k}, \tilde{r})}) + \\ 
+\delta_{\tilde{x}_3, 0} \sum_{a=0}^{\ell(\tilde{k}, \tilde{r}, 3)-x_3}\varphi_3(a, T^{(\tilde{k}, \tilde{r})})].
\end{multline*}
Поскольку  $\varphi_3(x, t)=0$ для $x<0$,  
\begin{multline*}
Q_{3, i+1}(\tilde{\gamma}, \tilde{x}_3)=\\
=(1-\delta_{\tilde{x}_3, 0}) \sum_{x_3=0}^{\tilde{x}_3 +  \ell(\tilde{k}, \tilde{r}, 3)}\sum_{\gamma \in {\mathbb H}_{-1}(\tilde{\gamma}, x_3)} Q_{3, i}(\gamma, x_3) \times 
\varphi_3(\tilde{x}_3 + \ell(\tilde{k}, \tilde{r}, 3) - x_3, T^{(\tilde{k}, \tilde{r})}) + \\
+ \delta_{\tilde{x}_3, 0} \sum_{x_3=0}^{\ell(\tilde{k}, \tilde{r}, 3)}\sum_{\gamma \in {\mathbb H}_{-1}(\tilde{\gamma}, x_3)} Q_{3, i}(\gamma, x_3) \sum_{a=0}^{\ell(\tilde{k}, \tilde{r}, 3) - x_3} \varphi_3(a, T^{(\tilde{k}, \tilde{r})}),
получаем утверждение теоремы
\end{multline*}
\end{proof}
Пусть $k$ и $r$ таковы,  что $\Gamma^{(k, r)}\in \Gamma$. Введем производящую функцию
\begin{equation*}
\mathfrak{M}^{(3, i)}(k, r, v) = \sum_{w=0}^{\infty} Q_{3, i}(\Gamma^{(k, r)}, w) v^w, 
\end{equation*}
и вспомогательные функции
\begin{equation*}
q_{k, r}(v) = v^{-\ell(k, r, 3)}\sum_{w=0}^{\infty} \varphi_3(w, T^{(k, r)})v^w,
\end{equation*}
Также для чисел $k_1$,  $r_1$ и $r_2$,  удовлетворяющих соотношениям $h_1(\Gamma^{(k_1, r_1)}) = \tilde{r}$ и $h_3(r_2)=\Gamma^{(\tilde{k}, \tilde{r})}$,  определим функции
\begin{multline}
\tilde{\alpha}_i(\tilde{k}, \tilde{r}, v) = \sum_{x_3=0}^{\ell(\tilde{k}, \tilde{r}, 3)}\sum_{\gamma \in {\mathbb H}_{-1}(\tilde{\gamma}, x_3)} Q_{3, i}(\gamma, x_3) \sum_{a=0}^{\ell(\tilde{k}, \tilde{r}, 3) - x_3} \varphi_3(a, T^{(\tilde{k}, \tilde{r})}) - \\
- \sum_{x_3=0}^{\ell(\tilde{k}, \tilde{r}, 3)}  \sum_{\gamma \in {\mathbb H}_{-1}(\tilde{\gamma}, x_3)} Q_{3, i}(\gamma, x_3) v^{x_3-\ell(\tilde{k}, \tilde{r}, 3)}  \sum_{w=0}^{\ell(\tilde{k}, \tilde{r}, 3) -x_3}
\varphi_3(w, T^{(\tilde{k}, \tilde{r})}) v^w;
\end{multline}
для $\Gamma^{(0, \tilde{r})} \in \Gamma$:
\begin{multline}
\alpha_i(0, \tilde{r}, v) =\tilde{\alpha}_i(0, \tilde{r}, v) + q_{0, \tilde{r}}(v) \times \sum_{x_3=0}^{L} \left[ Q_{3, i}(\Gamma^{(k_1, r_1)}, x_3) + Q_{3, i}(\Gamma^{(0, \tilde{r}\ominus_0 1)}, x_3) \right] v^{x_3}; \end{multline}
для $ \Gamma^{(\tilde{k},  \tilde{r})} \in C_{\tilde{k}}^{\mathrm{I}}$:
\begin{multline}
\alpha_i(\tilde{k}, \tilde{r}, v) =\tilde{\alpha}_i(\tilde{k}, \tilde{r}, v) - q_{\tilde{k}, \tilde{r}}(v)\sum_{x_3=0}^{L} \left[ Q_{3, i}(\Gamma^{(\tilde{k}, \tilde{r}\ominus_{\tilde{k}} 1)}, x_3) + Q_{3, i}(\Gamma^{(0, r_2)}, x_3) \right] v^{x_3}+ \\ 
+ q_{\tilde{k}, \tilde{r}}(v)  \sum_{x_3\geqslant 0} Q_{3, i}(\Gamma^{(0, r_2)}, x_3) v^{x_3};
\end{multline}
для $\Gamma^{(\tilde{k},  \tilde{r})} \in C_{\tilde{k}}^{\mathrm{O}} \cup C_{\tilde{k}}^{\mathrm{N}}$:
\begin{equation}
\alpha_i(\tilde{k}, \tilde{r}, v) =\tilde{\alpha}_i(\tilde{k}, \tilde{r}, v).
\end{equation}

Важным этапом на пути нахождения условий существования стационарного распределения марковской цепи $\MarkThree$ является нахождение рекуррентных соотношений, которым подчиняются частичные производящие функции. 
\begin{theorem}
Пусть $\tilde{\gamma}=\Gamma^{(\tilde{k}, \tilde{r})} \in \Gamma$. Тогда имеют место следующие рекуррентные по $i \geqslant 0$ соотношения для производящих функций марковской цепи $\MarkThree$:
\begin{enumerate}
\item для $ \Gamma^{(0, \tilde{r})} \in \Gamma$,  $\tilde{r} = \overline{1, n_0}$ 
\begin{equation}
\mathfrak{M}^{(3, i+1)}(0, \tilde{r}, v) = \alpha_i(0, \tilde{r}, v);
\label{three:generation:rek:one}
\end{equation}
\item для $\Gamma^{(\tilde{k}, \tilde{r})} \in \Gamma $,  $\tilde{k} =\overline{1, d}$,  $\tilde{r}=\overline{1, n_{\tilde{k}}}$
\begin{equation}
\mathfrak{M}^{(3, i+1)}(\tilde{k}, \tilde{r}, v) = q_{\tilde{k}, \tilde{r}} (v)\times  \mathfrak{M}^{(3, i)}(\tilde{k}, \tilde{r} \ominus_{\tilde{k}} 1, v) + \alpha_i(\tilde{k}, \tilde{r}, v).
\label{three:generation:rek:two}
\end{equation}
\end{enumerate}

\label{theorem:gen:rek}
\end{theorem}
\begin{proof}
Для доказательства воспользуемся рекуррентными соотношениями \eqref{prob:rek}:
\begin{multline}
\mathfrak{M}^{(3, i+1)}(\tilde{k}, \tilde{r}, v) =\\
=\sum_{w=0}^{\infty} Q_{3, i+1}(\Gamma^{(\tilde{k}, \tilde{r})}, w) v^w = Q_{3, i+1}(\Gamma^{(\tilde{k}, \tilde{r})}, 0) + \sum_{w=1}^{\infty} Q_{3, i+1}(\Gamma^{(\tilde{k}, \tilde{r})}, w) v^w =\\
=\sum_{x_3=0}^{\ell(\tilde{k}, \tilde{r}, 3)}\sum_{\gamma \in {\mathbb H}_{-1}(\tilde{\gamma}, x_3)} Q_{3, i}(\gamma, x_3) \sum_{a=0}^{\ell(\tilde{k}, \tilde{r}, 3) - x_3} \varphi_3(a, T^{(\tilde{k}, \tilde{r})}) + \\
+ \sum_{w=1}^{\infty} \sum_{x_3=0}^{w +  \ell(\tilde{k}, \tilde{r}, 3)}\sum_{\gamma \in {\mathbb H}_{-1}(\tilde{\gamma}, x_3)} Q_{3, i}(\gamma, x_3) \times 
\varphi_3(w + \ell(\tilde{k}, \tilde{r}, 3) - x_3, T^{(\tilde{k}, \tilde{r})}) v^w.
\label{sum:zero}
\end{multline}
После изменения порядка суммирования по $x_3$ и $w$ второе слагаемое распадается еще на два слагаемых
\begin{multline}
\sum_{w=1}^{\infty} \sum_{x_3=0}^{w +  \ell(\tilde{k}, \tilde{r}, 3)}\sum_{\gamma \in {\mathbb H}_{-1}(\tilde{\gamma}, x_3)} Q_{3, i}(\gamma, x_3) \times 
\varphi_3(w + \ell(\tilde{k}, \tilde{r}, 3) - x_3, T^{(\tilde{k}, \tilde{r})}) v^w = \\ \displaybreak[0]
= \sum_{x_3=0}^{\ell(\tilde{k}, \tilde{r}, 3)}\sum_{w=1}^{\infty}\sum_{\gamma \in {\mathbb H}_{-1}(\tilde{\gamma}, x_3)} Q_{3, i}(\gamma, x_3) \times 
\varphi_3(w + \ell(\tilde{k}, \tilde{r}, 3) - x_3, T^{(\tilde{k}, \tilde{r})}) v^w + \\ \displaybreak[0]
+\sum_{x_3=\ell(\tilde{k}, \tilde{r}, 3) + 1}^{\infty}\sum_{w=x_3-\ell(\tilde{k}, \tilde{r}, 3)}^{\infty}\sum_{\gamma \in {\mathbb H}_{-1}(\tilde{\gamma}, x_3)} Q_{3, i}(\gamma, x_3) \times 
\varphi_3(w + \ell(\tilde{k}, \tilde{r}, 3) - x_3, T^{(\tilde{k}, \tilde{r})}) v^w.
\label{double:sum}
\end{multline}
Сначала преобразуем первое слагаемое в правой части выражения~\eqref{double:sum}:
\begin{multline}
 \sum_{x_3=0}^{\ell(\tilde{k}, \tilde{r}, 3)}\sum_{w=1}^{\infty}  \sum_{\gamma \in {\mathbb H}_{-1}(\tilde{\gamma}, x_3)} Q_{3, i}(\gamma, x_3) \times 
\varphi_3(w + \ell(\tilde{k}, \tilde{r}, 3) - x_3, T^{(\tilde{k}, \tilde{r})}) v^w =\displaybreak[0] \\
=  \sum_{x_3=0}^{\ell(\tilde{k}, \tilde{r}, 3)}  \sum_{\gamma \in {\mathbb H}_{-1}(\tilde{\gamma}, x_3)} Q_{3, i}(\gamma, x_3)  \sum_{w=1}^{\infty} 
\varphi_3(w + \ell(\tilde{k}, \tilde{r}, 3) - x_3, T^{(\tilde{k}, \tilde{r})}) v^w =\displaybreak[0] \\
=\sum_{x_3=0}^{\ell(\tilde{k}, \tilde{r}, 3)}  \sum_{\gamma \in {\mathbb H}_{-1}(\tilde{\gamma}, x_3)} Q_{3, i}(\gamma, x_3) v^{x_3-\ell(\tilde{k}, \tilde{r}, 3)}  \sum_{w=\ell(\tilde{k}, \tilde{r}, 3) + 1 -x_3}^{\infty}
\varphi_3(w, T^{(\tilde{k}, \tilde{r})}) v^w.
\label{sum:one}
\end{multline}
Аналогично для второго слагаемого выражения~\eqref{double:sum}:
\begin{multline}
 \sum_{x_3=\ell(\tilde{k}, \tilde{r}, 3) + 1}^{\infty}\sum_{w=x_3-\ell(\tilde{k}, \tilde{r}, 3)}^{\infty} \sum_{\gamma \in {\mathbb H}_{-1}(\tilde{\gamma}, x_3)} Q_{3, i}(\gamma, x_3) \times 
\varphi_3(w + \ell(\tilde{k}, \tilde{r}, 3) - x_3, T^{(\tilde{k}, \tilde{r})}) v^w = \\
=  \sum_{x_3=\ell(\tilde{k}, \tilde{r}, 3) + 1}^{\infty} \sum_{\gamma \in {\mathbb H}_{-1}(\tilde{\gamma}, x_3)} Q_{3, i}(\gamma, x_3) v^{x_3-\ell(\tilde{k}, \tilde{r}, 3)}\sum_{w=0}^{\infty}  
\varphi_3(w, T^{(\tilde{k}, \tilde{r})}) v^w = \\
= \sum_{x_3=0}^{\infty} \sum_{\gamma \in {\mathbb H}_{-1}(\tilde{\gamma}, x_3)} Q_{3, i}(\gamma, x_3) v^{x_3-\ell(\tilde{k}, \tilde{r}, 3)}\sum_{w=0}^{\infty} 
\varphi_3(w, T^{(\tilde{k}, \tilde{r})}) v^w - \\
- \sum_{x_3=0}^{\ell(\tilde{k}, \tilde{r}, 3)} \sum_{\gamma \in {\mathbb H}_{-1}(\tilde{\gamma}, x_3)} Q_{3, i}(\gamma, x_3) v^{x_3-\ell(\tilde{k}, \tilde{r}, 3)}\sum_{w=0}^{\infty}
\varphi_3(w, T^{(\tilde{k}, \tilde{r})}) v^w.
\label{sum:two}
\end{multline}

Подставляя упрощенные выражения \eqref{sum:one} и \eqref{sum:two} в равенство \eqref{double:sum},  и затем равенcтво \eqref{double:sum} в последнее равенство  \eqref{sum:zero},  получим:
\begin{multline}
\mathfrak{M}^{(3, i+1)}(\tilde{k}, \tilde{r}, v) = \sum_{x_3=0}^{\ell(\tilde{k}, \tilde{r}, 3)}\sum_{\gamma \in {\mathbb H}_{-1}(\tilde{\gamma}, x_3)} Q_{3, i}(\gamma, x_3) \sum_{a=0}^{\ell(\tilde{k}, \tilde{r}, 3) - x_3} \varphi_3(a, T^{(\tilde{k}, \tilde{r})}) + \\ \displaybreak[0]
+ \sum_{x_3=0}^{\ell(\tilde{k}, \tilde{r}, 3)}  \sum_{\gamma \in {\mathbb H}_{-1}(\tilde{\gamma}, x_3)} Q_{3, i}(\gamma, x_3) v^{x_3-\ell(\tilde{k}, \tilde{r}, 3)}  \sum_{w=\ell(\tilde{k}, \tilde{r}, 3) + 1 -x_3}^{\infty}
\varphi_3(w, T^{(\tilde{k}, \tilde{r})}) v^w + \\ \displaybreak[0]
+ \sum_{x_3=0}^{\infty} \sum_{\gamma \in {\mathbb H}_{-1}(\tilde{\gamma}, x_3)} Q_{3, i}(\gamma, x_3) v^{x_3-\ell(\tilde{k}, \tilde{r}, 3)}\sum_{w=0}^{\infty} 
\varphi_3(w, T^{(\tilde{k}, \tilde{r})}) v^w - \\ \displaybreak[0]
- \sum_{x_3=0}^{\ell(\tilde{k}, \tilde{r}, 3)} \sum_{\gamma \in {\mathbb H}_{-1}(\tilde{\gamma}, x_3)} Q_{3, i}(\gamma, x_3) v^{x_3-\ell(\tilde{k}, \tilde{r}, 3)}\sum_{w=0}^{\infty}  
\varphi_3(w, T^{(\tilde{k}, \tilde{r})}) v^w.
\label{check:1}
\end{multline}
Сгруппируем второе и четвертое слагаемые в равенстве~\eqref{check:1}:
\begin{multline*}
\mathfrak{M}^{(3, i+1)}(\tilde{k}, \tilde{r}, v)= \sum_{x_3=0}^{\ell(\tilde{k}, \tilde{r}, 3)}\sum_{\gamma \in {\mathbb H}_{-1}(\tilde{\gamma}, x_3)} Q_{3, i}(\gamma, x_3) \sum_{a=0}^{\ell(\tilde{k}, \tilde{r}, 3) - x_3} \varphi_3(a, T^{(\tilde{k}, \tilde{r})}) + \\
+ \sum_{x_3=0}^{\ell(\tilde{k}, \tilde{r}, 3)}  \sum_{\gamma \in {\mathbb H}_{-1}(\tilde{\gamma}, x_3)} Q_{3, i}(\gamma, x_3) v^{x_3-\ell(\tilde{k}, \tilde{r}, 3)}  [ \sum_{w=\ell(\tilde{k}, \tilde{r}, 3) + 1 -x_3}^{\infty}
\varphi_3(w, T^{(\tilde{k}, \tilde{r})}) v^w -\\-\sum_{w=0}^{\infty} 
\varphi_3(w, T^{(\tilde{k}, \tilde{r})}) v^w ] 
+ q_{\tilde{k}, \tilde{r}}(v) \sum_{x_3=0}^{\infty} \sum_{\gamma \in {\mathbb H}_{-1}(\tilde{\gamma}, x_3)} Q_{3, i}(\gamma, x_3) v^{x_3}
\end{multline*}
и,  следовательно, 
\begin{multline}
\mathfrak{M}^{(3, i+1)}(\tilde{k}, \tilde{r}, v)= \sum_{x_3=0}^{\ell(\tilde{k}, \tilde{r}, 3)}\sum_{\gamma \in {\mathbb H}_{-1}(\tilde{\gamma}, x_3)} Q_{3, i}(\gamma, x_3) \sum_{a=0}^{\ell(\tilde{k}, \tilde{r}, 3) - x_3} \varphi_3(a, T^{(\tilde{k}, \tilde{r})}) - \\
-\sum_{x_3=0}^{\ell(\tilde{k}, \tilde{r}, 3)}  \sum_{\gamma \in {\mathbb H}_{-1}(\tilde{\gamma}, x_3)} Q_{3, i}(\gamma, x_3) v^{x_3-\ell(\tilde{k}, \tilde{r}, 3)}   \sum_{w=0}^{\ell(\tilde{k}, \tilde{r}, 3) -x_3}
\varphi_3(w, T^{(\tilde{k}, \tilde{r})}) v^w   +\\
+ q_{\tilde{k}, \tilde{r}}(v) \sum_{x_3=0}^{\infty} \sum_{\gamma \in {\mathbb H}_{-1}(\tilde{\gamma}, x_3)} Q_{3, i}(\gamma, x_3) v^{x_3} =\\
=\tilde{\alpha}_i(\tilde{k}, \tilde{r}, v) + q_{\tilde{k}, \tilde{r}}(v) \sum_{x_3=0}^{\infty} \sum_{\gamma \in {\mathbb H}_{-1}(\tilde{\gamma}, x_3)} Q_{3, i}(\gamma, x_3) v^{x_3}.
\label{rekur:general:second}
\end{multline}

Рассмотрим более подробно сумму $\sum_{x_3=0}^{\infty} \sum_{\gamma \in {\mathbb H}_{-1}(\tilde{\gamma}, x_3)} Q_{3, i}(\gamma, x_3) v^{x_3}$ в зависимости от $\tilde{\gamma}$.
В случае,  если $\tilde{\gamma} = \Gamma^{(0, \tilde{r})}$,  то из определения ${\mathbb H}_{-1}(\cdot, \cdot)$ следует,  что ${\mathbb H}_{-1}(\tilde{\gamma}, x_3) = \{\Gamma^{(k_1, r_1)},  \Gamma^{(0, \tilde{r}\ominus_0 1)}\}$ для $x_3 \leqslant L$ и
${\mathbb H}_{-1}(\tilde{\gamma}, x_3) \hm=\emptyset$ для $x_3 > L$. Здесь пара $(k_1, r_1)$ такая,  что $h_1(\Gamma^{(k_1, r_1)}) = \tilde{r}$. Тогда сумма примет вид
\begin{equation}
\sum_{x_3=0}^{\infty} \sum_{\gamma \in {\mathbb H}_{-1}(\tilde{\gamma}, x_3)} Q_{3, i}(\gamma, x_3) v^{x_3} = \sum_{x_3=0}^{L} \left[ Q_{3, i}(\Gamma^{(k_1, r_1)}, x_3) + Q_{3, i}(\Gamma^{(0, \tilde{r}\ominus_0 1)}, x_3) \right] v^{x_3}.
\label{rekur:additional:first}
\end{equation}
В случае,  если $\tilde{\gamma} \in C_{\tilde{k}}^{\mathrm{I}}$,  из определения  ${\mathbb H}_{-1}(\cdot, \cdot)$ находим,  что ${\mathbb H}_{-1}(\tilde{\gamma}, x_3) = \emptyset$ для $x_3 \leqslant L$ и ${\mathbb H}_{-1}(\tilde{\gamma}, x_3) = \{\Gamma^{(\tilde{k}, \tilde{r}\ominus_{\tilde{k}} 1)},  \Gamma^{(0, r_2)}\}$ для $x_3 > L$. Здесь $r_2$ таково,  что $h_3(r_2)=\Gamma^{(\tilde{k}, \tilde{r})}$. Сумма принимает вид
\begin{multline}
\sum_{x_3=0}^{\infty} \sum_{\gamma \in {\mathbb H}_{-1}(\tilde{\gamma}, x_3)} Q_{3, i}(\gamma, x_3) v^{x_3} = \sum_{x_3=L+1}^{\infty} \left[ Q_{3, i}(\Gamma^{(\tilde{k}, \tilde{r}\ominus_{\tilde{k}} 1)}, x_3) + Q_{3, i}(\Gamma^{(0, r_2)}, x_3) \right] v^{x_3} = \\=
\sum_{x_3=0}^{\infty} \left[ Q_{3, i}(\Gamma^{(\tilde{k}, \tilde{r}\ominus_{\tilde{k}} 1)}, x_3) + Q_{3, i}(\Gamma^{(0, r_2)}, x_3) \right] v^{x_3} -\\
-\sum_{x_3=0}^{L} \left[ Q_{3, i}(\Gamma^{(\tilde{k}, \tilde{r}\ominus_{\tilde{k}} 1)}, x_3) + Q_{3, i}(\Gamma^{(0, r_2)}, x_3) \right] v^{x_3}.
\label{rekur:additional:second}
\end{multline}
И,  наконец,  если $\tilde{\gamma}\in C_{\tilde{k}}^{\mathrm{O}} \cup C_{\tilde{k}}^{\mathrm{N}}$,  то ${\mathbb H}_{-1}(\tilde{\gamma}, x_3) = \{\Gamma^{(\tilde{k}, \tilde{r}\ominus_{\tilde{k}} 1)}\}$ для всех $x_3\geqslant 0$. Тогда сумма будет выглядеть следующим образом:
\begin{equation}
\sum_{x_3=0}^{\infty} \sum_{\gamma \in {\mathbb H}_{-1}(\tilde{\gamma}, x_3)} Q_{3, i}(\gamma, x_3) v^{x_3} = \sum_{x_3=0}^{\infty} Q_{3, i}(\Gamma^{(\tilde{k}, \tilde{r}\ominus_{\tilde{k}} 1)}, x_3) v^{x_3}.
\label{rekur:additional:third}
\end{equation}
Подставляя найденные выражения \eqref{rekur:additional:first},  \eqref{rekur:additional:second} и \eqref{rekur:additional:third} в правую часть равенств \eqref{rekur:general:second},  получаем утверждение теоремы.
\end{proof}

\section[Достаточное условие существования стационарного {распределения} последовательности $\MarkThree$]%
{ Достаточное условие существования стационарного {распределения} последовательности $\boldsymbol{\MarkThree}$}
Получение достаточных условий будет проведено с использованием итеративно-мажорантного подхода (см. работы~\cite{Fedotkin:1988, Fedotkin:1989}). В основе этого подхода лежит изучение динамики одномерных сечений многомерной счетной марковской цепи. 
\begin{lemma}
Для величин $\alpha_i(k, r, v)$,  $\Gamma^{(k, r)} \in \Gamma$,  некоторого фиксированного $0 \hm< \varepsilon_0 < 1$,  существуют конечные постоянные $M(k, r)$,  $\Gamma^{(k, r)}\in \Gamma$,  не зависящие от $v$ и $i$,  такие,  что верны следующие неравенства равномерно для всех $i \geqslant 0$ и $ 1-\varepsilon_0 < v \hm< 1  + \varepsilon_0$:
\begin{equation}
|\alpha_i(k, r, v)| \leqslant M(k, r).
\end{equation}

\end{lemma}
\begin{proof}
Начнем с оценки величины $\tilde{\alpha}_i(k, r, v)$,  входящей в выражение для всех величин $\alpha_i(k, r, v)$. Из определения величин $\tilde{\alpha}_i(k, r, v)$ имеем 
\begin{multline*}
|\tilde{\alpha}_i(k, r, v) | 
\leqslant
\biggl| \sum_{x_3=0}^{\ell(k, r, 3)}\sum_{\gamma \in {\mathbb H}_{-1}(\Gamma^{(k, r)}, x_3)} Q_{3, i}(\gamma, x_3) \sum_{a=0}^{\ell(k, r, 3) - x_3} \varphi_3(a, T^{(k, r)}) \biggr| +  \\
+ \biggl| \sum_{x_3=0}^{\ell(k, r, 3)}  \sum_{\gamma \in {\mathbb H}_{-1}(\Gamma^{(k, r)}, x_3)} Q_{3, i}(\gamma, x_3) v^{x_3-\ell(k, r, 3)}  \sum_{w=0}^{\ell(k, r, 3) -x_3}
\varphi_3(w, T^{(k, r)}) v^w\biggr| \leqslant\\
\leqslant
\biggl| \sum_{x_3=0}^{\ell(k, r, 3)}\sum_{\gamma \in {\mathbb H}_{-1}(\Gamma^{(k, r)}, x_3)}  \sum_{a=0}^{\ell(k, r, 3) - x_3} \varphi_3(a, T^{(k, r)}) \biggr| +  \\
+ \biggl| \sum_{x_3=0}^{\ell(k, r, 3)}  \sum_{\gamma \in {\mathbb H}_{-1}(\Gamma^{(k, r)}, x_3)}  v^{x_3-\ell(k, r, 3)}  \sum_{w=0}^{\ell(k, r, 3) -x_3}
\varphi_3(w, T^{(k, r)}) v^w\biggr|, 
%\leqslant \\ \leqslant \sum_{x_3=0}^{\ell(k, r, 3)}\sum_{\gamma \in {\mathbb H}_{-1}(\Gamma^{(k, r)}, x_3)} Q_{3, i}(\gamma, x_3) \sum_{a=0}^{\ell(k, r, 3) - x_3} \varphi_3(a, T^{(k, r)}) + \\+ \sum_{x_3=0}^{\ell(k, r, 3)}  \sum_{\gamma \in {\mathbb H}_{-1}(\Gamma^{(k, r)}, x_3)} Q_{3, i}(\gamma, x_3) v^{x_3-\ell(k, r, 3)}  \sum_{w=0}^{\ell(k, r, 3) -x_3} \varphi_3(w, T^{(k, r)}) v^w \leqslant \\ \leqslant (\ell(k, r, 3) +1 ) + \sum_{x_3=0}^{\ell(k, r, 3)} \sum_{\gamma \in {\mathbb H}_{-1}(\Gamma^{(k, r)}, x_3)} Q_{3, i}(\gamma, x_3) (1-\varepsilon_0)^{x_3-\ell(k, r, 3)} \exp{(\lambda_3 T^{(k, r)} (f_3(1+\varepsilon_0) - 1))} \leqslant \\ \leqslant \ell(k, r, 3) +1 + (1-\varepsilon_0)^{-\ell(k, r, 3)} \exp{(\lambda_3 T^{(k, r)} (f_3(1+\varepsilon_0) - 1))} \sum_{x_3=0}^{\ell(k, r, 3)} \sum_{\gamma \in {\mathbb H}_{-1}(\Gamma^{(k, r)}, x_3)} Q_{3, i}(\gamma, x_3)  \leqslant \\ \leqslant \ell(k, r, 3) +1 + (1-\varepsilon_0)^{-\ell(k, r, 3)} \exp{(\lambda_3 T^{(k, r)} (f_3(1+\varepsilon_0) - 1))} (\ell(k, r, 3) +1)  = \tilde{M}(k, r)
\end{multline*}
где правая часть последнего неравенства ограничена и не зависит от $i$.
Поскольку все суммы состоят из конечного числа ограниченных слагаемых,  то
\begin{equation*}
|\tilde{\alpha}_i(k, r, v) | \leqslant \tilde{M}(k, r)
\end{equation*}
для некоторой конечной величины $\tilde{M}(k, r)$.

Теперь рассмотрим оставшиеся величины. Для $\Gamma^{(0, r)} \in \Gamma$,  $r=\overline{1, n_0}$:
\begin{equation*}
|\alpha_i(0, r, v)| = | \tilde{\alpha}_i(0, r, v) + q_{0, r}(v) \times \sum_{x_3=0}^{L} \left[ Q_{3, i}(\Gamma^{(k_1, r_1)}, x_3) + Q_{3, i}(\Gamma^{(0, r\ominus_0 1)}, x_3) \right] v^{x_3}|.
%\leqslant \\ \leqslant    | \tilde{\alpha}_i(0, r, v) | + |q_{0, r}(v) |\times  (1+\varepsilon_0)^{L} \sum_{x_3=0}^{L}  \left[ Q_{3, i}(\Gamma^{(k_1, r_1)}, x_3) + Q_{3, i}(\Gamma^{(0, r\ominus_0 1)}, x_3) \right]  \leqslant \\ \leqslant| \tilde{\alpha}_i(0, r, v) | + v ^ {-\ell(0, r, 3)} \exp{(\lambda_3 T^{(0, r)} (f_3(1+\varepsilon_0) - 1))}\times (1+\varepsilon_0)^{L}  \leqslant \tilde{M}(0, r) + \\ + (1-\varepsilon_0) ^ {-\ell(0, r, 3)} \exp{(\lambda_3 T^{(0, r)} (f_3(1+\varepsilon_0) - 1))}\times (1+\varepsilon_0)^{L}  = M(0, r)
\end{equation*}
Поскольку $\tilde{\alpha}_i(0, r, v)$ ограничена,  а приведенная сумма состоит из конечного числа ограниченных слагаемых,  то для ограниченности $|\alpha_i(0, r, v)| $ достаточно показать ограниченность $|q_{0, r}(v)|$. Учитывая определение $\varphi_3(\cdot, \cdot)$,  имеем
\begin{multline*}
|q_{k, r}(v)| = | v^{-\ell(k, r, 3)}\sum_{w=0}^{\infty} \varphi_3(w, T^{(k, r)})v^w| = | v^ {-\ell(k, r, 3)} \exp{(\lambda_3 T^{(k, r)} (f_3(v) - 1))} |
\leqslant \\ 
\leqslant  (1-\varepsilon_0) ^ {-\ell(k, r, 3)} \exp{(\lambda_3 T^{(k, r)} (f_3(1+\varepsilon_0) - 1))}, 
\end{multline*}
где последнее выражение,  очевидно,  ограничено,  $\Gamma^{(k, r)} \in \Gamma$. Следовательно, 
\begin{equation*}
|\alpha_i(0, r, v)| \leqslant M(0, r).
\end{equation*}

Для $\Gamma^{(k,  r)} \in C_{k}^{\mathrm{I}}$ имеем:
\begin{multline*}
|\alpha_i(k, r, v) |=|\tilde{\alpha}_i(k, r, v) - q_{k, r}(v)\sum_{x_3=0}^{L} \left[ Q_{3, i}(\Gamma^{(k, r\ominus_k 1)}, x_3) + Q_{3, i}(\Gamma^{(0, r_2)}, x_3) \right] v^{x_3}+\\  +q_{{k}, {r}}(v)  \sum_{x_3\geqslant 0} Q_{3, i}(\Gamma^{(0, {r})}, x_3) v^{x_3}| = |\tilde{\alpha}_i(k, r, v)  -\\- q_{k, r}(v) \sum_{x_3=0}^{L} \left[ Q_{3, i}(\Gamma^{(k, r\ominus_k 1)}, x_3) + Q_{3, i}(\Gamma^{(0, r_2)}, x_3) \right] v^{x_3} + q_{{k}, {r}}(v)\times \mathfrak{M}^{(3, i)}(0, r, v) |.
%\leqslant \\ \leqslant \tilde{M}(k, r) + (1-\varepsilon_0) ^ {-\ell(k, r, 3)} \exp{(\lambda_3 T^{(k, r)} (f_3(1+\varepsilon_0) - 1))} (1+\varepsilon_0)^{L} +\\+ (1-\varepsilon_0) ^ {-\ell(k, r, 3)} \exp{(\lambda_3 T^{(k, r)} (f_3(1+\varepsilon_0) - 1))} \times |\alpha_{i-1}(0, r, v)| = M(k, r)
\end{multline*}
Поскольку $\mathfrak{M}^{(3, i)}(0, r, v) = \alpha_{i-1}(0, r, v)$,  то 
\begin{multline*}
|\alpha_i(k, r, v) |= |\tilde{\alpha}_i(k, r, v)   q_{k, r}(v) \sum_{x_3=0}^{L} \left[ Q_{3, i}(\Gamma^{(k, r\ominus_k 1)}, x_3) + Q_{3, i}(\Gamma^{(0, r_2)}, x_3) \right] v^{x_3} + \\+q_{{k}, {r}}(v)\times \alpha_{i-1}(0, r, v)|, 
\end{multline*}
где все слагаемые ограничены в силу уже доказанного. Значит, 
\begin{equation*}
|\alpha_i(k, r, v) |\leqslant M(k, r),  \quad \Gamma^{(k,  r)} \in C_{k}^{\mathrm{I}}.
\end{equation*}

В выражении $|\alpha_i(k, r, v)| $ для $\Gamma^{(k,  r)} \in C_{k}^{\mathrm{O}} \cup C_{k}^{\mathrm{N}}$  все величины также ограничены,  следовательно, 
\begin{equation*}
|\alpha_i(k, r, v) |\leqslant M(k, r),  \quad \Gamma^{(k,  r)} \in C_{k}^{\mathrm{O}} \cup C_{k}^{\mathrm{N}}.
\end{equation*}
Этим завершается доказательство.

\end{proof}

Найденные рекуррентные соотношения для производящих функций позволяют сформулировать и доказать следующую теорему.
\begin{theorem}
Для того,  чтобы марковская цепь $\MarkThree$ имела стационарное распределение $Q(\gamma, x)$,  $(\gamma, x)\in \Gamma \times {\mathbb Z}_+$, достаточно выполнения неравенства 
\begin{equation}
\min_{k=\overline{1, d}} { \frac{\sum_{r = 1}^{n_k} \ell(k, r, 3) }{\lambda_3 f_3'(1) \sum_{r=1}^{n_k} T^{(k, r)} }}>1.
\label{sufficient:low}
\end{equation}
\label{sufficient:low:theorem}
\end{theorem}
\begin{proof}
Предположим обратное,  а именно,  что при выполнении условия \eqref{sufficient:low} марковская цепь $\MarkThree$ не имеет стационарного распределения. 
Тогда для любого состояния $(\gamma, x)\in \Gamma \times {\mathbb Z}_+$ и независимо от начального распределения $\Pr(\Gamma_{0}=\Gamma^{(k, r)},  \varkappa_{3, 0}=x)$, 
$(\Gamma^{(k, r)}, x)\in \Gamma \times {\mathbb Z}_+$,  
имеют место предельные равенства 
\begin{equation}
\lim_{i \to \infty} \Pr(\Gamma_{i}=\Gamma^{(k, r)},  \varkappa_{3, i}=x) =0,  \quad  (\Gamma^{(k, r)}, x)\in \Gamma \times {\mathbb Z}_+.
\label{zero:limit:equations}
\end{equation} 
Для доказательства этого факта достаточно рассмотреть все возможные случаи,  предполагая апериодичность рассматриваемой цепи (см. рассуждения \cite[гл. $3$, \S~3-4]{Shiryaev}):
\begin{enumerate}
\item все состояния цепи $\MarkThree$ невозвратные,  тогда предельные соотношения выполняются в силу \cite[с. 541,  лемма $2$]{Shiryaev};
\item существует хотя бы одно возвратное состояние,  тогда все состояния возвратные (поскольку все состояния сообщающиеся); и пусть все состояния нулевые,  тогда предельное соотношение также выполняется \cite[с. 541,  лемма $3$]{Shiryaev};
\item все состояния возвратные и существует хотя бы одно положительное,  тогда все состояния положительные и пределы 
$$
\lim_{i \to \infty} \Pr(\Gamma_{i}=\Gamma^{(k, r)},  \varkappa_{3, i}=x) > 0
$$
являются стационарными вероятностями (см. рассуждения~{\cite[с. 549,  теорема $1$]{Shiryaev}}),  что противоречит предположению.
\end{enumerate}
Для периодической цепи приведенные рассуждения достаточно провести для циклических подклассов.

Выберем это начальное распределение так,  что при некотором $v_0 >1$  будет выполнено неравенство $\mathfrak{M}^{(3, 0)}(k, r, v_0) <\infty$ для всех $\Gamma^{(k, r)}\in \Gamma$. Это ограничение,  в силу теоремы \eqref{theorem:gen:rek},  обеспечивает при любом конечном $i\geqslant 0$ существование функций 
\begin{equation}
\mathfrak{M}^{(3, i)}(k, r, v), \quad \frac{d}{dv} \left[\mathfrak{M}^{(3, i)}(k, r, v)\right],  \quad \Gamma^{(k, r)} \in \Gamma, 
\end{equation}
по крайней мере в некоторой окрестности точки $v=1$.

В силу равенств \eqref{zero:limit:equations} для любого натурального $N$ найдется некоторое число $\mathfrak{I}$,  что для всех $i > \mathfrak{I}$ будет
$$
1 > (1+N) \sum_{x=0}^{N} \sum_{\Gamma^{(k, r)}\in \Gamma}  \Pr(\Gamma_{i}=\Gamma^{(k, r)},  \varkappa_{3, i}=x)
$$ и,  значит,  $1 > (1+N) \sum_{x=0}^{N}  \Pr(\varkappa_{3, i}=x)$. Тогда
\begin{multline*}
E[\varkappa_{3, i}] = \sum_{x=0}^{\infty} x \Pr(\varkappa_{3, i}=x) = \sum_{x=0}^{N} x \Pr(\varkappa_{3, i}=x) + \sum_{x=N+1}^{\infty} x \Pr(\varkappa_{3, i}=x) \geqslant \\ \geqslant  \sum_{x=N+1}^{\infty} x \Pr(\varkappa_{3, i}=x) \geqslant \sum_{x=N+1}^{\infty} (N+1)
\Pr(\varkappa_{3, i}=x) \geqslant (N+1) \sum_{x=N+1}^{\infty} \Pr( \varkappa_{3, i}=x) =\\ =  (N+1) \left(1 - \sum_{x=0}^{N} \Pr(\varkappa_{3, i}=x)\right) \geqslant (N+1) \left(1 - \frac{1}{N+1}\right) = N.
\end{multline*}
Следовательно,  $E[\varkappa_{3, i}]$ неограниченно возрастает при $i \to \infty$. 

Другое рассуждение,  однако,  приводит к противоположному результату. Действительно,  при $\min_{k=\overline{1, d}} { \frac{\sum_{r = 1}^{n_k} \ell(k, r, 3) }{\lambda_3 f_3'(1) \sum_{r=1}^{n_k} T^{(k, r)} }}>1$ имеем для $k=\overline{1, d}$:
\begin{multline}
 \left.\left(\prod_{r=1}^{n_k}q_{k, r}(v)\right) ' \right|_{v=1} = 
  \left.\left(\prod_{r=1}^{n_k}v^{-\ell(k, r, 3)}\sum_{w=0}^{\infty} \varphi_3(w, T^{(k, r)})v^w \right) ' \right|_{v=1} = \\ =
   \left.\left(\prod_{r=1}^{n_k} v^{-\ell(k, r, 3)}\exp(\lambda_3 T^{(k, r)} (f_3(v)-1))\right) ' \right|_{v=1} = \\ =
    \left.\left(v^{-\sum_{r=1}^{n_k}\ell(k, r, 3)}\exp(\lambda_3 (f_3(v)-1)\sum_{r=1}^{n_k} T^{(k, r)}) \right) ' \right|_{v=1} = \\ =
\lambda_3 f_3'(1) \sum_{r=1}^{n_k} T^{(k, r)} -\sum_{r=1}^{n_k} \ell(k, r, 3)  < 0.
\label{derivative:cycle}
\end{multline}

Пусть $\mathfrak{M}_+^{(3, 0)}(k, r, v) =\mathfrak{M}^{(3, 0)}(k, r, v)$. В некоторой окрестности точки  $v = 1$  последовательности $\left\{\mathfrak{M}_+^{(3, i)}(k, r, v)\colon i \geqslant 0\right\}$,  $\Gamma^{(k, r)} \in \Gamma$,  рекуррентного отображения 
\begin{enumerate}
\item для $ \Gamma^{(0, r)} \in \Gamma$ $$\mathfrak{M}_+^{(3, i+1)}(0, r, v) = M(0, r);$$
\item для $\Gamma^{(k,  r)} \in C_{k}$,   $k=1$,  $2$,  $\dots$,  $d$, 
$$\mathfrak{M}_+^{(3,  i+1)}(k, r, v) = q_{k, r} (v)\times  \mathfrak{M}_+^{(3, i)}(k, r \ominus_{k} 1, v) + M(k, r);$$
\end{enumerate}
будут мажорантными соответственно для последовательностей $\{\mathfrak{M}^{(3, i)}(k, r, v)\colon \hm{} i \geqslant 0\}$,  $\Gamma^{(k, r)} \in \Gamma$,  рекуррентного отображения из теоремы~\ref{theorem:gen:rek}.

Из приведенного рекуррентного отображения для мажорантной последовательности видно,  что компонента $\mathfrak{M}_+^{(3, i+1)}(k, r, v)$ зависит только от величины $\mathfrak{M}_+^{(3, i)}(k, r \ominus_{k} 1, v)$ того же цикла $C_k$,  $k=\overline{1, d}$,  и не зависит от величин других циклов. И поскольку числа $M(k, r)$,  $\Gamma^{(k, r)}\in \Gamma$,  конечны и не зависят от $v$ и $i$,  для сходимости всего мажорантного отображения $\left\{\mathfrak{M}_+^{(3, i)}(k, r, v)\colon i \geqslant 0\right\}$,  $\Gamma^{(k, r)} \in \Gamma$,  достаточно сходимости для каждого $k=\overline{1, d}$ подблока $\left\{\mathfrak{M}_+^{(3, i)}(k, r, v)\colon i \geqslant 0\right\}$,  $r =\overline{1, n_k}$. 

Пусть $k =\overline{1, d}$ фиксировано. В матричном виде рекуррентное отображение для блока $\left\{\mathfrak{M}_+^{(3, i)}(k, r, v)\colon i \geqslant 0\right\}$,  $r =\overline{1, n_k}$,  будет иметь вид:
\begin{multline*}
\left[ \begin{array}{c}
    \mathfrak{M}_+^{(3, i+1)}(k, 1, v) \\
    \mathfrak{M}_+^{(3, i+1)}(k, 2, v) \\
    \mathfrak{M}_+^{(3, i+1)}(k, 3, v) \\
    \hdotsfor{1} \\
    \mathfrak{M}_+^{(3, i+1)}(k, n_k, v)
\end{array}\right]
=
\left[ \begin{array}{c c c c c c}
    0       & 0  & \dots & 0 &   q_{k, 1} \\
    q_{k, 2}       & 0  & \dots & 0 & 0 \\
    0       & q_{k, 3}  & \dots  &0  & 0 \\
    \hdotsfor{5} \\
    0       & 0  & \dots &  q_{k, n_k} & 0 \\
\end{array}\right]
\times \\ \times
\left[ \begin{array}{c}
    \mathfrak{M}_+^{(3, i)}(k, 1, v) \\
    \mathfrak{M}_+^{(3, i)}(k, 2, v) \\
    \mathfrak{M}_+^{(3, i)}(k, 3, v) \\
    \hdotsfor{1} \\
    \mathfrak{M}_+^{(3, i)}(k, n_k, v)\\
\end{array}\right]
+
\left[ \begin{array}{c}
    M(k, 1) \\
    M(k, 2) \\
    M(k, 3) \\
    \hdotsfor{1} \\
    M(k, n_k) \\
\end{array}\right].
\end{multline*}
Тогда характеристический многочлен для этого отображения легко подсчитывается и имеет вид 
$$
x^{n_k} - \prod_{r=1}^{n_k}q_{k, r} (v), 
$$
и приравнивая его к нулю,  находим,  что норма всех собственных чисел одинакова и равна $\left(\prod_{r=1}^{n_k}q_{k, r} (v)\right)^{1/n_k}$. В точке $v=1$ модуль собственного числа $|\left(\prod_{r=1}^{n_k}q_{k, r} (v)\right)^{1/n_k}|$ равен $1$,  а его производная
\begin{multline*}
\left(\frac{1}{n_k} \left.\left(\prod_{r=1}^{n_k}q_{k, r} (v)\right)^{1/n_k - 1} \times \left(\prod_{r=1}^{n_k}q_{k, r}(v)\right) '  \right)\right|_{v=1}=\\
=\frac{1}{n_k} \left(\lambda_3 f_3'(1) \sum_{r=1}^{n_k} T^{(k, r)} -\sum_{r=1}^{n_k} \ell(k, r, 3)\right), 
\end{multline*}
в соответствии с условием \eqref{derivative:cycle},  отрицательна.

Следовательно,  в некоторой правой окрестности $v \in (1,  1 + \varepsilon_1)$,  $\varepsilon_1 \hm> 0$,  точки $v=1$ модуль всех собственных чисел $\left(\prod_{r=1}^{n_k}q_{k, r} (v)\right)^{1/n_k}$,  $k\hm=\overline{1, d}$,  будет меньше $1$ и,  значит,  мажорантная последовательность сходится. Этот факт,  в свою очередь,  влечет сходимость исходной последовательности $\left\{\mathfrak{M}^{(3, i)}(k, r, v)\colon i \geqslant 0\right\}$,  $\Gamma^{(k, r)} \in \Gamma$,  для $v \in [1,  1 + \varepsilon_1)$.

Последовательности $\left\{\mathfrak{M}_+^{(3, i)}(k, r, v_1)\colon i \geqslant 0\right\}$,  $\Gamma^{(k, r)} \in \Gamma$,  сходятся при $v_1 \hm\in [1, 1+\varepsilon_1)$ и,  следовательно,  их сумма 
$\sum_{k, r} \mathfrak{M}^{(3, i)}(k, r, v)$ при любом $i\geqslant 0$ является аналитической,  ограниченной  функцией.
И поскольку
\begin{equation}
\sum_{k, r} \mathfrak{M}^{(3, i)}(k, r, v) = \sum_{k, r} \sum_{w=0}^{\infty} Q_{3, i}(\Gamma^{(k, r)}, w) v^w = 
 \sum_{w=0}^{\infty} \Pr(\varkappa_{3, i}=w) v^w, 
\end{equation}
теперь без труда получается,  что числовая последовательность 
$$
\left\{\sum_{k, r} \frac{d}{dv}\left(\mathfrak{M}^{(3, i)}(k, r, v)\right)_{v=1}= E[\varkappa_{3, i}]; i\geqslant 0\right\} 
$$
в силу интегральной формулы Коши равномерно по $i$ ограничена некоторой постоянной величиной. Поэтому принятое предположение не будет справедливым. Доказательство этим завершается.
\end{proof}








\section[Необходимое условие существования стационарного {распределения} последовательности ${\MarkThree}$]%
{Необходимое условие существования стационарного {распределения} последовательности $\boldsymbol{\MarkThree}$}
В этом разделе нас будет интересовать необходимое условие существования стационарного распределения марковской цепи $\MarkThree$.
\begin{theorem}
Для того,  чтобы марковская цепь $\MarkThree$ имела стационарное распределение $Q_3(\gamma, x)$,  $(\gamma, x)\in \Gamma \times {\mathbb Z}_+$,  необходимо выполнение неравенства
$$
\max_{k=\overline{1, d}} { \frac{\sum_{r = 1}^{n_{k}}\ell(k, r, 3)}{\lambda_3 f_3'(1) \sum_{r = 1}^{n_k} T^{(k, r)}} } >1.
$$
\end{theorem}

\begin{proof}

Допустим,  что стационарное распределение марковской цепи $\MarkThree$ существует. Тогда выбрав это распределение в качестве начального $Q_3(\gamma, w)$,  $(\gamma, w)\in \Gamma\times {\mathbb Z}_+$,  обеспечивается существование пределов 
$$
\lim_{i\to \infty} Q_{3, i}(\gamma, w) = Q_3(\gamma, w), 
$$
равных стационарным вероятностям соответствующих состояний. 

Определив производящие функции
$$
\mathfrak{M}^{(3)}(k, r, v) = \sum_{w=0}^{\infty} Q_3(\gamma, w) v^w, 
$$
можем переписать соотношения $\eqref{three:generation:rek:one}$ и $\eqref{three:generation:rek:two}$ в новом виде:
\begin{enumerate}
\item для $ \Gamma^{(0, \tilde{r})} \in \Gamma$,  $\tilde{r} = \overline{1, n_0}$ 
\begin{equation}
\mathfrak{M}^{(3)}(0, \tilde{r}, v) = \alpha(0, \tilde{r}, v);
\label{three:generation:one}
\end{equation}
\item для $\Gamma^{(\tilde{k}, \tilde{r})} \in \Gamma $,  $\tilde{k} =\overline{1, d}$,  $\tilde{r}=\overline{1, n_{\tilde{k}}}$
\begin{equation}
\mathfrak{M}^{(3)}(\tilde{k}, \tilde{r}, v) = q_{\tilde{k}, \tilde{r}} (v)\times  \mathfrak{M}^{(3)}(\tilde{k}, \tilde{r} \ominus_{\tilde{k}} 1, v) + \alpha(\tilde{k}, \tilde{r}, v);
\label{three:generation:two}
\end{equation}
\end{enumerate}
где 
\begin{multline}
\tilde{\alpha}(\tilde{k}, \tilde{r}, v) = \sum_{x_3=0}^{\ell(\tilde{k}, \tilde{r}, 3)}\sum_{\gamma \in {\mathbb H}_{-1}(\tilde{\gamma}, x_3)} Q_3(\gamma, x_3) \sum_{a=0}^{\ell(\tilde{k}, \tilde{r}, 3) - x_3} \varphi_3(a, T^{(\tilde{k}, \tilde{r})}) - \\
- \sum_{x_3=0}^{\ell(\tilde{k}, \tilde{r}, 3)}  \sum_{\gamma \in {\mathbb H}_{-1}(\tilde{\gamma}, x_3)} Q_3(\gamma, x_3) \sum_{w=0}^{\ell(\tilde{k}, \tilde{r}, 3) -x_3}
\varphi_3(w, T^{(\tilde{k}, \tilde{r})}) v^{w-(\ell(\tilde{k}, \tilde{r}, 3)-x_3)},
\end{multline}
для $ \Gamma^{(0, \tilde{r})} \in \Gamma$
\begin{multline}
\alpha(0, \tilde{r}, v) =\tilde{\alpha}(0, \tilde{r}, v) + q_{0, \tilde{r}}(v) \times \sum_{x_3=0}^{L} \left[ Q_3(\Gamma^{(k_1, r_1)}, x_3) + Q_3(\Gamma^{(0, \tilde{r}\ominus_0 1)}, x_3) \right] v^{x_3},
\end{multline}
для $ \Gamma^{(\tilde{k},  \tilde{r})} \in C_{\tilde{k}}^{\mathrm{I}}$
\begin{multline}
\alpha(\tilde{k}, \tilde{r}, v) =\tilde{\alpha}(\tilde{k}, \tilde{r}, v) - q_{\tilde{k}, \tilde{r}}(v)\sum_{x_3=0}^{L} \left[ Q_3(\Gamma^{(\tilde{k}, \tilde{r}\ominus_{\tilde{k}} 1)}, x_3) + Q_3(\Gamma^{(0, r_2)}, x_3) \right] v^{x_3}+ \\ 
+ q_{\tilde{k}, \tilde{r}}(v)  \mathfrak{M}^{(3)}(0, r_2, v),
\end{multline}
и для $\Gamma^{(\tilde{k},  \tilde{r})} \in C_{\tilde{k}}^{\mathrm{O}} \cup C_{\tilde{k}}^{\mathrm{N}}$
\begin{equation}
\alpha(\tilde{k}, \tilde{r}, v) =\tilde{\alpha}(\tilde{k}, \tilde{r}, v).
\end{equation}
Разложим функцию $q_{k, r}(v)$ по формуле Тейлора по степеням $(v-1)$
\begin{multline*}
    q_{k, r}(v) =  v^{-\ell(k, r, 3)} \exp{(\lambda_3 T^{(k, r)} (f_3(v)-1))}= \\=1 + (\lambda_3 T^{(k, r)} f_3'(1) - \ell(k, r, 3))(v-1) + O((v-1)^2).
\end{multline*}
Просуммируем соотношения $\eqref{three:generation:one}$ и $\eqref{three:generation:two}$
\begin{multline}
 \sum_{k=0}^{d} \sum_{r=1}^{n_k} \mathfrak{M}^{(3)}(k, r, v) =\\
 =\sum_{r=1}^{n_0} \alpha (0, r, v) + \sum_{k=1}^{d}\sum_{r=1}^{n_k} \bigl[ q_{k, r}(v) \mathfrak{M}^{(3)}(k, r\ominus_k 1, v) + \alpha(k, r, v)\bigr] = \\
 = \sum_{k=1}^{d}\sum_{r=1}^{n_k} q_{k, r}(v) \mathfrak{M}^{(3)}(k, r\ominus_k 1, v)   + \sum_{k=1}^{d}\sum_{r=1}^{n_k} \alpha(k, r, v)  + \sum_{r=1}^{n_0} \alpha (0, r, v).
 \label{summed:neccessary}
\end{multline}
Разложим по формуле Тейлора слагаемые $\sum_{k=1}^{d}\sum_{r=1}^{n_k} \alpha(k, r, v)$ и $\sum_{r=1}^{n_0} \alpha (0, r, v)$:
\begin{multline*}
    \tilde{\alpha} (\tilde{k}, \tilde{r}, v) = \sum_{x_3=0}^{\ell(\tilde{k}, \tilde{r}, 3)}\sum_{\gamma \in {\mathbb H}_{-1}(\tilde{\gamma}, x_3)} Q_3(\gamma, x_3) \sum_{w=0}^{\ell(\tilde{k}, \tilde{r}, 3) - x_3} \varphi_3(w, T^{(\tilde{k}, \tilde{r})}) (1-v^{w-(\ell(\tilde{k}, \tilde{r}, 3)-x_3)})=\\= - (v-1) \sum_{x_3=0}^{\ell(\tilde{k}, \tilde{r}, 3)}\sum_{\gamma \in {\mathbb H}_{-1}(\tilde{\gamma}, x_3)} Q_3(\gamma, x_3) \sum_{w=0}^{\ell(\tilde{k}, \tilde{r}, 3) - x_3} \varphi_3(w, T^{(\tilde{k}, \tilde{r})}) (w-(\ell(\tilde{k}, \tilde{r}, 3)-x_3)) + \\ + O((v-1)^2).
\end{multline*}
В частности,  для $k=0$ величина $\ell(k, r, 3)$ равна нулю,  поэтому $\tilde{\alpha} (0, r, v) \hm= O((v-1)^2)$.
Теперь непосредственно получим для состояний продления
\begin{multline*}
    \sum_{\tilde{r}=1}^{n_0} \alpha (0, \tilde{r}, v) =\\
    =\sum_{\tilde{r}=1}^{n_0} q_{0, \tilde{r}}(v) \times \sum_{x_3=0}^{L} \bigl[ Q_3(\Gamma^{(k_1, r_1)}, x_3) + Q_3(\Gamma^{(0, \tilde{r}\ominus_0 1)}, x_3) \bigr] v^{x_3} + O((v-1)^2) =\\
    = \sum_{\tilde{r}=1}^{n_0}(1 + (\lambda_3 T^{(0, \tilde{r})} f_3'(1) - \ell(0, \tilde{r}, 3))(v-1))  \times\\
    \times \sum_{x_3=0}^{L} \left[ Q_3(\Gamma^{(k_1, r_1)}, x_3) + Q_3(\Gamma^{(0, \tilde{r}\ominus_0 1)}, x_3) \right] v^{x_3} + O((v-1)^2),
\end{multline*}
для входных состояний
\begin{multline*}
    \sum_{\tilde{k}, \tilde{r}\colon \Gamma^{(\tilde{k},  \tilde{r})} \in C_{\tilde{k}}^{\mathrm{I}}} \alpha(\tilde{k}, \tilde{r}, v) =  \\ \displaybreak[1]  = \sum_{\tilde{k}, \tilde{r}\colon \Gamma^{(\tilde{k},  \tilde{r})} \in C_{\tilde{k}}^{\mathrm{I}}}q_{\tilde{k}, \tilde{r}}(v) \biggl[ \mathfrak{M}^{(3)}(0, r_2, v) -   \sum_{x_3=0}^L\bigl(Q_3(\Gamma^{(\tilde{k}, \tilde{r}\ominus_{\tilde{k}} 1)}, x_3) + Q_3(\Gamma^{(0, r_2)}, x_3) \bigr) v^{x_3} \biggr] + \\ \displaybreak[1]
    + \sum_{\tilde{k}, \tilde{r}\colon \Gamma^{(\tilde{k},  \tilde{r})} \in C_{\tilde{k}}^{\mathrm{I}}} \tilde{\alpha}(\tilde{k}, \tilde{r}, v)
    = \sum_{\tilde{k}, \tilde{r}\colon \Gamma^{(\tilde{k},  \tilde{r})} \in C_{\tilde{k}}^{\mathrm{I}}}(1 + (\lambda_3 T^{(\tilde{k}, \tilde{r})} f_3'(1) - \ell(\tilde{k}, \tilde{r}, 3))(v-1) )\times \\ \displaybreak[1]
     \times \biggl[ \mathfrak{M}^{(3)}(0, r_2, v) -   \sum_{x_3=0}^L\bigl(Q_3(\Gamma^{(\tilde{k}, \tilde{r}\ominus_{\tilde{k}} 1)}, x_3) + Q_3(\Gamma^{(0, r_2)}, x_3) \bigr) v^{x_3} \biggr] -\\ \displaybreak[1] - (v-1)\sum_{\tilde{k}, \tilde{r}\colon \Gamma^{(\tilde{k},  \tilde{r})} \in C_{\tilde{k}}^{\mathrm{I}}} \sum_{x_3=0}^{\ell(\tilde{k}, \tilde{r}, 3)}\sum_{\gamma \in {\mathbb H}_{-1}(\tilde{\gamma}, x_3)} Q_3(\gamma, x_3) \times\\
     \times \sum_{w=0}^{\ell(\tilde{k}, \tilde{r}, 3) - x_3} \varphi_3(w, T^{(\tilde{k}, \tilde{r})}) (w-(\ell(\tilde{k}, \tilde{r}, 3)-x_3)) + O((v-1)^2),
\end{multline*}
и для выходных состояний
\begin{multline*}
    \sum_{\tilde{k}, \tilde{r}\colon \Gamma^{(\tilde{k},  \tilde{r})} \in C_{\tilde{k}}^{\mathrm{O}}\cup C_{\tilde{k}}^{\mathrm{N}}} \alpha(\tilde{k}, \tilde{r}, v) =  \sum_{\tilde{k}, \tilde{r}\colon \Gamma^{(\tilde{k},  \tilde{r})} \in C_{\tilde{k}}^{\mathrm{O}}\cup C_{\tilde{k}}^{\mathrm{N}}} \tilde{\alpha}(\tilde{k}, \tilde{r}, v) = \\ \pagebreak[1] =
    -(v-1)\sum_{\tilde{k}, \tilde{r}\colon \Gamma^{(\tilde{k},  \tilde{r})} \in C_{\tilde{k}}^{\mathrm{O}}\cup C_{\tilde{k}}^{\mathrm{N}}} \sum_{x_3=0}^{\ell(\tilde{k}, \tilde{r}, 3)}\sum_{\gamma \in {\mathbb H}_{-1}(\tilde{\gamma}, x_3)} Q_3(\gamma, x_3)\times\\
    \times\sum_{w=0}^{\ell(\tilde{k}, \tilde{r}, 3) - x_3} \varphi_3(w, T^{(\tilde{k}, \tilde{r})}) (w-(\ell(\tilde{k}, \tilde{r}, 3)-x_3)) + O((v-1)^2).
\end{multline*}

Подставим получившиеся соотношения в выражение \eqref{summed:neccessary}. Поскольку любому входному состоянию системы соответсвует одно и только одно состояние продления и наоборот,  то выражение \eqref{summed:neccessary} примет вид:
\begin{multline}
 0 = O((v-1)^2) + (v-1) \sum_{\tilde{k}=1}^{d}\sum_{\tilde{r}=1}^{n_{\tilde{k}}} (\lambda_3 T^{(\tilde{k}, \tilde{r})} f_3'(1) - \ell(\tilde{k}, \tilde{r}, 3)) \mathfrak{M}^{(3)}(\tilde{k}, \tilde{r}\ominus_{\tilde{k}} 1, v)   +\\ \displaybreak[0] +  (v-1) \sum_{\tilde{r}=1}^{n_0} (\lambda_3 T^{(0, \tilde{r})} f_3'(1) - \ell(0, \tilde{r}, 3))  \times \sum_{x_3=0}^{L} \left[ Q_3(\Gamma^{(k_1, r_1)}, x_3) + Q_3(\Gamma^{(0, \tilde{r}\ominus_0 1)}, x_3) \right] v^{x_3}  +\\ \displaybreak[0] + (v-1) \sum_{\tilde{k}, \tilde{r}\colon \Gamma^{(\tilde{k},  \tilde{r})} \in C_{\tilde{k}}^{\mathrm{I}}} (\lambda_3 T^{(\tilde{k}, \tilde{r})} f_3'(1) - \ell(\tilde{k}, \tilde{r}, 3))\times \\ \displaybreak[0] 
     \times \biggl[ \mathfrak{M}^{(3)}(0, r_2, v) -   \sum_{x_3=0}^L\bigl(Q_3(\Gamma^{(\tilde{k}, \tilde{r}\ominus_{\tilde{k}} 1)}, x_3) + Q_3(\Gamma^{(0, r_2)}, x_3) \bigr) v^{x_3} \biggr] -\\ \displaybreak[0]- (v-1) \sum_{\tilde{k}, \tilde{r}\colon \Gamma^{(\tilde{k},  \tilde{r})} \in C_{\tilde{k}}^{\mathrm{O}}\cup C_{\tilde{k}}^{\mathrm{N}}} \sum_{x_3=0}^{\ell(\tilde{k}, \tilde{r}, 3)}\sum_{\gamma \in {\mathbb H}_{-1}(\tilde{\gamma}, x_3)} Q_3(\gamma, x_3) \times \\ \displaybreak[0] \times \sum_{w=0}^{\ell(\tilde{k}, \tilde{r}, 3) - x_3} \varphi_3(w, T^{(\tilde{k}, \tilde{r})}) (w-(\ell(\tilde{k}, \tilde{r}, 3)-x_3)).
\end{multline}
Разделим обе части равенства на $(v-1)$,  устремим $v$ к единице и сгруппируем слагаемые:
\begin{multline}
\allowdisplaybreaks
 0 = \sum_{\tilde{k}, \tilde{r}\colon \Gamma^{(\tilde{k},  \tilde{r})} \in C_{\tilde{k}}^{\mathrm{I}}} (\lambda_3 T^{(\tilde{k}, \tilde{r})} f_3'(1) - \ell(\tilde{k}, \tilde{r}, 3))\times \\ 
     \times \biggl[\mathfrak{M}^{(3)}(\tilde{k}, \tilde{r}\ominus_{\tilde{k}} 1, 1) - \sum_{x_3=0}^L Q_3(\Gamma^{(\tilde{k}, \tilde{r}\ominus_{\tilde{k}} 1)}, x_3)   +  \mathfrak{M}^{(3)}(0, r_2, 1) -   \sum_{x_3=0}^L  Q_3(\Gamma^{(0, r_2)}, x_3)  \biggr]
 +\\+ \sum_{\tilde{k}, \tilde{r}\colon \Gamma^{(\tilde{k},  \tilde{r})} \in C_{\tilde{k}}^{\mathrm{O}}\cup C_{\tilde{k}}^{\mathrm{N}}} (\lambda_3 T^{(\tilde{k}, \tilde{r})} f_3'(1) - \ell(\tilde{k}, \tilde{r}, 3)) \mathfrak{M}^{(3)}(\tilde{k}, \tilde{r}\ominus_{\tilde{k}} 1, 1)   +\\+ \sum_{\tilde{r}=1}^{n_0} \lambda_3 T^{(0, \tilde{r})} f_3'(1)  \times \sum_{x_3=0}^{L} \left[ Q_3(\Gamma^{(k_1, r_1)}, x_3) + Q_3(\Gamma^{(0, \tilde{r}\ominus_0 1)}, x_3) \right]   +\\+   \sum_{\substack{\tilde{k}, \tilde{r}\colon \\ \Gamma^{(\tilde{k},  \tilde{r})} \in C_{\tilde{k}}^{\mathrm{O}}\cup C_{\tilde{k}}^{\mathrm{N}}}} \sum_{x_3=0}^{\ell(\tilde{k}, \tilde{r}, 3)}\sum_{\gamma \in {\mathbb H}_{-1}(\tilde{\gamma}, x_3)} Q_3(\gamma, x_3) \sum_{w=0}^{\ell(\tilde{k}, \tilde{r}, 3) - x_3} \varphi_3(w, T^{(\tilde{k}, \tilde{r})}) (\ell(\tilde{k}, \tilde{r}, 3)-x_3 - w).
 \label{neccessary:to:paste}
\end{multline}
Подставим в \eqref{three:generation:one} и \eqref{three:generation:two} значение $v=1$:
\begin{align*}
    \mathfrak{M}^{(3)}(0, r, 1) &= \sum_{x_3=0}^{L} \biggl(Q(\Gamma^{(k_1, r_1)},  x_3) + Q_3(\Gamma^{(0, r\ominus_{0}1)},  x_3) \biggr), \\
    \mathfrak{M}^{(3)}(k, r, 1) &=\mathfrak{M}^{(3)}(k, r\ominus_{k}1, 1) +\mathfrak{M}^{(3)}(0, r_2, 1)-\\
    - &\sum_{x_3=0}^{L} \biggl(Q_3(\Gamma^{(k, r\ominus_{k}1)},  x_3) + Q_3(\Gamma^{(0, r_2)},  x_3) \biggr),   
    \quad \Gamma^{(\tilde{k},  \tilde{r})} \in C_{\tilde{k}}^{\mathrm{I}},\\
    \mathfrak{M}^{(3)}(k, r, 1) &= \mathfrak{M}^{(3)}(k, r\ominus_k 1, 1),  \quad \Gamma^{(\tilde{k},  \tilde{r})} \in C_{\tilde{k}}^{\mathrm{O}} \bigcup C_{\tilde{k}}^{\mathrm{N}}.
\end{align*}
Из последнего равенства следует,  что 
$$
\mathfrak{M}^{(3)}(k, n_k, 1) =  \mathfrak{M}^{(3)}(k, n_k\ominus_k 1, 1) = \ldots =  \mathfrak{M}^{(3)}(k, 1, 1) = M_k.
$$
Упростим с помощью получившихся равенств выражение \eqref{neccessary:to:paste}:
\begin{multline}
 0 = \sum_{\tilde{k}, \tilde{r}\colon \Gamma^{(\tilde{k},  \tilde{r})} \in C_{\tilde{k}}^{\mathrm{I}}} (\lambda_3 T^{(\tilde{k}, \tilde{r})} f_3'(1) - \ell(\tilde{k}, \tilde{r}, 3))
     \times   M_{\tilde{k}}
 +\\+ \sum_{\tilde{k}, \tilde{r}\colon \Gamma^{(\tilde{k},  \tilde{r})} \in C_{\tilde{k}}^{\mathrm{O}}\cup C_{\tilde{k}}^{\mathrm{N}}} (\lambda_3 T^{(\tilde{k}, \tilde{r})} f_3'(1) - \ell(\tilde{k}, \tilde{r}, 3)) M_{\tilde{k}}   +\\+ \sum_{\tilde{r}=1}^{n_0} \lambda_3 T^{(0, \tilde{r})} f_3'(1)  \times \sum_{x_3=0}^{L} \left[ Q_3(\Gamma^{(k_1, r_1)}, x_3) + Q_3(\Gamma^{(0, \tilde{r}\ominus_0 1)}, x_3) \right]   +\\+  \sum_{\substack{\tilde{k}, \tilde{r}\colon\\ \Gamma^{(\tilde{k},  \tilde{r})} \in C_{\tilde{k}}^{\mathrm{O}}\cup C_{\tilde{k}}^{\mathrm{N}}}} \sum_{x_3=0}^{\ell(\tilde{k}, \tilde{r}, 3)}\sum_{\gamma \in {\mathbb H}_{-1}(\tilde{\gamma}, x_3)} Q_3(\gamma, x_3) \sum_{w=0}^{\ell(\tilde{k}, \tilde{r}, 3) - x_3} \varphi_3(w, T^{(\tilde{k}, \tilde{r})}) (\ell(\tilde{k}, \tilde{r}, 3)-x_3 - w).
 \label{neccessary:to:paste:one}
\end{multline}
Сгруппируем слагаемые:
\begin{multline}
 0 = \sum_{\tilde{k} = 1}^d \biggl( M_{\tilde{k}} \times \sum_{\tilde{r} = 1}^{n_{\tilde{k}}} (\lambda_3 T^{(\tilde{k}, \tilde{r})} f_3'(1) - \ell(\tilde{k}, \tilde{r}, 3))\biggr)
 +\\+ \sum_{\tilde{r}=1}^{n_0} \lambda_3 T^{(0, \tilde{r})} f_3'(1)  \times \sum_{x_3=0}^{L} \left[ Q_3(\Gamma^{(k_1, r_1)}, x_3) + Q_3(\Gamma^{(0, \tilde{r}\ominus_0 1)}, x_3) \right]   +\\+  \sum_{\substack{\tilde{k}, \tilde{r}\colon \\ \Gamma^{(\tilde{k},  \tilde{r})} \in C_{\tilde{k}}^{\mathrm{O}}\cup C_{\tilde{k}}^{\mathrm{N}}}} \sum_{x_3=0}^{\ell(\tilde{k}, \tilde{r}, 3)}\sum_{\gamma \in {\mathbb H}_{-1}(\tilde{\gamma}, x_3)} Q_3(\gamma, x_3) \sum_{w=0}^{\ell(\tilde{k}, \tilde{r}, 3) - x_3} \varphi_3(w, T^{(\tilde{k}, \tilde{r})}) (\ell(\tilde{k}, \tilde{r}, 3)-x_3 - w).
 \label{neccessary:to:paste:two}
\end{multline}
Предположение,  что для любых $\tilde{k}=\overline{1, d}$  выражение 
$$
\sum_{\tilde{r} = 1}^{n_{\tilde{k}}}\ell(\tilde{k}, \tilde{r}, 3) \big/\lambda_3 f_3'(1) \sum_{\tilde{r} = 1}^{n_{\tilde{k}}} T^{(\tilde{k}, \tilde{r})} 
$$
меньше либо равно $1$,  приводит к невозможному выводу $Q_3(\Gamma^{(0, 1)}, 0) = 0$. Что и требовалось доказать.

\end{proof}

















\section[Достаточное условие существования стационарного распределения последовательности $\{(\Gamma_i,  \varkappa_{1, i}, \varkappa_{3, i}); i \geqslant 0\}$]{Достаточное условие существования стационарного распределения последовательности $\boldsymbol{\{(\Gamma_i,  \varkappa_{1, i}, \varkappa_{3, i}); i \geqslant 0\}}$}
Теперь добавим состояние $\varkappa_{1,i}$ очереди $O_1$ в качестве дополнительной компоненты состояния марковской цепи $\MarkThree$.
Докажем марковость получившейся последовательности $\{(\Gamma_i,  \varkappa_{1, i}, \varkappa_{3, i}); i \geqslant 0\}$.
\begin{theorem}
Пусть $\Gamma_0=\Gamma^{(k, r)}\in \Gamma$ и $(\varkappa_{1, 0},  \varkappa_{3, 0})=(x_{1, 0},  x_{3, 0})\in \mathbb{Z}_+^2$ фиксированы. Тогда последовательность $\{(\Gamma_i,  \varkappa_{1, i}, \varkappa_{3, i}); i \geqslant 0\}$ является однородной счетной цепью Маркова.
\end{theorem}
\begin{proof}
Действительно,  поскольку $\Gamma_{i+1}$ функционально выражается через $\Gamma_i$ и $\varkappa_{3, i}$ (см.~\eqref{gammaFunc}),  то
\begin{multline*}
\Pr (\{ \Gamma_{i+1} =\Gamma^{(k_{i+1}, r_{i+1})}, \varkappa_{1, i+1} = x_{1, i+1}, \varkappa_{3, i+1} = x_{3, i+1}\} | \cap_{t=0}^{i} A_t(k_t;r_t;x^t))=\\
=\delta_{\Gamma^{(k_{i+1}, r_{i+1})}, h(\Gamma^{(k_i, r_i)}, x_{3, i})}\times\\
\times \Pr (\{ \varkappa_{1, i+1} = x_{1, i+1},   \varkappa_{3, i+1} = x_{3, i+1}\} |\cap_{t=0}^{i} A_t(k_t;r_t;x^t)), 
\end{multline*}
для $\Gamma^{(k_i, r_i)}\in \Gamma$,  $(x_{1, i},  x_{3, i})\in {\mathbb Z}_+^2$,  $i\geqslant 0$. Множество $A_t(k_t;r_t;x^t)$ определено в~\eqref{A:definition}. Учитывая равенство \eqref{kappa:1:kappa:3:conditional},  убеждаемся,  что вероятность 
\begin{multline*}
\Pr (\{ \Gamma_{i+1} =\Gamma^{(k_{i+1}, r_{i+1})}, \varkappa_{1, i+1} = x_{1, i+1}, \varkappa_{3, i+1} = x_{3, i+1}\} | \cap_{t=0}^{i}A_t(k_t;r_t;x^t)) = \\
=\delta_{\Gamma^{(k_{i+1}, r_{i+1})}, h(\Gamma^{(k_i, r_i)}, x_{3, i})} \times \widetilde{\varphi}_3(k_{i+1}, r_{i+1}, h_T(\Gamma^{(k_i, r_i)}, x_{3, i}), x_{3, i}, x_{3, i+1})
\times \\ \times \widetilde{\varphi}_1(k_{i+1}, r_{i+1}, h_T(\Gamma^{(k_i, r_i)}, x_{3, i}), x_{1, i}, x_{1, i+1})
\end{multline*}
зависит только от значений $(\Gamma_i, \varkappa_{1, i}, \varkappa_{3, i})$ и $(\Gamma_{i+1}, \varkappa_{1, i+1},  \varkappa_{3, i+1})$. Следовательно,  
\begin{multline*}
\Pr (\{ \Gamma_{i+1} =\Gamma^{(k_{i+1}, r_{i+1})}, \varkappa_{1, i+1} = x_{1, i+1}, \varkappa_{3, i+1} = x_{3, i+1}\} |\cap_{t=0}^{i}A_t(k_t;r_t;x^t))=\\
=\Pr (\{  \Gamma_{i+1} =\Gamma^{(k_{i+1}, r_{i+1})}, \varkappa_{1, i+1} = x_{1, i+1}, \varkappa_{3, i+1} = x_{3, i+1}\} | \\ | \{ \Gamma_i=\Gamma^{(k_i, r_i)}, \varkappa_{1, i}=x_{1, i},  \varkappa_{3, i}=x_{3, i}\}) = \\
=\Pr (\{ \Gamma_{i+1} =\Gamma^{(k_{i+1}, r_{i+1})},  \varkappa_{1, i+1} = x_{1, i+1}, \varkappa_{3, i+1} = x_{3, i+1}\} | \\ |\cap_{t=0}^{i}\{ \Gamma_t=\Gamma^{(k_t, r_t)},  \varkappa_{1, t}=x_{1, t},  \varkappa_{3, t}=x_{3, t}\}), 
\end{multline*}
что доказывает марковость последовательности $\{(\Gamma_i,  \varkappa_{1, i}, \varkappa_{3, i}); i \geqslant 0\}$.
\end{proof}
Обозначим для $\gamma \in \Gamma$ и $(x_1, x_3) \in {\mathbb Z}_+^2$
\begin{equation}
Q_{1, i}(\gamma, x_1, x_3) = \Pr(\Gamma_{i}=\gamma,  \varkappa_{1, i}=x_1,  \varkappa_{3, i}=x_3).
\end{equation}
Как и ранее, для нахождения рекуррентных соотношений для частичных производящих функций марковской цепи $\{(\Gamma_i,  \varkappa_{1, i}, \varkappa_{3, i}); i \geqslant 0\}$, важно найти рекуррентные соотношения для вероятностей $Q_{1, i}(\gamma, x_1, x_3)$.
\begin{theorem}
Пусть $\tilde{\gamma} =\Gamma^{(\tilde{k}, \tilde{r})}\in \Gamma$ и $(\tilde{x}_1,  \tilde{x}_3) \in {\mathbb Z}_+^2$. Тогда для переходных вероятностей $\{Q_{1, i}(\cdot, \cdot, \cdot)\}_{i\geqslant 0}$ марковской цепи $\{(\Gamma_i,  \varkappa_{1, i}, \varkappa_{3, i}); i \geqslant 0\} $ имеют место следующие рекуррентные соотношения:
\begin{multline*}
Q_{1, i+1}(\tilde{\gamma}, 0,  0)= 
\sum_{x_1=0}^{\ell(\tilde{k}, \tilde{r}, 1)} \sum_{x_3=0}^{\ell(\tilde{k}, \tilde{r}, 3)} \sum_{\gamma \in {\mathbb H}_{-1}(\tilde{\gamma}, x_3)}Q_{1, i}(\gamma, x_1,  x_3)\times \\ \times
\sum_{a=0}^{\ell(\tilde{k}, \tilde{r}, 3)-x_3}\varphi_3(a, T^{(\tilde{k}, \tilde{r})}) \times \sum_{a=0}^{\ell(\tilde{k}, \tilde{r}, 1)-x_1}\varphi_1(a, T^{(\tilde{k}, \tilde{r})}), 
\end{multline*}
\begin{multline*}
Q_{1, i+1}(\tilde{\gamma}, 0,  \tilde{x}_3)= 
\sum_{x_1=0}^{\ell(\tilde{k}, \tilde{r}, 1)} \sum_{x_3=0}^{\tilde{x}_3 + \ell(\tilde{k}, \tilde{r}, 3)} \sum_{\gamma \in {\mathbb H}_{-1}(\tilde{\gamma}, x_3)}Q_{1, i}(\gamma, x_1,  x_3) \times  \\ \times \varphi_3(\tilde{x}_3 + \ell(\tilde{k}, \tilde{r}, 3) - x_3, T^{(\tilde{k}, \tilde{r})})  \times \sum_{a=0}^{\ell(\tilde{k}, \tilde{r}, 1)-x_1}\varphi_1(a, T^{(\tilde{k}, \tilde{r})}),  \quad \tilde{x}_3 > 0, 
\end{multline*}
\begin{multline*}
Q_{1, i+1}(\tilde{\gamma}, \tilde{x}_1,  0)= 
\sum_{x_1=0}^{\tilde{x}_1 + \ell(\tilde{k}, \tilde{r}, 1) } \sum_{x_3=0}^{\ell(\tilde{k}, \tilde{r}, 3)} \sum_{\gamma \in {\mathbb H}_{-1}(\tilde{\gamma}, x_3)}Q_{1, i}(\gamma, x_1,  x_3) \times  \\ \times \sum_{a=0}^{\ell(\tilde{k}, \tilde{r}, 3)-x_3}\varphi_3(a, T^{(\tilde{k}, \tilde{r})}) \times \varphi_1(\tilde{x}_1 + \ell(\tilde{k}, \tilde{r}, 1) - x_1, T^{(\tilde{k}, \tilde{r})}),  \quad \tilde{x}_1 > 0, 
\end{multline*}
\begin{multline*}
Q_{1, i+1}(\tilde{\gamma}, \tilde{x}_1,  \tilde{x}_3)= 
\sum_{x_1=0}^{\tilde{x}_1 +\ell(\tilde{k}, \tilde{r}, 1)} \sum_{x_3=0}^{\tilde{x}_3 +\ell(\tilde{k}, \tilde{r}, 3)} \sum_{\gamma \in {\mathbb H}_{-1}(\tilde{\gamma}, x_3)}Q_{1, i}(\gamma, x_1,  x_3) \times  \\ \times \varphi_3(\tilde{x}_3 + \ell(\tilde{k}, \tilde{r}, 3) - x_3, T^{(\tilde{k}, \tilde{r})})  \times \varphi_1(\tilde{x}_1 + \ell(\tilde{k}, \tilde{r}, 1)-x_1, T^{(\tilde{k}, \tilde{r})}),  \quad \tilde{x}_1 > 0,  \tilde{x}_3 > 0.
\end{multline*}
\label{prob:rek:1}
\end{theorem}
\begin{proof}
По формуле полной вероятности имеем
\begin{multline*}
Q_{1, i+1}(\tilde{\gamma}, \tilde{x}_1, \tilde{x}_3) = \Pr(\Gamma_{i+1}=\tilde{\gamma},  \varkappa_{1, i+1}=\tilde{x}_1,  \varkappa_{3, i+1}=\tilde{x}_3) = \\
= \sum_{x_1=0}^{\infty}\sum_{x_3=0}^{\infty}\sum_{\gamma \in \Gamma} \Pr(\Gamma_{i}=\gamma,  \varkappa_{1, i}=x_1,  \varkappa_{3, i}=x_3) \times \\ \times  \Pr(\Gamma_{i+1}=\tilde{\gamma},  \varkappa_{1, i+1}=\tilde{x}_1,  \varkappa_{3, i+1}=\tilde{x}_3 | \Gamma_{i}=\gamma, \varkappa_{1, i}=x_1,  \varkappa_{3, i}=x_3) =  \\ 
=\sum_{x_1=0}^{\infty} \sum_{x_3=0}^{\infty}\sum_{\gamma \in \Gamma} Q_{1, i}(\gamma,  x_1,  x_3) \times \delta_{\tilde{\gamma}, h(\gamma, x_3)}\times \\ \times
\Pr(\varkappa_{1, i+1}=\tilde{x}_1 ,  \varkappa_{3, i+1}=\tilde{x}_3 | \Gamma_{i}=\gamma,  \varkappa_{1, i}=x_1,  \varkappa_{3, i}=x_3).
\end{multline*}
Тогда из определения $ {\mathbb H}_{-1}(\tilde{\gamma}, x_3)$ следует,  что 
\begin{multline*}
Q_{1, i+1}(\tilde{\gamma}, \tilde{x}_1,  \tilde{x}_3) =\sum_{x_1=0}^{\infty} \sum_{x_3=0}^{\infty}\sum_{\gamma \in {\mathbb H}_{-1}(\tilde{\gamma}, x_3)} Q_{1, i}(\gamma, x_1,  x_3) \times \\ \times 
\Pr(\varkappa_{1, i+1}=\tilde{x}_1,  \varkappa_{3, i+1}=\tilde{x}_3 | \Gamma_{i}=\gamma,  \varkappa_{1, i}=x_1,  \varkappa_{3, i}=x_3)
\end{multline*}
и, учитывая \eqref{kappa:1:kappa:3:conditional}, продолжаем цепочку выкладок
\begin{multline*}
Q_{1, i+1}(\tilde{\gamma}, \tilde{x}_1,  \tilde{x}_3)=\sum_{x_1=0}^{\infty} \sum_{x_3=0}^{\infty}\sum_{\gamma \in {\mathbb H}_{-1}(\tilde{\gamma}, x_3)} Q_{1, i}(\gamma, x_1, x_3) \times 
\tilde{\varphi}_3(\tilde{k}, \tilde{r},  T^{(\tilde{k}, \tilde{r})}, x_3, \tilde{x}_3) \times \\ \times 
\tilde{\varphi}_1(\tilde{k}, \tilde{r},  T^{(\tilde{k}, \tilde{r})}, x_1, \tilde{x}_1) 
= \sum_{x_1=0}^{\infty} \sum_{x_3=0}^{\infty} \sum_{\gamma \in {\mathbb H}_{-1}(\tilde{\gamma}, x_3)} Q_{1, i}(\gamma, x_1,  x_3) \times \\ \times
[ (1-\delta_{\tilde{x}_3, 0})\varphi_3(\tilde{x}_3 + \ell(\tilde{k}, \tilde{r}, 3) - x_3, T^{(\tilde{k}, \tilde{r})}) +\delta_{\tilde{x}_3, 0} \sum_{a=0}^{\ell(\tilde{k}, \tilde{r}, 3)-x_3}\varphi_3(a, T^{(\tilde{k}, \tilde{r})})] \times \\ 
\times 
[ (1-\delta_{\tilde{x}_1, 0})\varphi_1(\tilde{x}_1 + \ell(\tilde{k}, \tilde{r}, 1) - x_1, T^{(\tilde{k}, \tilde{r})}) +\delta_{\tilde{x}_1, 0} \sum_{a=0}^{\ell(\tilde{k}, \tilde{r}, 1)-x_1}\varphi_1(a, T^{(\tilde{k}, \tilde{r})})].
\end{multline*}

Поскольку  $\varphi_i(x, t)=0$ для $x<0$,  $i=1, 2$,  получаем утверждение теоремы.
\end{proof}

Пусть
\begin{align*}
  S^1_{0, r} = & 
  \biggl\{
  (\Gamma^{(0, r)}, x_1,  x_3) \colon \; (x_1,  x_3)\in Z^2_+, \; x_3 > L - \max\limits_{k=1,  2, 
    \ldots,  d}
  \biggl\{ \sum_{t=1}^{n_k} \ell({k, t, 3}) \biggl\}\biggl\}, \\
   & 1 \leqslant r \leqslant n_0,  \\
  S^1_{k, r} = & 
  \biggl\{
  (\Gamma^{(k, r)}, x_1,  x_3) \colon \; (x_1,  x_3)\in Z^2_+, \; x_3 > L - \sum_{t=1}^{r} \ell({k, t, 3})
  \biggr\},   \\
  & 1 \leqslant k \leqslant d,  \quad 1 \leqslant r \leqslant n_k.
\end{align*}

Из теоремы~\ref{important:states:basic} следует 
\begin{theorem}
Множество существенных состояний марковской цепи $\{(\Gamma_i,  \varkappa_{1, i}, \varkappa_{3, i}); i \geqslant 0\}$ имеет вид $\raisebox{-.5ex}{$\biggl($}\bigcup\limits_{1 \leqslant r \leqslant n_0}S^1_{0, r}\raisebox{-.5ex}{$\biggr)$}\cup \raisebox{-.5ex}{$\biggl($}\bigcup\limits_{\substack{1 \leqslant k \leqslant d\\ 1 \leqslant r \leqslant n_k}} S^1_{k, r}\raisebox{-.5ex}{$\biggr)$}$.
\end{theorem}
Пусть $k$ и $r$ таковы,  что $\Gamma^{(k, r)}\in \Gamma$. Введем частичные производящие функции
\begin{equation*}
\mathfrak{M}^{(1, i)}(k, r, v_1, v_3) = \sum_{w_1=0}^{\infty}\sum_{w_3=0}^{\infty} Q_{1, i}(\Gamma^{(k, r)}, w_1, w_3) v_1^{w_1} v_3^{w_3},
\end{equation*}
и вспомогательные функции
\begin{align*}
q^{(1)}(k,r, v_1) &= v_1^{-\ell(k,r,1)}\sum_{w=0}^{\infty} \varphi_1(w,T^{(k,r)})v_1^w;\\
q^{(3)}(k,r, v_3) &= v_3^{-\ell(k,r,3)}\sum_{w=0}^{\infty} \varphi_3(w,T^{(k,r)})v_3^w.
\end{align*}

\begin{lemma}
Пусть  $\tilde{\gamma}=\Gamma^{(\tilde{k}, \tilde{r})}\in \Gamma$. Тогда верно следующее соотношение:
\begin{multline*}
\mathfrak{M}^{(1, i+1)}(\tilde{k}, \tilde{r}, v_1, v_3) 
= \sum_{w_1=0}^{\infty}\sum_{w_3=0}^{\infty} \sum_{\gamma \in {\mathbb H}_{-1}(\tilde{\gamma}, w_3)} Q_{1, i}(\Gamma^{(k, r)}, w_1, w_3) \times \\ \times \bigl[ v_1^{w_1} q^{(1)}(\tilde{k}, \tilde{r}, v_1) + I(\tilde{\gamma}\in \Gamma^{\mathrm{I}}) \sum_{a=0}^{\ell(\tilde{k}, \tilde{r}, 1)-w_1} \varphi_1(a, T^{(\tilde{k}, \tilde{r})})(1-v_1^{w_1+a-\ell(\tilde{k}, \tilde{r}, 1)})\bigr] \times \\ 
\times \bigl[ v_3^{w_3} q^{(3)}(\tilde{k}, \tilde{r}, v_3) + I(\tilde{\gamma}\in \Gamma^{\mathrm{III}}) \sum_{a=0}^{\ell(\tilde{k}, \tilde{r}, 3)-w_3} \varphi_3(a, T^{(\tilde{k}, \tilde{r})})(1-v_3^{w_3+a-\ell(\tilde{k}, \tilde{r}, 3)})\bigr].
\end{multline*}
\label{second:approach:lemma:first:step}
\end{lemma}
\begin{proof}
Запишем по формуле повторного математического ожидания:
\begin{multline}
\mathfrak{M}^{(1, i+1)}(\tilde{k}, \tilde{r}, v_1, v_3) =\\
= E[v_1^{\varkappa_{1, i+1}}v_3^{\varkappa_{1, i+1}}I(\Gamma_{i+1}=\tilde{\Gamma})] =
\sum_{w_1=0}^{\infty}\sum_{w_3=0}^{\infty} \sum_{\gamma \in \Gamma} Q_{1, i}(\Gamma^{(k, r)}, w_1, w_3) \times \\
\times
E[v_1^{\varkappa_{1, i+1}}v_3^{\varkappa_{1, i+1}}I(\Gamma_{i+1}=\tilde{\Gamma}) | \varkappa_{1, i}=w_1, \varkappa_{3, i}=w_3,  \Gamma_i=\gamma] = \\ =
\sum_{w_1=0}^{\infty}\sum_{w_3=0}^{\infty} \sum_{\gamma \in {\mathbb H}_{-1}(\tilde{\gamma}, x_3)} Q_{1, i}(\Gamma^{(k, r)}, w_1, w_3) 
\times \\ \times E[v_1^{\max{\{0,  w_1 + \eta_{1, i} - \ell(\tilde{k}, \tilde{r}, 1)\}}} v_3^{\max{\{0,  w_3 + \eta_{3, i} - \ell(\tilde{k}, \tilde{r}, 3)\}}} | \varkappa_{1, i}=w_1, \varkappa_{3, i}=w_3,  \Gamma_i=\gamma] = \\ =
\sum_{w_1=0}^{\infty}\sum_{w_3=0}^{\infty} \sum_{\gamma \in {\mathbb H}_{-1}(\tilde{\gamma}, x_3)} Q_{1, i}(\Gamma^{(k, r)}, w_1, w_3) 
\times \\ \times E[v_1^{\max{\{0,  w_1 + \eta_{1, i} - \ell(\tilde{k}, \tilde{r}, 1)\}}} | \varkappa_{1, i}=w_1, \varkappa_{3, i}=w_3,  \Gamma_i=\gamma] \times \\ \times E[ v_3^{\max{\{0,  w_3 + \eta_{3, i} - \ell(\tilde{k}, \tilde{r}, 3)\}}} | \varkappa_{1, i}=w_1, \varkappa_{3, i}=w_3,  \Gamma_i=\gamma].
\label{second:try:gen}
\end{multline}
В случае $\tilde{\gamma}\not\in \Gamma^{\mathrm{I}}$ очередь $O_1$ не обслуживается и,  следовательно,  $\ell(\tilde{k}, \tilde{r}, 1)=0$. Поэтому
$$
\max{\{0,  w_1 + \eta_{1, i} - \ell(\tilde{k}, \tilde{r}, 1)\}} = w_1 + \eta_{1, i} - \ell(\tilde{k}, \tilde{r}, 1).
$$
Аналогично при $\tilde{\gamma}\not\in \Gamma^{\mathrm{III}}$ очередь $O_3$ не обслуживается и $\ell(\tilde{k}, \tilde{r}, 3)=0$. Откуда получаем 
$$
\max{\{0,  w_3 + \eta_{3, i} - \ell(\tilde{k}, \tilde{r}, 3)\}} = w_3 + \eta_{3, i} - \ell(\tilde{k}, \tilde{r}, 3).
$$

Рассмотрим случай $\tilde{\gamma}\in \Gamma^{\mathrm{I}}$. Распишем 
\begin{multline}
    E[v_1^{\max{\{0,  w_1 + \eta_{1, i} - \ell(\tilde{k}, \tilde{r}, 1)\}}} | \varkappa_{1, i}=w_1, \varkappa_{3, i}=w_3,  \Gamma_i=\gamma] = \\ =
    E[v_1^{ w_1 + \eta_{1, i} - \ell(\tilde{k}, \tilde{r}, 1)} | \varkappa_{1, i}=w_1, \varkappa_{3, i}=w_3,  \Gamma_i=\gamma] + \\ +
     E[v_1^{\max{\{0,  w_1 + \eta_{1, i} - \ell(\tilde{k}, \tilde{r}, 1)\}}} - v_1^{ w_1 + \eta_{1, i} - \ell(\tilde{k}, \tilde{r}, 1)} | \varkappa_{1, i}=w_1, \varkappa_{3, i}=w_3,  \Gamma_i=\gamma] = \\ =
      v_1^{w_1} q^{(1)}(\tilde{k}, \tilde{r}, v_1) +
     \sum_{a=0}^{\ell(\tilde{k}, \tilde{r}, 1) - w_1} \varphi_1(a, T^{(\tilde{k}, \tilde{r})})(1-v_1^{w_1+a-\ell(\tilde{k}, \tilde{r}, 1)}).
\label{second:try:first}
\end{multline}
Аналогично получим для  $\tilde{\gamma}\in \Gamma^{\mathrm{III}}$:
\begin{multline}
    E[v_3^{\max{\{0,  w_3 + \eta_{3, i} - \ell(\tilde{k}, \tilde{r}, 3)\}}} | \varkappa_{1, i}=w_1, \varkappa_{3, i}=w_3,  \Gamma_i=\gamma] = \\ =
     v_3^{w_1} q^{(3)}(\tilde{k}, \tilde{r}, v_3) +
     \sum_{a=0}^{\ell(\tilde{k}, \tilde{r}, 3) - w_3} \varphi_3(a, T^{(\tilde{k}, \tilde{r})})(1-v_3^{w_3+a-\ell(\tilde{k}, \tilde{r}, 3)}).
\label{second:try:second}
\end{multline}
Подставляя полученные выражения \eqref{second:try:first},  \eqref{second:try:second} в выражение  \eqref{second:try:gen},  получаем утверждение леммы.
\end{proof}

Из этой леммы следует существование величин $\mathfrak{M}^{(1, i)}(k,  r, v_1, v_3)$ хотя бы в некоторой окресности точки $(v_1, v_3)=(1, 1)$,  для $i>0$,  $k=\overline{0;d}$,  $r=\overline{1;n_{k}}$.

Ранее в работе уже доказана ограниченность  производящих функций
$$\mathfrak{M}^{(1,i)}(k,r,1,v_3)~\hm= ~E[v_3^{\varkappa_{3,i}}{I(\Gamma_{i}=\Gamma^{(k,r)})}]
$$ по $i\geqslant 0$ для всех $v_3\in [1,1+\varepsilon_3]$, при некотором $0 < \varepsilon_3 < \varepsilon$. Цель следующей леммы~--- доказать аналогичный результат для величин $\mathfrak{M}^{(1,i)}(k,r,v_1,1)$, $i\geqslant 0$.

\begin{lemma}
Если 
$$
\min_{k=\overline{0,d}} { \frac{\sum_{r = 1}^{n_k} \ell(k,r,1) }{\lambda_1 f_1'(1) \sum_{r=1}^{n_k} T^{(k,r)} }}>1,
$$
то числовая последовательность $\{\mathfrak{M}^{(1,i)}(k,r,v,1); i\geqslant 0\}$ ограничена при $v \hm\in [1, 1+\varepsilon_1]$, для некоторого $ 0 < \varepsilon_1 < \varepsilon$, где $\varepsilon>0$ определено в \eqref{GeneratingFunc}.
\label{generating:1:limited}
\end{lemma}
\begin{proof}
 Введем случайные последовательности $\{\varkappa_{1}^{(1)}(i); i\geqslant0\}$ и $\{\varkappa_{1}^{(2)}(i); i \hm\geqslant 0\}$ следующим образом. Положим для $i=0$: $\varkappa_{1}^{(1)}(0)=0$ и $\varkappa_{1}^{(2)}(0)=\varkappa_{1,0}$. Далее введем рекуррентные соотношения:  
\begin{equation*}
  \varkappa_{1}^{(1)}(i+1) =
  \begin{cases}
    \max{\{0,\varkappa_{1}^{(1)}(i) + \eta_{1,i} - \xi_{1,i}\}},&  \text{если } \Gamma_{i+1}=\Gamma^{(k,r)}, k>0, r=\overline{1,n_k}; \\
    \varkappa_{1}^{(1)}(i),&  \text{если } \Gamma_{i+1}=\Gamma^{(0,r)}, r=\overline{1,n_0};
  \end{cases}
\end{equation*}
\begin{equation*}
  \varkappa_{1}^{(2)}(i+1) =
  \begin{cases}
    \varkappa_{1}^{(2)}(i),&  \text{если } \Gamma_{i+1}=\Gamma^{(k,r)}, k>0, r=\overline{1,n_k};\\
    \max{\{0,\varkappa_{1}^{(2)}(i) + \eta_{1,i} - \xi_{1,i}\}},&  \text{если } \Gamma_{i+1}=\Gamma^{(0,r)}, r=\overline{1,n_0}.
  \end{cases}
\end{equation*}
Тогда последовательность $\varkappa_{1,i}^+=\varkappa_{1}^{(1)}(i) + \varkappa_{1}^{(2)}(i)$ является мажорирующей для последовательности $\varkappa_{1,i}$, т.е. $\varkappa_{1,i}(\omega) \leqslant \varkappa_{1,i}^+(\omega)$, $\forall \omega \in \Omega$. Доказательство этого факта несложно и проводится по индукции. Заметим, что из него следует для $v\geqslant 1$ неравенство
\begin{equation}
E[v^{\varkappa_{1,i}}]\leqslant E[v^{\varkappa_{1}^{(1)}(i)} v^{\varkappa_{1}^{(2)}(i)}].
\label{cenzor:estimate}
\end{equation}
                               

Наблюдение за вновь введенными величинами $\varkappa_{1}^{(1)}(i)$ и $\varkappa_{1}^{(2)}(i)$ будем осуществлять в случайные моменты времени $\theta_{i}^{(1)}$ и $\theta_{i}^{(2)}$ соответственно определяемые следующими соотношениями:
\begin{equation}
\begin{aligned}
  \theta_{0}^{(1)}=0; & \quad \theta_{i+1}^{(1)}=\theta_{i}^{(1)} + \min{\{s>0\colon \Gamma_{\theta_{i}^{(1)}+s} = \Gamma^{(k,n_k)}, k>0\}};\\
  \theta_{0}^{(2)}=0; & \quad  \theta_{i+1}^{(2)}=\theta_{i}^{(2)} + \min{\{s>0\colon \Gamma_{\theta_{i}^{(2)}+s} = \Gamma^{(0,r)}, r=\overline{1,n_0}\}}.
  \label{stop:times}
\end{aligned}
\end{equation}
Также нам понадобятся следующие обозначения:
\begin{equation}
  \hat{\varkappa}_{1,i}^{(1)}=\varkappa_{1}^{(1)}(\theta_{i}^{(1)}), \quad   \hat{\varkappa}_{1,i}^{(2)}=\varkappa_{1}^{(2)}(\theta_{i}^{(2)}).
  \label{stop:queue}
\end{equation}
Пусть $k>0$, $r \in \{2, 3, \ldots, n_k\}$. В веденных обозначениях рассмотрим выражение для $E[v^{\varkappa_{1}^{(1)}(i+1)} I(\Gamma_{i+1}=\Gamma^{(k,r)})]$:
\begin{multline*}
  E[v^{\varkappa_{1}^{(1)}(i+1)} I(\Gamma_{i+1}=\Gamma^{(k,r)})] = \\ =\sum_{w_1 \geqslant 0} \sum_{w_3 \geqslant 0} \sum_{\gamma \in \Gamma} E[v^{\varkappa_{1}^{(1)}(i+1)} I(\Gamma_{i+1}=\Gamma^{(k,r)}, \varkappa_{1}^{(1)}(i)=w_1, \varkappa_{3,i}=w_3,\Gamma_i=\gamma)] =\\= \sum_{w_1\geqslant 0} \sum_{w_3\geqslant 0} E[v^{w_1 + \eta_{1,i}-\ell(k,r,1)} I(\varkappa_{1}^{(1)}(i)=w_1, \varkappa_{3,i}=w_3,\Gamma_i=\Gamma^{(k,r-1)})] + \widetilde{C}_1=\\
  =\sum_{w_1\geqslant 0} v^{w_1 } \Pr (\varkappa_{1}^{(1)}(i)=w_1, \Gamma_i=\Gamma^{(k,r-1)}) q^{(1)}(k,r,v) + \widetilde{C}_1 = \\
  =q^{(1)}(k,r,v) E[v^{\varkappa_{1}^{(1)}(i)} I(\Gamma_i=\Gamma^{(k,r-1)})] + \widetilde{C}_1.
\end{multline*}
И далее по индукции:
\begin{equation*}
  E[v^{\varkappa_{1}^{(1)}(i+n_k-1)} I(\Gamma_{i+n_k-1}=\Gamma^{(k,n_k)})] = \prod_{r=2}^{n_k} q^{(1)}(k,r,v) E[v^{\varkappa_{1}^{(1)}(i)} I(\Gamma_{i}=\Gamma^{(k,1)})].
\end{equation*}
Для $w_1, w_3 \in Z_+$, $\gamma, \gamma_1, \gamma_2 \in \Gamma$, $C \subset [0, +\infty)$ введем множества 
\begin{align*}
  A_i^{(1)}(w_1,w_3,\gamma) &= \{\omega\colon \varkappa_{1}^{(1)}(\theta_{i}^{(1)})=w_1; \varkappa_{3,\theta_{i}^{(1)}}=w_3, \Gamma_{\theta_{i}^{(1)}}=\gamma\};\\
  A_i^{(1)}(w_1,C,\gamma) &= \bigcup_{w_3 \in C} A_i^{(1)}(w_1,w_3,\gamma);\quad B_i^{(1)}(\gamma) =\{\omega\colon \Gamma_{\theta_{i}^{(1)}}=\gamma\};\\
  C_i^{(1)}(\gamma_1,w_1,w_3,\gamma_2)&= B_{i+1}^{(1)}(\gamma_1) \cap A_i^{(1)}(w_1,w_3,\gamma_2) .
\end{align*}
Пусть $\tilde{k}>0$. Тогда
\begin{multline*}
  E[v^{\hat{\varkappa}_{1,i+1}^{(1)}} I(\Gamma_{\theta_{i+1}^{(1)}}= \Gamma^{(\tilde{k},n_{\tilde{k}})})] =\\=E[v^{\hat{\varkappa}_{1,i+1}^{(1)}} I(B_{i+1}^{(1)}(\Gamma^{(\tilde{k},n_{\tilde{k}})}))] 
  =E[v^{\varkappa_{1}^{(1)}(\theta_{i+1}^{(1)})} I(B_{i+1}^{(1)}(\Gamma^{(\tilde{k},n_{\tilde{k}})}))] =\\
  =\prod_{\tilde{r}=2}^{n_{\tilde{k}}} q^{(1)}(\tilde{k},\tilde{r},v) E[v^{\varkappa_{1}^{(1)}(\tau)} I(B_{i+1}^{(1)}(\Gamma^{(\tilde{k},n_{\tilde{k}})}))] + \widetilde{C}_1=\\
  %=\prod_{\tilde{r}=2}^{n_{\tilde{k}}} q^{(1)}(\tilde{k},\tilde{r},v)    \sum_{\substack{w_1\geqslant 0, \\ w_3 \geqslant 0}}  \sum_{\gamma \in \Gamma} E[v^{\varkappa_{1}^{(1)}(\tau)}  I( C_i^{(1)}(\Gamma^{(\tilde{k},n_{\tilde{k}})},w_1,w_3,\gamma ))] + \widetilde{C}_1= \\
  =\prod_{\tilde{r}=2}^{n_{\tilde{k}}} q^{(1)}(\tilde{k},\tilde{r},v) \sum_{\substack{w_1\geqslant 0, \\ w_3 \leqslant L}} \sum_{k=1}^d E[v^{\varkappa_{1}^{(1)}(\tau)} I( C_i^{(1)}(\Gamma^{(\tilde{k},n_{\tilde{k}})},w_1,w_3,\Gamma^{(k,n_k)} ))] + \\ +
    \prod_{\tilde{r}=2}^{n_{\tilde{k}}} q^{(1)}(\tilde{k},\tilde{r},v) \sum_{\substack{w_1\geqslant 0, \\ w_3 > L}}  E[v^{\varkappa_{1}^{(1)}(\tau)} I( C_i^{(1)}(\Gamma^{(\tilde{k},n_{\tilde{k}})},w_1,w_3,\Gamma^{(\tilde{k},n_{\tilde{k}})} ))] +\widetilde{C}_1,
\end{multline*}
где мы обозначили для краткости $\tau = \theta_{i+1}^{(1)} -n_{\tilde{k}} + 1$.
Далее поскольку 
$$\varkappa_{1}^{(1)}(\tau) = \max{\{0; \varkappa_{1 }^{(1)}(\theta_{i}^{(1)}) +\eta_{1,\tau}-\ell(1,\tilde{k},1) \}},
$$
то продолжим цепочку рассуждений:
\begin{multline}
E[v^{\hat{\varkappa}_{1,i+1}^{(1)}} I(\Gamma_{\theta_{i+1}^{(1)}}= \Gamma^{(\tilde{k},n_{\tilde{k}})})]= \prod_{\tilde{r}=2}^{n_{\tilde{k}}} q^{(1)}(\tilde{k},\tilde{r},v) \sum_{\substack{w_1\geqslant 0, \\ w_3 \leqslant L}} \sum_{k=1}^d v^{w_1} v^{- \ell(1,\tilde{k},1)} \times \\ \times E[v^{ \eta_{1,\tau} } I( C_i^{(1)}(\Gamma^{(\tilde{k},n_{\tilde{k}})},w_1,w_3,\Gamma^{(k,n_k)} ))] +
\prod_{\tilde{r}=2}^{n_{\tilde{k}}} q^{(1)}(\tilde{k},\tilde{r},v)\times \\ \times \sum_{\substack{w_1\geqslant 0, \\ w_3 > L}} v^{w_1}v^{- \ell(1,\tilde{k},1)}  E[v^{ \eta_{1,\tau}} I( C_i^{(1)}(\Gamma^{(\tilde{k},n_{\tilde{k}})},w_1,w_3,\Gamma^{(\tilde{k},n_{\tilde{k}})} ))] + \widetilde{C}_2 = \\
  = Q_1(v,\tilde{k})  \biggl(\sum_{w_1\geqslant 0} \sum_{k=1}^d v^{w_1} \Pr(C_i^{(1)}(\Gamma^{(\tilde{k},n_{\tilde{k}})},w_1,[0;L],\Gamma^{(k,n_k)} )) + \\
   + \sum_{w_1\geqslant 0}  v^{w_1} \Pr(C_i^{(1)}(\Gamma^{(\tilde{k},n_{\tilde{k}})},w_1,(L;\infty),\Gamma^{(\tilde{k},n_{\tilde{k}})} ))\biggr) + \widetilde{C}_2,
   \label{censor:before:sum}
  \end{multline}
где $Q_1(v,\tilde{k})= \prod_{\tilde{r}=1}^{n_{\tilde{k}}} q^{(1)}(\tilde{k},\tilde{r},v) $. Просуммируем по $\tilde{k}$ получившийся в \eqref{censor:before:sum} результат:
\allowdisplaybreaks
\begin{multline*}
\sum_{\tilde{k}=1}^d E[v^{\hat{\varkappa}_{1,i+1}^{(1)}} I(\Gamma_{\theta_{i+1}^{(1)}}= \Gamma^{(\tilde{k},n_{\tilde{k}})})]=   E[v^{\hat{\varkappa}_{1,i+1}^{(1)}} ]=\\
  = \sum_{\tilde{k}=1}^d Q_1(v,\tilde{k})  \biggl(\sum_{w_1\geqslant 0} \sum_{k=1}^d v^{w_1} \Pr(C_i^{(1)}(\Gamma^{(\tilde{k},n_{\tilde{k}})},w_1,[0;L],\Gamma^{(k,n_k)} )) + \\
   + \sum_{w_1\geqslant 0}  v^{w_1} \Pr(C_i^{(1)}(\Gamma^{(\tilde{k},n_{\tilde{k}})},w_1,(L;\infty),\Gamma^{(\tilde{k},n_{\tilde{k}})} ))\biggr) + \widetilde{C}_3 \leqslant \\ \leqslant
   \max_{ \tilde{k}=\overline{1;d} }{\{Q_1(v,\tilde{k}) \}} \sum_{w_1\geqslant 0}  v^{w_1} \biggl(\sum_{\tilde{k}=1}^d \sum_{k=1}^d  \Pr(C_i^{(1)}(\Gamma^{(\tilde{k},n_{\tilde{k}})},w_1,[0;L],\Gamma^{(k,n_k)} )) + \\
   +  \sum_{\tilde{k}=1}^d \Pr(C_i^{(1)}(\Gamma^{(\tilde{k},n_{\tilde{k}})},w_1,(L;\infty),\Gamma^{(\tilde{k},n_{\tilde{k}})} ))\biggr) + \widetilde{C}_3 = \\
   =\max_{ \tilde{k}=\overline{1;d} }{\{Q_1(v,\tilde{k}) \}} \sum_{w_1\geqslant 0}  v^{w_1} \Pr(\hat{\varkappa}_{1,i}^{(1)}=w_1)+ \widetilde{C}_3=\max_{ \tilde{k}=\overline{1;d} }{\{Q_1(v,\tilde{k}) \}} E[v^{\hat{\varkappa}_{1,i}^{(1)}} ]+ \widetilde{C}_3.
  \end{multline*}

Для $\hat{\varkappa}_{1,i}^{(2)}$ можно провести похожие рассуждения и в итоге получить оценки:
\begin{align}
 E[v^{\hat{\varkappa}_{1,i+1}^{(1)}} ] &\leqslant \max_{ \tilde{k}=\overline{1;d} }{\{Q_1(v,\tilde{k}) \}} E[v^{\hat{\varkappa}_{1,i}^{(1)}} ]+ \widetilde{C}_3;\\
  E[v^{\hat{\varkappa}_{1,i+n_0}^{(2)}}] &\leqslant Q_1(v,0) E[v^{\hat{\varkappa}_{1,i}^{(2)}} ] + \widetilde{C}_4,
  \end{align}
  где $r=\overline{1,n_0}$. Для $k=\overline{0;d}$ верны равенства $Q_1(1,k)=1$. Предположив выполненным условие $\min_{k=\overline{0,d}} { \frac{\sum_{r = 1}^{n_k} \ell(k,r,1) }{\lambda_1 f_1'(1) \sum_{r=1}^{n_k} T^{(k,r)} }}>1$, получим, что величины
\begin{multline}
 \left.\left(Q_1(v,k)\right) ' \right|_{v_1=1} = 
  \left.\left(\prod_{r=1}^{n_k}v^{-\ell(k,r,1)}\sum_{w=0}^{\infty} \varphi_1(w,T^{(k,r)})v^w \right) ' \right|_{v_1=1} = \\ =
   \left.\left(\prod_{r=1}^{n_k} v^{-\ell(k,r,1)}\exp(\lambda_1 T^{(k,r)} (f_1(v)-1))\right) ' \right|_{v_1=1} = \\ =
    \left.\left(v^{-\sum_{r=1}^{n_k}\ell(k,r,1)}\exp(\lambda_1 (f_1(v)-1)\sum_{r=1}^{n_k} T^{(k,r)}) \right) ' \right|_{v_1=1} = \\ =
\lambda_1 f_1'(1) \sum_{r=1}^{n_k} T^{(k,r)} -\sum_{r=1}^{n_k} \ell(k,r,1), 
\label{derivative:cycle:1}
\end{multline}
определяющие знак производной, отрицательны. Поэтому $|Q_1(v,k)|<1$ для всех $k=\overline{1;d}$  хотя бы в некоторой правой окрестности $1 \leqslant v \leqslant (1+ \varepsilon_1)^{1/2}$ точки $v=1$. Этот факт, в свою очередь, обеспечивает ограниченность в этой же окрестности величин $E[v^{\hat{\varkappa}_{1,i}^{(1)}} ]$ и $E[v^{\hat{\varkappa}_{1,i}^{(2)}}] $ равномерно по $i$.


Далее из определений \eqref{stop:times} и \eqref{stop:queue} следует, что для любого $i\geqslant 0$ существуют такие $j_1$ и $j_2$, что
\begin{equation*}
\varkappa_{1}^{(1)}(i) \leqslant \hat{\varkappa}_{1,j_1}^{(1)}, \quad
\varkappa_{1}^{(2)}(i) \leqslant \hat{\varkappa}_{1,j_2}^{(2)}.
\end{equation*}
Следовательно, из \eqref{cenzor:estimate} и неравенства Коши-Буняковского заключаем, что
\begin{equation*}
\mathfrak{M}^{(1,i)}(k,r,v,1) \leqslant \bigl( E[v^{2 \varkappa_{1}^{(1)}(i)}] E[v^{2\varkappa_{1}^{(2)}(i)}]\bigr)^{1/2}
 \leqslant \bigl( E[v^{2 \hat{\varkappa}_{1,j_1}^{(1)}}] E[v^{2\hat{\varkappa}_{1,j_2}^{(2)}}]\bigr)^{1/2}
\end{equation*}
и, значит, для любого $v$ хотя бы из окрестности $[1, 1+\varepsilon_1]$ исходная последовательность $\{\mathfrak{M}^{(1,i)}(k,r,v,1); i~\geqslant~0\}$  ограничена равномерно по $i$.

\end{proof}




Важным результатом работы является достаточное условие существования стационарного режима последовательности $\{(\Gamma_i,  \varkappa_{1, i}, \varkappa_{3, i}); i \geqslant 0\}$.
\begin{theorem}
Для того,  чтобы марковская цепь $\{(\Gamma_i,  \varkappa_{1, i}, \varkappa_{3, i}); i \geqslant 0\}$ имела стационарное распределение $Q_1(\gamma, x_1, x_3)$,  $(\gamma, x_1, x_3)\in \Gamma \times {\mathbb Z}^2_+$,  достаточно выполнения неравенств
\begin{equation}
\min_{k=\overline{0, d}} { \frac{\sum_{r = 1}^{n_k} \ell(k, r, 1) }{\lambda_1 f_1'(1) \sum_{r=1}^{n_k} T^{(k, r)} }}>1,  \quad 
\min_{k=\overline{1, d}} { \frac{\sum_{r = 1}^{n_k} \ell(k, r, 3) }{\lambda_3 f_3'(1) \sum_{r=1}^{n_k} T^{(k, r)} }}>1.
\label{sufficient:double}
\end{equation}
\label{sufficient:double:theorem}
\end{theorem}

\begin{proof}
Предположим обратное,  а именно,  что при выполнении условия \eqref{sufficient:double} марковская цепь $\{(\Gamma_i,  \varkappa_{1, i}, \varkappa_{3, i}); i \geqslant 0\}$ не имеет стационарного распределения. 
Тогда для любого состояния $(\gamma, x_1, x_3)\in \Gamma \times {\mathbb Z}^2_+$ и независимо от начального распределения 
$$
\Pr(\Gamma_{0}=\Gamma^{(k, r)},  \varkappa_{1, 0}=x_1,  \varkappa_{3, 0}=x_3), \quad (\Gamma^{(k, r)}, x_1, x_3)\in \Gamma \times {\mathbb Z}^2_+,
$$
имеют место предельные равенства 
\begin{equation}
\lim_{i \to \infty} \Pr(\Gamma_{i}=\Gamma^{(k, r)},  \varkappa_{1, i}=x_1,  \varkappa_{3, i}=x_3) =0,  \quad  (\Gamma^{(k, r)}, x_1, x_3)\in \Gamma \times {\mathbb Z}^2_+.
\label{zero:limit:equations:1}
\end{equation} 
Для доказательства этого факта достаточно рассмотреть все возможные случаи,  предполагая апериодичность рассматриваемой цепи (см. рассуждения \cite[гл. $3$,   \S~3-4]{Shiryaev}):
\begin{enumerate}
\item все состояния цепи $\{(\Gamma_i,  \varkappa_{1, i}, \varkappa_{3, i}); i \geqslant 0\}$ невозвратные,  тогда предельные соотношения выполняются в силу \cite[с. 541,  лемма $2$]{Shiryaev};
\item существует хотя бы одно возвратное состояние,  тогда все состояния возвратные (поскольку все состояния сообщающиеся); и пусть все состояния нулевые,  тогда предельное соотношение также выполняется \cite[с. 541,  лемма $3$]{Shiryaev};
\item все состояния возвратные и существует хотя бы одно положительное,  тогда все состояния положительные и пределы 
$$
\lim_{i \to \infty} \Pr(\Gamma_{i}\hm= \Gamma^{(k, r)},  \varkappa_{1, i}\hm= x_1,  \varkappa_{3, i}\hm= x_3) > 0
$$ являются стационарными вероятностями (см.~рассуждения~{\cite[с. 549,  теорема $1$]{Shiryaev}}),  что противоречит предположению.
\end{enumerate}
Для периодической цепи приведенные рассуждения достаточно провести для циклических подклассов.

Выберем начальное распределение так,  что при некоторых $v_1 >1$,  $v_3 >1$ будут выполнены неравенства $\mathfrak{M}^{(1, 0)}(k, r, v_1, 1) <\infty$,  $\mathfrak{M}^{(1, 0)}(k, r, 1, v_3) <\infty$ для всех $\Gamma^{(k, r)}\in \Gamma$. Это ограничение,  в силу леммы \eqref{generating:1:limited} и полученных ранее результатов касательно марковской цепи ${\MarkThree}$,  обеспечивает при любом конечном $i\geqslant 0$ существование функций 
\begin{equation}
\mathfrak{M}^{(1, i)}(k, r, v_1, 1),  \quad \mathfrak{M}^{(1, i)}(k, r, 1, v_3),
\end{equation}
\begin{equation}
\frac{d}{dv_1} \left[\mathfrak{M}^{(1, i)}(k, r, v_1, 1)\right],  \quad \frac{d}{dv_3} \left[\mathfrak{M}^{(1, i)}(k, r, 1, v_3)\right].
\end{equation}
по крайней мере в некоторой окрестности точек $v_1 = 1$,  $v_3=1$.

В силу равенств \eqref{zero:limit:equations:1} для любого натурального $N$ найдется некоторое число $\mathfrak{I}$,  что для всех $i > \mathfrak{I}$ будет выполнено условие
$$1 > (1+N) \sum_{x_1=0}^{N}\sum_{x_3=0}^{N} \sum_{\Gamma^{(k, r)}\in \Gamma}  \Pr(\Gamma_{i}=\Gamma^{(k, r)},  \varkappa_{1, i} \hm{} = x_1,  \varkappa_{3, i}=x_3)$$
и,  значит,  
$$1 >(1+N) \sum_{x_1=0}^{N}\sum_{x_3=0}^{N} \Pr(\varkappa_{1, i} = x_1,  \varkappa_{3, i}=x_3)
.$$
Тогда
\begin{multline*}
E[\varkappa_{3, i}+\varkappa_{1, i}] = \sum_{x_1=0}^{\infty}\sum_{x_3=0}^{\infty} (x_1 + x_3) \Pr(\varkappa_{1, i} = x_1,  \varkappa_{3, i}=x_3) \geqslant \\
\geqslant
 \sum_{x_1=0}^{\infty}\sum_{x_3=N+1}^{\infty} x_3 \Pr(\varkappa_{1, i} = x_1,  \varkappa_{3, i}=x_3) +  \sum_{x_1=N+1}^{\infty}\sum_{x_3=0}^{\infty} x_1 \Pr(\varkappa_{1, i} = x_1,  \varkappa_{3, i}=x_3) \geqslant \\
 \geqslant
  \sum_{x_1=0}^{\infty}\sum_{x_3=N+1}^{\infty} (N+1) \Pr(\varkappa_{1, i} = x_1,  \varkappa_{3, i}=x_3) + \sum_{x_1=N+1}^{\infty}\sum_{x_3=0}^{\infty} (N+1) \times \\ \times \Pr(\varkappa_{1, i} = x_1,  \varkappa_{3, i}=x_3)
  \geqslant
  (N+1)\Bigl[\sum_{x_1=0}^{\infty}\sum_{x_3=N+1}^{\infty} \Pr(\varkappa_{1, i} = x_1,  \varkappa_{3, i}=x_3) + \\+ \sum_{x_1=N+1}^{\infty}\sum_{x_3=0}^{\infty}\Pr(\varkappa_{1, i} = x_1,  \varkappa_{3, i}=x_3)\Bigr] 
  \geqslant (N+1)(1-\Pr(\varkappa_{1, i} \leqslant N,  \varkappa_{3, i} \leqslant N)  ) 
  \geqslant \\ \geqslant
  (N+1)\biggl(1-\frac{1}{N+1}\biggr) = N.
\end{multline*}
Следовательно,  $E[\varkappa_{3, i}+\varkappa_{1, i}]$ неограниченно возрастает при $i \to \infty$. 

Другое рассуждение,  однако,  приводит к противоположному результату. Действительно,  последовательность $\{E[\varkappa_{3, i}]; i \geqslant 0\}$ ограничена  и нетрудно проверить,  что
$$
E[\varkappa_{1, i}] =\sum_{\Gamma^{(k, r)}\in \Gamma} \frac{d}{dv}\left.\left(\mathfrak{M}^{(1, i)}(k, r, v_1, 1)\right)\,\,\right|_{v_1=1}, 
$$
где величина справа в силу интегральной формулы Коши и леммы \ref{generating:1:limited} равномерно по $i$ ограничена некоторой постоянной величиной. 
 Поэтому принятое предположение не будет справедливым. Доказательство этим завершается.
\end{proof}






















































