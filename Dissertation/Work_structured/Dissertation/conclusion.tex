\chapter*{Заключение}						% Заголовок
\addcontentsline{toc}{chapter}{Заключение}	% Добавляем его в оглавление
В приведенной работе был рассмотрен тандем управляющих систем, управление в которых осуществляется по циклическому алгоритму и алгоритму с продлением. Основные результаты работы заключаются в следующем.

    \begin{enumerate}
        \item Построена строгая математическая модель тандема с циклическим алгоритмом управления и алгоритмом с продлением. Отличительной особенностью системы также является немгновенность перемещения требований между системами. 
        \item Доказана марковость случайной последовательности, включающей длину низкоприоритетной очереди. Проведена классификация состояний цепи по арифметическим свойствам переходных вероятностей этой последовательности. А также найдены достаточное и необходимое условия существования стационарного распределения.
        \item Проведен аналогичный анализ для случайной последовательности, включающей очереди первичных требований: доказана ее марковость, проведена классификация состояний и найдено достаточное условие существования стационарного распределения.
        \item Найдено условие ограниченности для последовательности математических ожиданий $    \{( E\varkappa_{4,i}); i \geqslant 0\}$.
%        \item Разработана имитационная модель для изучения исходной системы и написана программа ее реализующая.
        \item На основе имитационной модели расширены результаты, полученные теоретически.
    \end{enumerate}
Для дальнейшего исследования можно выделить следующие направления и задачи.
\begin{enumerate}
    \item Найти необходимые и достаточные условия существования стационарного распределения для марковской цепи, включающей в себя состояние обслуживающего устройства и четыре очереди и описывающей всю систему целиком.
    \item Провести более глубокое изучение вида кривой равных загрузок для определения оптимальных значений параметров системы с точки зрения средневзвешенного времени пребывания требования.
    \item Расширить класс рассматриваемых графов переходов для обслуживающего устройства.
    \item Рассмотреть систему с б\'{о}льшим числом входных потоков, что соответствует, например, нескольким подряд идущим перекресткам. В идеале нужно придти к модели, содержащей произвольное число таких перекрестков.
\end{enumerate}


