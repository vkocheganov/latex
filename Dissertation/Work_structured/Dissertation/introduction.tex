\chapter*{Введение}							% Заголовок
\addcontentsline{toc}{chapter}{Введение}	% Добавляем его в оглавление

\newcommand{\actuality}{{\textbf\actualityTXT}}
\newcommand{\progress}{}
\newcommand{\aim}{{\textbf\aimTXT}}
\newcommand{\tasks}{\textbf{\tasksTXT}}
\newcommand{\novelty}{\textbf{\noveltyTXT}}
\newcommand{\influence}{\textbf{\influenceTXT}}
\newcommand{\methods}{\textbf{\methodsTXT}}
\newcommand{\defpositions}{\textbf{\defpositionsTXT}}
\newcommand{\reliability}{\textbf{\reliabilityTXT}}
\newcommand{\probation}{\textbf{\probationTXT}}
\newcommand{\contribution}{\textbf{\contributionTXT}}
\newcommand{\publications}{\textbf{\publicationsTXT}}

{\actuality}
%С самых истоков становления теории массового обслуживания перед исследователями, среди прочего, стояла оптимизационная задача. Так, например, в упомянутой выше работе  А.К.~Эрланга ставилась задача поиска оптимального (минимального) числа телефонных линий для удовлетворительного обслуживания абонентов. В терминах дисциплины исследования операций действие по обслуживанию требований является <<операцией>> над некоторыми абстрактными объектами (абонентами, заявками, требованиями), эффективность осуществления которой является одним из предметов исследования. При этом операция по обслуживанию зачастую выполняется под действием как детерминированных, так и случайных факторов. В качестве критериев эффективности операции обслуживания обычно выступают вероятность простоя обслуживающих устройств, среднее время пребывания требования в системе, вероятность отказа устройства, средняя длина очереди, среднее количество занятых устройств и т.д. С учетом названных критериев формируется конечный функционал, оптимизация которого и является главной целью. Примерами работ, в которых задача теории массового обслуживания ставится в терминах исследования операций, являются \cite{Bailey:1957, Flagle:1962, Buslenko:1968}.
%Taha, Carter, Murthy
%С точки зрения фундаментального образования, теория массового обслуживания также является неотъемлимой и устоявшейся ветвью исследования операций. Классические учебники (см. \cite{Taha, Carter, Murthy, Hillier} и др.) по исследованию операций включают в себя разделы про марковские цепи, случайные процессы и теорию массового обслуживания (теорию очередей). Наряду с базовыми понятиями линейного программирования и теории игр в учебниках вводится марковская цепь, определяются основные классы ее состояний и рассматриваются вопросы эргодичности. Обобщения марковской цепи строятся посредством случайных  процессов: марковских или процессов рождения и гибели. Далее этот материал обычно применяется в разделе теории массового обслуживания. Инструменты для аппробации построенных вероятностных моделей, как правило, предоставляются в учебниках по исследованию операций в разделе с названием <<имитационное моделирование>> (<<simulation>>).
С каждым годом количество задач, в которых требуется применение математических методов для выбора некоторого оптимального решения, постоянно растет. Создание автоматизированных систем практически невозможно без предварительного исследования управляемого процесса методами математического моделирования. Подобного рода проблемы попадают в сферу ответственности такой науки как исследование операций. Однако предмет изучения данной дисциплины весьма широк и в разных источниках трактуется по-разному. Например, книга Е.С.~Вентцель~\cite{Ventcel} начинается со следующего объяснения термина <<исследование операций>>: <<Под этим термином мы будем понимать применение математических, количественных методов для обоснования решений во всех областях целенаправленной человеческой деятельности>>. Далее в книге~Е.С.~Вентцель раскрывается понятие <<решения>> как выбор действия для достижения конкретной цели с приминением того или другого математического аппарата. В качестве примера задачи исследования операций в этой книге приводится задача из теории очередей: библиотечное обслуживание. В фондах библиотеки имеется большое количество литературы разной тематики и для удовлетворения запросов абонентов нужно разработать оптимальную схему обслуживания.

В книге~\cite{Hillier} появление научной отрасли под названием  <<исследование операций>>  связывается с созданием больших корпораций, управление которыми породило качественно новые задачи для руководства. Отдельные части компаний становятся крупнее, обрастают своими целями, нетривиальным образом коррелируемыми с общей целью компании. С этим связана проблема выделения ресурсов для того или иного подразделения компании для достижения наиболее выгодного существования всей компании в целом. Так или иначе, на сегодняшний момент существует множество литературы, в которой дается последовательное изложение современного состояния исследования операций и, в том числе, ее важной подобласти, теории очередей (см., например, книги~\cite{Taha, Carter, Murthy, Hillier, Encylc, Ventcel, Eltarenko}).

Теория массового обслуживания (теория очередей) предоставляет математический аппарат для анализа систем, в которых имеется операция по обслуживанию (<<обработка>>) некоторых объектов при наличии случайных факторов. При этом тип обслуживания и самих объектов не имеет значения, что делает область применения этой дисциплины достаточно широкой: системы связи, автоматические линии производства, системы медицинского обслуживания, системы управления транспорта и т.д.  В роли объектов могут выступать, например, абоненты, заявки, требования.

Математический аппарат теории очередей эволюционировал на протяжении всего 20-го века. Так, в начале 1930-х годов разработанная математическая теория случайных процессов позволила строить модели в более компактном виде, нежели в виде большого числа дифференциальных и интегральных уравнений для вероятностей или плотностей распределений, как делали до этого. Потребность в управлении и оптимизации систем массового обслуживания побудила исследователей взглянуть на построение моделей с позиций исследования операций и теории управляющих систем.

% Теория вероятностей начала зарождаться в XVII веке благодаря азартным играм и случайности, заложенной в них. Первые математические модели таких игр были построены математиками П.~Ферма и Б.~Паскаль в  переписке друг с другом. В решении задач они приблизились к таким понятиями как вероятность, математическое ожидание, теорема сложения и умножения вероятностей, хотя явно  их  не вводили. Значительный толчок в развитии теории вероятностей осуществил Я.~Бернулли. Сформулировав и доказав закон больших чисел, он ввел в научный обиход понятие вероятности.

Первые исследования в области систем массового обслуживания были сделаны  А.К.~Эрлангом \cite{Erlang:1909,Erlang:1917}, Ф.В.~Йохансеном \cite{Johannsen} и Ф.~Поллачеком \cite{Pollaczek:1934}. Из физических соображений ими были составлены интегральные и дифференциальные уравнения для функции распределения числа занятых линий  и времени  ожидания на телефонной станции. Найденные формулы создали базу для будущего развития теории. Дальнейший вклад в развитие теории очередей внесли следующие зарубежные и отечественные ученые: К.~Пальм, Д.Дж.~Кендалл, Д.~Линдли, Л.~Такач, Д.Р.~Кокс, У.Л.~Смит, Т.Л.~Саати, Л.~Клейнрок,  Н.Т.Дж.~Бейли,  А.Н.~Колмогоров, А.Я.~Хинчин, Б.В.~Гнеденко, И.Н.~Коваленко,  С.Н.~Бернштейн, Н.П.~Бусленко, А.А.~Боровков, В.С.~Королюк, Г.П.~Башарин, Г.П.~Климов, Ю.В.~Прохоров, А.Д.~Соловьев,  Б.А.~Севастьянов и др. Краткое представление об истории вопроса можно составить, ознакомившись с монографиями, журнальными обзорами, например, \cite{Borovkov, Bocharov:1995, GnedenkoKovalenko, KoksSmith, Kovalenko:1963, asmussen, kalashnikov}. Технологический прогресс конца XX века привел к развитию таких направлений, как сети массового обслуживания и тандемы систем массового обслуживания \cite{ivnickii, jackson, reich}. Сегодня класс задач, к которым применимы методы теории очередей, становится еще шире: медицинское и банковское обслуживание, управление информационным и транспортным трафиками и т.п. (см., например, работы \cite{dudin:2011, farhadov, Haight:1963, haidemann, raghavan, rogiest}).
 
При изучении любой реальной системы математическими методами первостепенным является этап построения модели. 
%Так, Б.~Паскаль и П.~Ферма при решении первых вероятностных задач обходились без самого понятия <<вероятности>> и строили математическую модель в терминах количества <<благоприятствующих>> и <<неблагоприятствующих>> исходов (см.~\cite{Gnedenko}). Само понятие <<вероятности>> как числа, заключенного между $0$ и $1$, было введено Я.~Бернулли при формулировке и выводе закона больших чисел. В контексте управляемых систем массового обслуживания, первые модели А.К.~Эрланга и Ф.~Поллячека были построены на языке дифференциальных и интегральных уравнений для плотностей некоторых случайных величин и вероятностей изучаемых событий.
Однако на каком бы этапе развития ни находилась та или иная теория, для получения качественно новых результатов часто не обойтись без расширения существующего математического аппарата.  
При построении моделей в теории массового обслуживания таким расширением стали в 1930-х годах аксиоматизация теории вероятностей и появление теории случайных процессов (см. работу \cite{Kolmogorov:1974}).
Создание аксиоматизированного понятия случайного процесса сформировало общий подход к исследованию систем массового обслуживания, который принято называть классическим (см. работу~\cite{Fedotkin:1996}). Для задания математической модели системы, исходя из содержательного смысла задачи, задаются следующие обязательные элементы: входящий поток, закон формирования очереди, емкость очереди, количество приборов обслуживания и закон обслуживания произвольного требования. Под состоянием системы при таком подходе естественно понимать количество требований в очередях. Однако для получения глубоких результатов важно, чтобы изучаемый процесс поддавался аналитике и был, например, марковским. В книге \cite{GnedenkoKovalenko} представлены основные приемы выделения из заданной модели процессов с марковским свойством. Так или иначе, исходя из содержательной постановки задачи, исследователем явно выписываются уравнения для распределений вероятностей интересующего случайного (марковского) процесса.

На основе классического подхода был произведен анализ многих видов систем массового обслуживания. Во-первых, это системы с одним, несколькими, а также неограниченным числом обслуживающих приборов (см.~\cite{afanasyeva, tatashev, brumelle}). Далее рассматривались как системы с пуассоновскими входными потоками требований, так и системы с более сложной структурой, например, дважды стохастические входные потоки (см.~\cite{dudin:1997, Grandell:1976, Neuts:1979}), то есть допускающие изменение мгновенной интенсивности во времени в соответствии с заданным стохастическим процессом. В некоторых системах для требований предполагается возможность  уходить на <<орбиту>> в случае отсутствия свободных приборов, то есть требования через случайное время после первой попытки запрашивают обслуживание снова (см.~\cite{fallin}). 

Также известны и достаточно сложные технические системы, в которых требования неоднородны и потоки требований разных типов оказываются конфликтными. Конфликтность означает, что в каждый момент времени могут обслуживаться требования не более чем из одного потока. Примерами таких систем являются пересечение транспортных магистралей~--- перекрестки, взлетно-посадочные комплексы в аэропортах, локальные вычислительные сети и сети передачи данных. В~конфликтных системах обслуживания обслуживающее устройство с необходимостью выполняет функцию управления потоками. В существующей литературе алгоритмы управления конфликтными потоками делятся на два типа: независящие от состояния системы (см. работы \cite{Darroch:1964,Neimark:1966}) и зависящие от нее  (см. работы \cite{Neimark:1968, Fedotkin:1976, Ferguson:1985, Takagi:1985}). В частности, в работе Ю.И.~Неймарка и М.А.~Федоткина \cite{Neimark:1966} строится модель перекрестка, для которого длительности сигналов светофора фиксированы, и находятся вероятностные характеристики стационарного режима. В работе \cite{Neimark:1968} авторы изучают алгоритм, учитывающий информацию о количестве машин в очередях в момент принятия решения. Обобщение на случай произвольного числа конфликтных потоков впервые было осуществлено в статье \cite{Yakushev:1990}.

 
  
Тандемы систем массового обслуживания широко используются при моделировании компьютерных и коммуникационных систем, колл-центров, аварийных служб, при планировании их мощностей, производительности и последующей оптимизации работы. 
Тандем является простейшей сетью из нескольких приборов, в которой заявка после обслуживания одним устройством  поступает в очередь на обслуживание следующим устройством.
Одной из первых работ, посвященных тандемам систем массового обслуживания, является работа \cite{reich}. В ней изучается распределение времени пребывания требования в системе с двумя обслуживающими устройствами. В предположении, что промежутки времени между поступлениями заявок в систему и времена обслуживания независимы и имеют экпоненциальные законы распределения, было показано, что время ожидания требования в очереди первого прибора стохастически не зависит от его времени ожидания в очереди второго прибора. Основные результаты теории тандемов в случае простейших стационарных входных потоков и экспоненциального времени обслуживания широко представлены, например, в работах \cite{Balsamo, GnedenkoKonig}. Модели с неэкспоненциальным временем обслуживания рассмотрены в работах \cite{Gomez:1,Gomez:2,Gomez:3}. Более общие модели включают в себя так называемые BMAP (Batch Markovian Arrival Process) входные потоки, особенностью которых является наличие корреляции количества пришедших требований в различные моменты времени. Такие потоки рассмотрены, например, в работах \cite{KlimenokDudin:2005,KlimenokDudin:2004,Klimenok:2010,Klimenok:2011,Klimenok:2015 }, где проведены аналитические расчеты условий стационарности и изучено поведение некоторых характеристик обслуживания для некоторых частных видов входных потоков и распределений времени обслуживания для двухфазных (тандемных) систем, в том числе с повторными попытками и нетерпеливыми требованиями.  

В связи со стремительным ростом числа машин в современных городах, все больший интерес стала представлять теория потоков транспортных средств. Результаты ранних исследований по этой тематике собраны, например, в книгах \cite{Haight:1963, Drew:1968,Inose}. Потоки машин обычно моделируются с помощью традиционных стохастических потоков событий, весьма полно изученных в классической теории массового обслуживания.  Динамика, обусловленная возможностью съезда машин с трассы, рассматривается в работах \cite{AfanasyevaBulinskaya:2013:1,AfanasyevaBulinskaya:2010,AfanasyevaBulinskaya:2013:2}. Основным объектом изучения в этих работах является плотность потока машин как функция от расстояния, на основе которой делаются выводы о пропускной способности перекрестков.

Управление уличным движением с помощью светофоров привело к исследованиям систем массового обслуживния с переменной структурой обслуживающего устройства. Работы М.Г.~Теплицкого \cite{Teplicki:1968, Teplicki:1969} содержат одни из первых исследований в этом направлении. М.А.~Федоткин и Ю.И.~Неймарк в своих работах \cite{Neimark:1966, Fedotkin:1969}  рассматривают управление потоками автомобилей на перекрестке, используя аппарат математической кибернетики и теории массового обслуживания. В данной системе путем изменения сигнала автомата-светофора возможно управлять режимами обслуживания входных потоков. Был найден оптимальный набор параметров управления для автомата-светофора. В работе~\cite{Neimark:1985} анализ системы проводится с применеием методов имитационного моделирования, поскольку аналитическое исследование вызывает большие сложности. В более поздних работах рассматривались, например, такие алгоритмы изменения состояния обслуживающего устройства, как циклический алгоритм (работы \cite{Proidakova:2008, Fedotkin:2014, Zorin:2014}), алгоритм с петлей (например, работа \cite{Zorin:2017}) и алгоритм с упреждением (например, работа \cite{Kuvykina:1990}). Адаптивные алгоритмы управления конфликтными потоками рассматривались в работах \cite{Kudelin:1996, Litvak:2000}. В контексте решения задач более специфичных для области исследования операций отметим следующие публикации: в работе \cite{Dunne:1964} исследуется линейный управляющий алгоритм, в работе \cite{Gordon:1969}~--- адаптивный алгоритм с информацией о размере очередей, в работе \cite{Day2012}~--- гибридный алгоритм (алгоритм с упреждением и алгоритм с обратной связью), в работах \cite{Vasilakos:1990,Cotton:1995,Mason:1999,Kokkonis:2016} рассматриваются комбинации нескольких адаптивных алгоритмов.

С самых истоков становления теории массового обслуживания перед исследователями, среди прочего, стояла оптимизационная задача. Так, например, в упомянутой выше работе А.К.~Эрланга ставилась задача поиска оптимального (минимального) числа телефонных линий для удовлетворительного обслуживания абонентов. Как отмечалось выше, в терминах дисциплины исследования операций действие по обслуживанию требований является <<операцией>> над некоторыми абстрактными объектами (абонентами, заявками, требованиями), эффективность осуществления которой является одним из предметов исследования. При этом операция по обслуживанию зачастую выполняется под действием как детерминированных, так и случайных факторов. В качестве критериев эффективности операции обслуживания обычно выступают вероятность простоя обслуживающих устройств, среднее время пребывания требования в системе, вероятность отказа устройства, средняя длина очереди, среднее количество занятых устройств и т.д. С учетом названных критериев формируется конечный функционал, оптимизация которого и является главной целью. Примерами работ, в которых задача теории массового обслуживания ставится в терминах исследования операций, являются \cite{Bailey:1957, Flagle:1962, Buslenko:1968}.
%Taha, Carter, Murthy

С точки зрения учебной литературы, теория массового обслуживания также является неотъемлимой и устоявшейся ветвью исследования операций. Классические учебники (см. \cite{Taha, Carter, Murthy, Hillier} и др.) по исследованию операций включают в себя разделы про марковские цепи, случайные процессы и теорию массового обслуживания (теорию очередей). Наряду с базовыми понятиями линейного программирования и теории игр в учебниках вводится марковская цепь, определяются основные классы ее состояний и рассматриваются вопросы эргодичности. Обобщения марковской цепи строятся посредством случайных  процессов: марковских или процессов рождения и гибели. Далее этот материал обычно применяется в разделе теории массового обслуживания. Инструменты для аппробации построенных вероятностных моделей, как правило, предоставляются в учебниках по исследованию операций в разделе с названием <<имитационное моделирование>> (<<simulation>>).

Решение оптимизационных задач в теории массового обслуживания породило понятие управляемой системы массового обслуживания, введеное в 1967~г. в работе О.И.~Бронштейна и В.В.~Рыкова \cite{BronshteinRykov}. В обзоре \cite{Rykov:1975} 1975~г. В.В.~Рыковым была проведена классификация таких систем и указана связь теории массового обслуживания с существовавшими на тот момент исследованиями в области управляемых случайных процессов. Под управляемой системой понимается любая система, хотя бы одна из составляющих (элементов) которой допускает применение управляющих воздействий. При этом основными элементами системы являются: 1) входящий поток требований; 2) длительности и механизм обслуживания; 3) структура системы; 4) дисциплина обслуживания. К примеру, для задачи об обслуживании клиентов на телефонной станции управление может быть применено к механизму обслуживания путем изменения количества обслуживающих операторов, а для задачи обслуживания автомобилей на перекрестке возможно управлять длительностью обслуживания требований для конкретного потока. Некоторые примеры управляемых систем массового обслуживания также представлены в работах \cite{VerbickyRykov, Solodyannikov, Motov}.

Для решения сформулированных выше задач применялся, как правило, классический подход. Данный подход,  имея в своем арсенале мощный аппарат теории вероятностей и случайных процессов, предполагает подробнейшее описание элементов математической модели и, в частности, описание характеристик каждой заявки в отдельности. Такое <<локальное>> задание потоков заявок обычно осуществляется при помощи следующих математических объектов: распределение длин интервалов между поступлениями заявок, целочисленный считающий процесс, точечный процесс или случайная мера (см. работы \cite{Fedotkin:1996, Kabanov:74, Franken, Hinchin, Jagerman}). Однако цена такого подробного описания~--- ограниченное количество реальных систем, для которых исследователь способен провести анализ или хотя бы построить строгую математическую модель. Так, например, при анализе потоков автотранспортных средств (см. работы \cite{Fedotkin:81,Bartlett}) интервалы между моментами поступления автомобилей к стоп-линии перекрестка оказываются статистически зависимыми. Данный факт следует из пространственной неоднородности транспортных потоков: при возникновении в потоке <<медленного>> автомобиля за ним образуется <<пачка>> автомобилей движущихся следом. Описание такого рода потоков довольно сложно построить, наблюдая за каждым отдельно взятым автомобилем. Данная проблема тесно связана с недостаточной разработанностью в настоящее время теории выходящих потоков: хорошо исследованы свойства выходящего потока только для простейших систем обслуживания. В работе \cite{Bocharov:1995} представлена сводка методов и результатов анализа выходящих потоков некоторых систем.

В контексте тандемов управляющих систем незнание характеристик выходящего потока одной подсистемы равносильно незнанию структуры входящего потока для следующей подсистемы, что существенно затрудняет анализ исследования всего тандема в целом. Такое незнание зачастую компенсируется предположением о мгновенности перемещения требований между узлами системы, что накладывает существенные ограничения на применимость результатов в реальных задачах. К примеру, в работах \cite{AfanasyevaBulinskaya:2010, AfanasyevaBulinskaya:2011} проводился анализ управления движением автомобилей между последовательными перекрестками. Предположение о небольшом расстоянии между перекрестками позволяло пренебречь временем движения между ними.  В случае отсутствия допущения о мгновенном перемещении требований анализ системы существенно затрудняется. Так, например, в работе \cite{Yamada} исследуется задержки автомобилей на смежных перекрестках. Вследствие сложности построенной математической модели, для анализа системы был выбран метод имитационного моделирования. 

Также при задании стохастических связей между элементами системы, которым должна подчиняться управляющая система, исследователь зачастую сталкивается со сложными задачами теории управляемых процессов. Кроме того, при анализе одновременно нескольких управляющих систем, их приходится задавать на едином унифицированном вероятностном пространстве. Все это существенно увеличивает сложность задачи.

Качественно новая методика к построению математических моделей управляющих систем массового обслуживания была предложена М.А.~Федоткиным в работах \cite{Fedotkin:1996, fedotkin:1998} и существенно доработана А.В.~Зориным в работах \cite{Zorin:2008:1, Zorin:2014, ZorineDissertation} и \cite{Zorin:2017}. Методика основана на понятии абстрактной управляющей системы, введеном А.А.~Ляпуновым и С.В.~Яблонским в работе \cite{Lyapunov}, и также носит название кибернетического подхода. Основными принципами подхода являются: 1) наблюдение за системой происходит в дискретные моменты времени; 2) управляющая система разделяется на логические блоки, между которыми определяются функциональные и стохастические связи при их взаимодействии во времени; 3) описание блоков системы должно быть нелокальным. В качестве блоков управляющей системы выделяют следующие: входные полюса, внешняя память, блок по переработке внешней памяти, внутренняя память, блок по переработке внутренней памяти, выходные полюса и внешняя среда. Некоторые из перечисленных блоков могут быть опущены при исследовании вследствие их вырожденности. Так, например, в работе \cite{Novgorod:2011} случайная среда имеет всего одно состояние, а в работе \cite{Zorin:2008:1}~--- несколько, поэтому блок со случайной средой во второй работе учитывается, а в первой~--- нет. Рассмотрение систем с позиции абстрактных управляющих систем Ляпунова--Яблонского позволяет исследовать их с единой позиции: разделяя системы на составные блоки, описывая каждый из них и вводя функционально-статистические связи между ними. Это существенно упрощает анализ уже известных систем, а также делает возможным исследование новых и более сложных. Кроме того, стало более естественным независимое рассмотрение подсистем  и изучение их свойств отдельно от основной модели. Так, в работах \cite{FedotkinRachinskaya, Fedotkin:2012} исследуется процесс формирования автомобильных <<пачек>> на дорогах как независимая система массового обслуживания. А в работах \cite{Proidakova:2008,Fedotkin:2009} изучались выходящие потоки систем массового обслуживания с циклическим алгоритмом управления входными пуассоновскими потоками и потоками Гнеденко--Коваленко. 

Аппарат абстрактных управляющих систем Ляпунова--Яблонского был удачно применен к анализу неклассических конфликтных управляющих систем массового обслуживания А.В.~Зориным в работах \cite{Zorin:2003,Zorin:2006,Zorin:2008:2,Zorin:2009:1,Zorin:2009:2,Zorin:2010:1,Zorin:2010:2,Zorin:2011:1,Zorin:2011:2,Zorin:2013 ,Zorin:2014:1,Zorin:2014:2,Zorin:2012:1, Zorin:2014:3}. В частности, была построена математическая модель для системы управления неординарными конфликтными потоками, формируемыми во внешней среде, в классе алгоритмов с переналадками и разделением времени, а также циклических алгоритмов с продлением. Была построена модель для системы управления неординарными рекуррентными потоками в классе циклических алгоритмов при наличии переналадок. Кроме того, отметим, что исследовать систему последовательных перекрестков с различными усложнениями стало возможным только с точки зрения абстрактных управляющих систем Ляпунова--Яблонского. Так, модель тандема перекрестков с немгновенным перемещением машин между ними была впервые предложена в работах~\cite{Zorin:2010:3, Zorin:2011:2, Zorin:2012:2}. В рассматриваемых там моделях динамика перемещения машин от одного перекрестка к другому задается бернулиевской случайной величиной: каждая машина с некоторой фиксированной вероятностью $0<p<1$ успевает доехать до следующего перекрестка и с противоположной вероятностью $1-p$ остается <<между>> ними. В работах автора \cite{ Minsk2015, Dm2015, Penza2017, ORM2018, Tomsk2017, ITMM2019, DCCN2015, DCCN2016:1, DCCN2016:2, DCCN2016:3,  DCCN2017, vestnikTvGU, vestnikVGAVT2, UBS} с точки зрения управляющей системы Ляпунова--Яблонского построена и исследована математическая модель тандема перекрестков с немгновенным перемещением автомобилей и циклическим алгоритмом управления светофором с продлением. Поскольку многие реальные системы могут быть представлены в виде конфликтной управляющей системы Ляпунова--Яблонского, то исследования в данной области являются актуальными.

Аппарат абстрактных управляющих систем также облегчает асимптотический анализ поведения систем обслуживания. К более ранним работам, в которых исследовалось предельное поведение операционных характеристик (число требований в системе, время ожидания и т.п.) классическими методами, можно отнести работы \cite{Davis:1995, Borovrkov:1964, Borovrkov:1980, Afanasieva:2011,Afanasieva:2008, Whitt:1971, Whitt:1982}. Также представляют большой интерес работы, в которых определяются условия существования стационарного распределения, например, \cite{Loynes:1962, Davis:1972, Whitt:2014, Choudhury:1995}. Основными известными методами здесь принято считать прямое решение уравнений стационарности, применение теоремы эргодичности Мустафы, применение критерия Найквиста-Михайлова. Отличительными особенностями этих методов является отсутствие универсальной последовательности действий при применении их к разным типам исследуемых систем. Существенно новым методом в этом вопросе является итеративно-мажорантный метод, впервые примененный М.А.~Федоткиным в контексте абстрактных управляющих систем обслуживания (см.~\cite{Fedotkin:1988, Fedotkin:1989}). Отличительными особенностями метода являются относительно легкая проверяемость получаемых условий и алгоритмизованность. Данная методология была успешно применена в задаче исследования перекрестка с приоритетными потоками \cite{Proidakova:2007}, в задаче изучения системы со случайной внешней средой \cite{Zorin:2008:1}, в задаче адаптивного управления потоками \cite{LitvakDissertation} и др. Более глубоко метод был доработан в диссертации А.В.~Зорина \cite{ZorineDissertation}. 

%pic_serv_1_stationar.png
Кроме преимуществ при получении аналитических результатов, аппарат управляющих систем Ляпунова--Яблонского дает существенные преимущества  при построении имитационных моделей разнообразных систем обслуживания и их численном анализе (см.~\cite{Zorine:2013}). Обращаясь к истории вопроса, как правило, для имитации использовался метод дискретных событий, описанный, например, в работах \cite{Simulation, AsmussenGlynn, FedotkinRachinskaya:2016, FedotkinADissertation}. За наблюдаемые события обычно выбирались изменение состояний обслуживающего устройства, приход в систему или уход из нее каждого отдельного требования и т.д. Генерируемое при этом множество возможных событий формирует исчерпывающее и зачастую избыточное описание процесса обслуживания в системе. Такое локальное описание приводит к потреблению большого количества ресурсов и длительному времени работы при программной реализации построенной имитационной модели. В противовес локальному, нелокальное описание позволяет избежать генерирования больших объемов данных и их длительной обработки. За наблюдаемые события могут браться, к примеру, лишь моменты смены состояний обслуживающих устройств, поэтому нет необходимости анализировать каждое требование отдельно. Благодаря, в частности, нелокальному описанию элементов системы удается построить достоверные оценки интересующих параметров управляющих систем на основе большего числа экспериментов.




%Однако классические модели не удается использовать для адекватного описания реальных потоков машин \cite{Bartlet:1963}.
%В работах \cite{Fedotkin:2009,Fedotkin:Kudryavcev:Rachinskaya:2010, Rachinskaya:Fedotkin:2011:1,Rachinskaya:Fedotkin:2011:2, Fedotkin:Kudryavcev:Rachinskaya:2011, Rachinskaya:Fedotkin:2012, Rachinskaya:Fedotkin:2013, Rachinskaya:Fedotkin:2014} предлагается учитывать не только вероятностные свойства последовательности моментов пересечения машинами так называемой виртуальной стоп-линии, но и определять свойства случайных конфигураций автомобилей на дороге. В указанных работах изучается возникновение так называемых пачек машин. Каждая пачка состоит из медленной головной машины и ожидающих возможности обгона машин за ней. Динамика длины пачки определяется возможностью обгона машинами из хвоста всей пачки. 







{\aim} Целями данной работы являются: 1)~построение и исследование математической модели тандема управляющих систем обслуживания по циклическому алгоритму с продлением; 2)~построение, реализация и анализ имитационной модели систем, осуществляющих циклическое управление с продлением тандемом перекрестков.

Для~достижения поставленных целей решаются следующие задачи:

1. Построение строгой вероятностной модели тандема управляющих систем с помощью явного построения вероятностного пространства и поточечного задания необходимых для исследования случайных величин и элементов.

2. Анализ построенной вероятностной модели, получение условий существования стационарного режима в различных подсистемах тандема.

3. Разработка имитационной модели тандема, определение момента достижения системы квазистационарного режима, анализ зависимости условий стационарности от управляющих параметров.



{\novelty} Результаты диссертации являются новыми и заключаются в следующем:

1. Впервые построена вероятностная модель тандема управляющих систем с немгновенным перемещением требований между ними, управление в которых осуществляется по циклическому алгоритму и алгоритму с продлением. В этой модели требования сначала поступают в первую систему на обслуживание по циклическому алгоритму, а затем немгновенно поступают во вторую систему на обслуживание по циклическому алгоритму с продлением. Немгновенность перемещения моделируется при помощи биномиальной случайной величины с параметром $p$, имеющим смысл вероятности перехода требования из одной системы в другую за определенный промежуток времени.
%Построена строгая вероятностная модель тандема управляющих систем с немгновенным перемещением требований между ними, управление в которых осуществляется по циклическому алгоритму и алгоритму с продлением. 

2. Впервые применен аппарат абстрактных управляющих систем Ляпунова--Яблонского для изучения указанной выше системы. Построенная по принципам кибернетического подхода вероятностная модель позволила провести разносторонний анализ системы. В частности, была проведена классификация состояний марковской цепи, описывающей динамику системы, найдены рекуррентные соотношения для соответствующих производящих функций и были изучены эргодические свойства системы. Также, благодаря этому подходу, была построена и реализована имитационная модель для численного анализа системы.
%Изучены эргодические свойства построенной модели, найдены условия существования стационарного режима для очередей первичных требований, а также для промежуточной очереди.

3. Впервые применен итеративно-мажорантный метод для нахождения достаточных условий существования стационарного распределения в указанной выше модели. Благодаря итеративно-мажорантному методу были найдены условия существования стационарного режима для очередей первичных требований, а также для промежуточной очереди.

%3. Разработана и реализована имитационная модель для тандема

%4. Проведено исследование вероятностной и имитационной моделей, и определена расширенная область стационарности системы при алгоритме с продлением.




{\influence} Научная значимость работы заключается в построении строгой вероятностной модели 
для качественно нового вида управляющей системы и в последовательном исследовании ее эргодических свойств. Успешно примененный в работе метод нелокального описания процессов существенно расширяет множество поддающихся исследованию реальных систем массового обслуживания. Строгая математическая модель позволяет оперировать существующим, хорошо разработанным вероятностным аппаратом для нахождения условий стационарности и нахождения оптимального управления системой. 
 Разработанные модели дают базу для изучения более комплексных тандемных систем, систем с более сложными входными потоками и алгоритмами управления.

Практическая значимость исследования состоит в том, что изученная управляющая система является адекватным описанием реальной системы тандема перекрестков, а также других сетей, состоящих из двух узлов с перемещающимися между ними требованиями и циклическими алгоритмами обслуживания с продлением на узлах.




{\methods} 
В диссертации применяется аппарат теории вероятностей, теории массового обслуживания, исследования операций, теории управляемых марковских процессов. Также применяются методы теории линейных отображений, математической статистики и теории функций комплексного переменного.
 При реализации имитационной модели на компьютере использовались языки программирования C++, Python.

Методология диссертации основывается на   представлении стохастических систем массового обслуживания в виде абстрактных управляющих систем Ляпунова--Яблонского. Использование данной методологии  позволяет разделить исследуемые системы на составные части (блоки), описать эти части математически,  задать правила их функционирования и взаимодействия между собой.
Для описания входных потоков было примено нелокальное описание.
%, что сделало возможным более глубокое математическое изучение рассматриваемых объектов.

\pagebreak
{\defpositions}

1. Методика построения вероятностного пространства для тандема систем обслуживания по циклическому алгоритму с продлением и задержкой требований между ними.

2. Методика определения условий существования стационарного режима в системах управления неординарными пуассоновскими потоками требований с использованием циклического алгоритма и алгоритма с продлением.

3. Методика определения фазы квазистационарного режима  управляющей системы обслуживания тандемного типа.





{\probation} Достоверность полученных результатов обеспечивается строгим применением используемого математического аппарата, проведением и сравнением статистических и численных исследований. Результаты работы находятся в соответствии с результатами, полученными ранее другими авторами при исследовании управляющих систем обслуживания.

Основные результаты диссертации докладывались и обсуждались на следующих  конференциях.
\begin{enumerate}
    \item Международная научная конференция <<Теория вероятностей, случайные процессы, математическая статистика и приложения>> (Минск, Республика Беларусь, 2015 г.).
    \item IX Международная конференция <<Дискретные модели в теории управляющих систем>> (Москва и Подмосковье, 2015 г.).
\item 8-я международная научная конференция <<Распределенные компьютерные и коммуникационные сети: управление, вычисление, связь>> DCCN-2015 (Москва, 2015 г.).
\item Международная научная конференция <<Distributed Computer and Communication Networks>> DCCN 2016 (Москва, 2016 г.).
\item XVIII Международная конференция <<Проблемы теоретической кибернетики>> (Пенза, 2017 г.).
\item XVI Международная конференция имени А.Ф. Терпугова <<Информационные технологии и математическое моделирование>> ИТММ-2017 (Казань, 2017 г.).
\item  20-я международная научная конференция <<Распределенные компьютерные и телекоммуникационные сети: управление, вычисление, связь>> DCCN-2017 (Москва, 2017 г.).
\item IX Московская международная конференция по исследованию операций (Москва, 2018~г.).
\item Четвертая международная конференция по стохастическим методам МКСМ-4 (пос.~Дивноморское, г.~Новороссийск).
\item XVIII Международная конференция имени А.Ф.~Терпугова <<Информационные технологии и математическое моделирование>> ИТММ-2019 (Саратов, 2019 г.).
\end{enumerate}


{\contribution} В совместных публикациях научному руководителю принадлежит постановка задачи и общее редактирование работ. Все исследования выполнены автором диссертации лично, все полученные результаты принадлежат автору. 

{\passport} Диссертационная работа выполнена в соответствии с паспортом специальности 01.01.09 «Дискретная математика и математическая кибернетика» и включает оригинальные результаты в области дискретной математики и математической кибернетики. 

Исследование, приведенное в работе, соответствует следующим
разделам паспорта специальности: 
\begin{itemize}
    \item пункт~4 (Математическая теория исследования операций и теория игр)~--- построена и изучена математическая модель для новой системы массового обслуживания методами теории исследования операций;
    \item пункт~2 (Теория управляющих систем)~--- математическая модель исследуемой системы представлена в виде абстрактной управляющей системы Ляпунова--Яблонского и построена на базе принципов кибернетического подхода.
\end{itemize} 

\ifnumequal{\value{bibliosel}}{0}{% Встроенная реализация с загрузкой файла через движок bibtex8
    \publications\ Основные результаты по теме диссертации изложены в XX печатных изданиях, 
    X из которых изданы в журналах, рекомендованных ВАК, 
    X "--- в тезисах докладов.%
}{% Реализация пакетом biblatex через движок biber
%Сделана отдельная секция, чтобы не отображались в списке цитированных материалов
    %\begin{refsection}%
        %\printbibliography[heading=countauthornotvak, env=countauthornotvak, keyword=biblioauthornotvak, section=1]%
        %\printbibliography[heading=countauthorvak, env=countauthorvak, keyword=biblioauthorvak, section=1]%
        %\printbibliography[heading=countauthorconf, env=countauthorconf, keyword=biblioauthorconf, section=1]%
        %\printbibliography[heading=countauthor, env=countauthor, keyword=biblioauthor, section=1]%
        %\publications\ Основные результаты по теме диссертации изложены в \arabic{citeauthor} печатных изданиях\nocite{bib1,bib2}, 
        %\arabic{citeauthorvak} из которых изданы в журналах, рекомендованных ВАК\cite{vestnikUNN,vestnikVGAVT1,vestnikVGAVT2,vestnikTGU}, 
        %\nocite{DCCN2010,Minsk2011,Novgorod2011,Novosibirsk2011,DCCN2013,Kazan,Minsk2015,RachinskayaStatistics,Dm2015,DCCN2015,Dm2016,Penza2017,DCCN2017,Tomsk2017,Soloviev2017}\arabic{citeauthorconf} "--- в тезисах докладов  \cite{DCCN2010,Minsk2011,Novgorod2011,Novosibirsk2011,DCCN2013,Kazan,Minsk2015,RachinskayaStatistics,Dm2015,DCCN2015,Dm2016,Penza2017,DCCN2017,Tomsk2017,Soloviev2017}.
        	\publications\ Основные результаты по теме диссертации изложены в 17 работах, 
	5 из них "--- в журналах, рекомендованных ВАК (\cite{vestnikTvGU,UBS,vestnikVGAVT2, vestnikSGU, MKSM2019}),
	2 "--- в библиографической базе Scopus, 2 "--- в библиографической базе Web of Science, 13 "--- в библиографической базе РИНЦ (\cite{ Dm2015, Penza2017, Tomsk2017, ITMM2019, DCCN2016:3, DCCN2016:1, DCCN2016:2,  DCCN2017, vestnikTvGU, vestnikVGAVT2, UBS, vestnikSGU, MKSM2019}),
	10 "--- в тезисах докладов (\cite{Minsk2015, Dm2015, DCCN2015, DCCN2016:3, Penza2017, Tomsk2017, DCCN2017, ORM2018, ITMM2019, MKSM2019}). Получено $1$ свидетельство о государственной регистрации программы для ЭВМ~\cite{RID}.
}

% \underline{\textbf{Объем и структура работы.}} Диссертация состоит из~введения, четырех глав, заключения и~приложения. Полный объем диссертации \textbf{ХХХ}~страниц текста с~\textbf{ХХ}~рисунками и~5~таблицами. Список литературы содержит \textbf{ХХX}~наименование.
 % Характеристика работы по структуре во введении и в автореферате не отличается (ГОСТ Р 7.0.11, пункты 5.3.1 и 9.2.1), потому её загружаем из одного и того же внешнего файла, предварительно задав форму выделения некоторым параметрам

\textbf{Объем и структура работы.} Диссертация состоит из~введения, четырёх глав, заключения и библиографического списка использованной литературы. 
%% на случай ошибок оставляю исходный кусок на месте, закомментированным
%Полный объём диссертации составляет  \ref*{TotPages}~страницу с~\totalfigures{}~рисунками и~\totaltables{}~таблицами. Список литературы содержит \total{citenum}~наименований.
%
Полный объём диссертации составляет
\formbytotal{TotPages}{страниц}{у}{ы}{}, включая
\formbytotal{totalcount@figure}{рисун}{ок}{ка}{ков}. % и
%\formbytotal{totalcount@table}{таблиц}{у}{ы}{}.   
 Список литературы содержит  
\formbytotal{citenum}{наименован}{ие}{ия}{ий}.

В разделе~1.1 задача управления системой ставится на содержательном уровне. На первом этапе задача формулируется в терминах теории массового обслуживания: описываются входные потоки, вид имеющихся очередей, обслуживающее устройство и т.п. Особое внимание уделяется описанию графа переходов обслуживающего устройства, имеющей нетривиальный характер. В качестве наглядной интерпретации системы приводится тандем перекрестков. 

Раздел~1.2 содержит построение строгой математической модели для системы, которая была представлена в предыдущем разделе. Данный этап является основополагающим для всего дальнейшего анализа.  При построении модели существенно используется аппарат абстрактных управляющих систем Ляпунова--Яблонского с соблюдением основных принципов: наблюдение за системой совершается в дискретные моменты времени,  строятся функциональные и вероятностные соотношения между блоками функционирования системы, а также описание блоков системы производится нелокально.

Исследуемая в работе система характеризуется следующими объектами: одно обслуживающее устройство и четыре очереди. Последовательность \linebreak 
 $\{(\Gamma_i, \varkappa_{1,i}, \varkappa_{2,i}, \varkappa_{3,i},  \varkappa_{4,i}); i \geqslant 0\}$ служит математическим описанием этих объектов. Глава~2 посвящена тем результатам, которые удается получить для этой пятимерной последовательности, несмотря на ее сложность: доказана марковость этой последовательности, а также проведена классификация ее состояний по арифметическим свойствам переходных вероятностей. Эти результаты позволят в следующей главе доказать марковость и провести классификацию состояний для последовательностей, содержащих только часть из упомянутых пяти компонент.


В главе~3 более подробно изучаются случайные последовательности, содержащие только часть очередей из последовательности $\{(\Gamma_i, \varkappa_i);  i \geqslant 0\}$. Исключение из рассмотрения нескольких компонент пятимерной марковской цепи позволяет найти достаточные, а в особом случае и необходимое условие существования стационарного распределения. Для нахождения условий стационарности используется хорошо зарекомендовавший себя итеративно-мажорантный метод, в котором последовательность математических ожиданий компонент цепи ограничивается последовательностью более простого вида.
Ограниченность математических ожиданий вследствие особенностей цепи влечет существование стационарного распределения.


Глава~4 имеет своими целями как подтвердить результаты, полученные аналитически в предыдущих главах, так и расширить их. Для этого  была разработана имитационная модель и написана программа для ее исследования. Для определения момента достижения системой стационарного режима подсчитываются различные статистики одновременно двух систем: смещенной, то есть системы с ненулевым количеством требований, и несмещенной, то есть системы с пустыми очередями. Основным показателем качества работы системы выбрана средневзвешенная оценка времени пребывания требования в системе. В завершении главы приведены конкретные эксперименты и анализ их результатов.





