
\section*{Общая характеристика работы}

%\newcommand{\actualitySynopsis}{\underline{\textbf{\actualityTXT}}}

% \underline{\textbf{\actualityTXT}} Теория массового обслуживания (ТМО) занимается построением и анализом моделей для сложных систем, осуществляющих большое количество однотипных операций по обслуживанию различного рода требований (заявок). Первые работы в этой области были мотивированы решением прикладных задач, связанных с организацией деятельности телефонных станций в начале XX века. Задачи, поставленные и рассмотренные Ф.В.~Йоханнсеном и А.К.~Эрлангом, заложили основу для так называемой классической ТМО. Дальнейшее развитие этой отрасли науки связано с именами таких ученых как Ф.~Поллачек, А.Н.~Колмогоров, А.Я.~Хинчин, Б.В.~Гнеденко, И.Н.~Коваленко, К.~Пальм, Д.Дж.~Кендалл, Л.~Такач, Д.Р.~Кокс, У.Л.~Смит, Т.Л.~Саати, Л.~Клейнрок, С.Н.~Бернштейн, Н.П.~Бусленко, А.А.~Боровков, В.С.~Королюк, Г.П.~Башарин, Г.П.~Климов, Ю.В.~Прохоров, А.Д.~Соловьев,  Б.А.~Севастьянов, Н.Т.Дж.~Бейли и др. В их работах закладываются основные понятия, формируются принципы и методы решения задач ТМО. Начиная с области телефонии, результаты теории очередей находят свое применение при исследовании систем управления наземным, водным и воздушным транспортом, систем организации медицинских учреждений, биологических систем, телекоммуникационных и компьютерных систем, процессов производства сложных объектов, в финансовой сфере и т.~д.

% Оптимизационные цели существуют у большинства прикладных исследований ТМО. В некоторой идеализации система массового обслуживания (СМО) есть система, которая, находясь под действием различных случайных, неопределенных, контролируемых факторов, осуществляет операции по обслуживанию требований. При этом возможно определить различные критерии эффективности осуществления операции: среднее время пребывания требования в системе, вероятность простоя обслуживающих приборов, вероятность отказа, среднее количество занятых приборов, средняя длина очереди, коэффициент загрузки системы, производительность системы и т.~д. Целью исследования таких систем является определение способов достижения наибольшей прибыли (наибольшей эффективности системы). В связи с такой постановкой задачи, ТМО неразрывно связана с отраслью исследования операций (ИО). Такая связь наблюдается, например, в работах Н.П.~Бусленко, Н.Т.Дж.~Бейли и др. Задачи ТМО в этом случае понимаются как задачи организационно-управленческого  характера, направленные на наиболее оптимальное использование ресурсов. 

% Следует также обратить внимание на связь ТМО и математической кибернетики (МК). В работе А.А.~Ляпунова и С.В.~Яблонского авторы выделяют понятие управляющей системы как одно из ключевых понятий МК. Кибернетика представляется как «наука об общих закономерностях строения управляющих систем и течения процессов управления». При изучении конкретных управляющих систем кибернетика взаимодействует со многими другими областями знаний, в том числе и с ТМО. В связи с этим привлечение аппарата и результатов МК, ИО и других дисциплин представляется актуальным и позволяет получать новые результаты.

% Начиная со второй половины XX века, появляются работы, посвященные теории управляемых систем обслуживания. Понятие управляемой СМО было введено О.И.~Бронштейном и В.В.~Рыковым. Отмечено, что в управляемых СМО можно выделить элементы, допускающие применение управляющих воздействий.  Каждый подобный элемент характеризуется набором параметров. Выбор значений управляющих параметров является стратегией управления. Изучению управляемых СМО  посвящены работы Н.М.~Воробьева, Б.Г.~Питтеля, А.Ф.~Терпугова, В.В.~Рыкова и др.

% Отметим несколько направлений исследований, связанных с тематикой диссертационной работы. Одним из важных направлений является изучение входных потоков системы. В первых работах по ТМО самой распространенной моделью входного потока служил простейший поток, или поток Пуассона. Действительно, многие реальные потоки требований обладают свойствами стационарности, ординарности и отсутствия последействия. Например, к таким потокам относятся транспортные потоки на крупных магистралях, потоки покупателей в крупных супермаркетах, потоки пациентов в поликлинику в период отсутствия массовых заболеваний, потоки отказов элементов сложных технических устройств. Однако часто реальные потоки составлены из неоднородных, зависимых требований. Влияние внешних условий на формирование потока приводит к тому, что проявляется неоднородность требований. При исследовании различных механизмов образования потока возникают такие модели, как поток Гнеденко--Коваленко, поток Бартлетта. Актуальность диссертационного исследования в этом направлении обусловлена тем, что в работе изучается механизм образования зависимости между требованиями, на его основе строится модель реальных потоков в виде неординарных пуассоновских потоков. 

% Следующее направление исследований связано с изучением СМО с переменной структурой обслуживающего устройства (ОУ). В работах М.А.~Федоткина методами ТМО и МК решалась задача управления потоками машин на перекрестке. В качестве ОУ рассматривался перекресток с установленным автоматом-светофором. Светофор может находиться в одном из множества состояний, меняющихся согласно некоторому закону. Каждое состояние характеризуется определенным режимом обслуживания входных потоков. Были найдены оптимальные значения для управляющих параметров светофора. Были рассмотрены системы с различными законами смены состояний ОУ: например, изучался циклический алгоритм для различных входных потоков, алгоритм с петлей, алгоритм с упреждением и другие адаптивные алгоритмы. Подобные модели дают адекватное математическое описание многих реальных сложных управляемых процессов обслуживания, учитывающих воздействие случайных факторов. Также следует отметить ряд работ, изучающих алгоритмическое управление потоками в рамках ИО. Наиболее наглядным приложением таких моделей являются системы управления дорожным транспортом. Например, в работе М.К.~Данна и Р.Б.~Потса рассматривается линейный управляющий алгоритм, определяются условия стабильности управления потоками, изучаются условия, при которых минимизируются средние задержки в обслуживании. В работе Р.Л.~Гордона изучается адаптивный алгоритм, использующий информацию о длинах очередей. В работах К.М.~Дэя, Д.М.~Баллока и соавторов рассматривается совместное использование двух управляющих алгоритмов (алгоритм с обратной связью и алгоритм с упреждением), обосновывается эффективность такого подхода. Конечной прикладной целью исследований является определение оптимальной стратегии управления системой. Работы Дж.~Хуанга, Б.~Кармели и соавторов содержат исследование системы обслуживания потоков разноклассовых клиентов в отделении неотложной помощи. Управление потоками в данной работе осуществляется на основе обратной связи по количеству заявок в очереди ожидания и времени пребывания в системе заявок, находящихся на обслуживании. Устанавливаются  условия асимптотической оптимальности с использованием методов компьютерной имитации. Отметим, что многие подобные исследования существенно опираются на физические характеристики и особенности системы. Если постановка задачи формулируется при изучении реальной физической системы, то зачастую результаты исследований сложно перенести на задачи другой физической природы. В этом смысле диссертационная работа является актуальной, поскольку предлагает рассматривать управляющие системы и алгоритмы, абстрагируясь от физической постановки задачи.


% Отметим также направление, связанное с исследованием предельного поведения СМО и условий ее стационарности (работы А.А.~Боровкова, Л.Г.~Афанасьевой, Е.В.~Булинской, В.~Уитта, Дж.~Дэвиса и др.). Многие из подобных исследований направлены на асимптотический анализ операционных характеристик системы (время ожидания, размер очереди, число требований в системе и т.~п.). Теоретический и прикладной интерес представляют работы, в которых определяются условия существования стационарного режима в системах обслуживания. Частой методологией отыскания условий являются интегральные преобразования функций, характеризующих систему. Также используются известные результаты теории управления и теории цепей Маркова, например, критерий устойчивости Найквиста--Михайлова, теорема эргодичности Мустафы и др. Нередко результатом подобных исследований являются условия существования стационарного режима, которые сложно проверить для реальных систем. Свойство стационарности системы сопряжено с понятием ее управляемости. Исследование стационарного режима является важным этапом при решении задачи оптимального управления системой. Желательным является получение условий, зависящих от управляющих параметров. В диссертационной работе применяется итеративно-мажорантный метод для отыскания легко проверяемых условий существования стационарного режима для двух новых систем управления конфликтными потоками.

% Наконец, следует обратить внимание на направление, связанное с приоритетными системами. Системы с абсолютным, относительным или иным приоритетом в разное время изучались И.М.~Духовным, О.И.~Бронштейном, А.В.~Печинкиным, П.П.~Бочаровым, В.Г.~Ушаковым и др. Системы, в которых входящие требования разнородны и могут быть разделены на классы, получают широкое распространение. В частности, это объясняется тем, что приоритетные системы служат математическими моделями для информационно-вычислительных систем и современных мультисерверных коммуникационных и компьютерных сетей. В диссертационной работе рассматривается как система с однородными входными потоками, так и система, в которой потоки различаются по своему приоритету. Во втором случае для управления потоками необходим адаптивный алгоритм. Кроме того, характеристика эффективности работы системы при решении оптимизационной задачи также должна учитывать приоритет заявок.

% Постановка задачи в первых классических работах по ТМО сводилась к поиску оптимального числа обслуживающих приборов, минимизирующего среднее время ожидания клиентов. Решение при этом находилось аналитически. Со временем системы, изучаемые методами ТМО и ИО, значительно усложнялись. В связи с этим возникает необходимость в новых критериях оценки качества функционирования системы, а также в новых методах исследования. Одним из самых распространенных методов при решении подобных оптимизационных задач на текущий момент является метод имитационного моделирования. Компьютерные имитационные модели позволяют учитывать достаточно большое число факторов, которые с трудом могут быть учтены при аналитическом исследовании в силу его возрастающей сложности. Кроме того, преимущество имитационных моделей связано с возможностью исследовать различные сценарии работы управляемых систем, сравнительно легко адаптировать модели к изменениям в физической постановке задачи. В диссертационной работе аналитические методы применяются наряду с численным исследованием путем имитационного моделирования. Такое объединение методологий представляется актуальным и увеличивает достоверность результатов.
% \newcommand{\progress}{\underline{\textbf{\progressTXT}}}
% \newcommand{\aim}{\underline{{\textbf\aimTXT}}}
% \newcommand{\tasks}{\underline{\textbf{\tasksTXT}}}
% \newcommand{\novelty}{\underline{\textbf{\noveltyTXT}}}
% \newcommand{\influence}{\underline{\textbf{\influenceTXT}}}
% \newcommand{\methods}{\underline{\textbf{\methodsTXT}}}
% \newcommand{\defpositions}{\underline{\textbf{\defpositionsTXT}}}
% \newcommand{\reliability}{\underline{\textbf{\reliabilityTXT}}}
% \newcommand{\probation}{\underline{\textbf{\probationTXT}}}
% \newcommand{\contribution}{\underline{\textbf{\contributionTXT}}}
% \newcommand{\publications}{\underline{\textbf{\publicationsTXT}}}
% %{\aim} Целями данной работы являются: 1)~построение и исследование математической модели тандема управляющих систем обслуживания по циклическому алгоритму с продлением; 2)~построение, реализация и анализ имитационной модели систем, осуществляющих циклическое управление с продлением тандемом перекрестков.

Для~достижения поставленных целей решаются следующие задачи:

1. Построение строгой вероятностной модели тандема управляющих систем с помощью явного построения вероятностного пространства и поточечного задания необходимых для исследования случайных величин и элементов.

2. Анализ построенной вероятностной модели, получение условий существования стационарного режима в различных подсистемах тандема.

3. Разработка имитационной модели тандема, определение момента достижения системы квазистационарного режима, анализ зависимости условий стационарности от управляющих параметров.



{\novelty} Основные результаты являются новыми и состоят в следующем:

1. Построена строгая вероятностная модель тандема управляющих систем с немгновенным перемещением требований между ними, управление в которых осуществляется по циклическому алгоритму и алгоритму с продлением. 

2. Изучены эргодические свойства построенной модели, найдены условия существования стационарного режима для очередей первичных требований, а также для промежуточной очереди.

3. Разработана и реализована имитационная модель для тандема

4. Проведено исследование вероятностной и имитационной моделей, и определена расширенная область стационарности системы при алгоритме с продлением.




{\influence} Научная значимость работы заключается в построении строгой вероятностной модели 
для качественно нового вида управляющей системы и в последовательном исследовании ее эргодических свойств. Успешно примененный в работе метод нелокального описания процессов существенно расширяет множество поддающихся исследованию реальных систем массового обслуживания. Строгая математическая модель позволяет оперировать существующим, хорошо разработанным вероятностным аппаратом для нахождения условий стационарности и нахождения оптимального управления системой. 
 Разработанные модели дают базу для изучения более комплексных тандемных систем, систем с более сложными входными потоками и алгоритмами управления.

Практическая значимость исследования состоит в том, что изученная управляющая система является адекватным описанием реальной системы тандема перекрестков, а также других сетей, состоящих из двух узлов с перемещающимися между ними требованиями и циклическими алгоритмами обслуживания с продлением на узлах.




{\methods} Методология диссертационной работы базируется на представлении стохастических систем массового обслуживания в виде кибернетических управляющих систем. Применение принципов кибернетического подхода позволяет выделить в изучаемых системах ключевые блоки, структурировать информацию о законах функционирования блоков и основных связях между ними. Для описания входных потоков был примен метод нелокального описания, что сделало возможным более глубокое математическое изучение рассматриваемых объектов. В работе используется аппарат теории вероятностей, теории массового обслуживания, исследования операций, теории управляемых марковских процессов, теории функций комплексного переменного. Также применяются методы математической статистики, матричных вычислений и теории систем линейных алгебраических уравнений. При разработке имитационных моделей использовался язык программирования C++. Для визуализации результатов некоторых численных исследований использовался язык Python.


{\defpositions}

1. Методика построения вероятностного пространства для тандема систем с немгновенным перемещением требований между ними.

2. Методика нахождения условий существования стационарного режима в системах управления потоками неоднородных требований с циклическим алгоритмом и алгоритмом с продлением.

3. Метод определения момента достижения управляемой системой обслуживания квазистационарного режима.





{\probation} Достоверность полученных результатов обеспечивается строгим применением используемого математического аппарата, проведением статистических и численных исследований. Результаты работы находятся в соответствии с результатами, полученными ранее другими авторами при исследовании управляющих систем обслуживания.

Основные результаты диссертации докладывались и обсуждались на следующих  конференциях.
\begin{enumerate}
    \item Международная научная конференция <<Теория вероятностей, случайные процессы, математическая статистика и приложения>> (Минск, Республика Беларусь, 2015 г.).
    \item IX Международная конференция <<Дискретные модели в теории управляющих систем>> (Москва и Подмосковье, 2015 г.).
\item 8-я международная научная конференция <<Распределенные компьютерные и коммуникационные сети: управление, вычисление, связь>> DCCN-2015 (Москва, 2015 г.).
\item Международная научная конференция <<Distributed Computer and Communication Networks>> DCCN 2016 (Москва, 2016 г.).
\item XVIII Международная конференция <<Проблемы теоретической кибернетики>> (Пенза, 2017 г.).
\item XVI Международная конференция имени А.Ф. Терпугова <<Информационные технологии и математическое моделирование>> ИТММ-2017 (Казань, 2017 г.).
\item  20-я международная научная конференция <<Распределенные компьютерные и телекоммуникационные сети: управление, вычисление, связь>> DCCN-2017 (Москва, 2017 г.).
\item IX Московская международная конференция по исследованию операций (Москва, 2018~г.).
\end{enumerate}


{\contribution} В совместных публикациях научному руководителю принадлежит постановка задачи и общее редактирование работ. Все исследования выполнены автором диссертации лично, все полученные результаты принадлежат автору. 


\ifnumequal{\value{bibliosel}}{0}{% Встроенная реализация с загрузкой файла через движок bibtex8
    \publications\ Основные результаты по теме диссертации изложены в XX печатных изданиях, 
    X из которых изданы в журналах, рекомендованных ВАК, 
    X "--- в тезисах докладов.%
}{% Реализация пакетом biblatex через движок biber
%Сделана отдельная секция, чтобы не отображались в списке цитированных материалов
    %\begin{refsection}%
        %\printbibliography[heading=countauthornotvak, env=countauthornotvak, keyword=biblioauthornotvak, section=1]%
        %\printbibliography[heading=countauthorvak, env=countauthorvak, keyword=biblioauthorvak, section=1]%
        %\printbibliography[heading=countauthorconf, env=countauthorconf, keyword=biblioauthorconf, section=1]%
        %\printbibliography[heading=countauthor, env=countauthor, keyword=biblioauthor, section=1]%
        %\publications\ Основные результаты по теме диссертации изложены в \arabic{citeauthor} печатных изданиях\nocite{bib1,bib2}, 
        %\arabic{citeauthorvak} из которых изданы в журналах, рекомендованных ВАК\cite{vestnikUNN,vestnikVGAVT1,vestnikVGAVT2,vestnikTGU}, 
        %\nocite{DCCN2010,Minsk2011,Novgorod2011,Novosibirsk2011,DCCN2013,Kazan,Minsk2015,RachinskayaStatistics,Dm2015,DCCN2015,Dm2016,Penza2017,DCCN2017,Tomsk2017,Soloviev2017}\arabic{citeauthorconf} "--- в тезисах докладов  \cite{DCCN2010,Minsk2011,Novgorod2011,Novosibirsk2011,DCCN2013,Kazan,Minsk2015,RachinskayaStatistics,Dm2015,DCCN2015,Dm2016,Penza2017,DCCN2017,Tomsk2017,Soloviev2017}.
        	\publications\ Основные результаты по теме диссертации изложены в 10 работах, 
	1 из них "--- в журнале, рекомендованном ВАК для защиты по специальности 01.01.09,
	2 "--- в библиографической базе Scopus, 2 "--- в библиографической базе Web of Science, 1 "--- в журнале, рекомендованном ВАК для защиты по смежной специальности 05.13.01, 9 "--- в библиографической базе РИНЦ,
	8 "--- в тезисах докладов. 
	%  \end{refsection}

}



 % Характеристика работы по структуре во введении и в автореферате не отличается (ГОСТ Р 7.0.11, пункты 5.3.1 и 9.2.1), потому ее загружаем из одного и того же внешнего файла, предварительно задав форму выделения некоторым параметрам


% {\aim} Целями данной работы являются: 1)~разработка и исследование математической модели потоков неоднородных требований; 2)~разработка и исследование математических и имитационных моделей систем, осуществляющих операции по управлению потоками и обслуживанию их неоднородных требований.

% Для~достижения поставленных целей решаются следующие задачи:

% 1. Выявление принципов образования потоков неоднородных зависимых требований, построение математической модели группы (пачки) зависимых требований, построение, исследование и обоснование корректности математической модели потока пачек.

% 2. Построение и изучение математической модели системы управления несколькими потоками неоднородных требований с помощью циклического алгоритма, получение условий существования в системе стационарного режима.

% 3. Построение математической модели системы управления несколькими потоками неоднородных требований с помощью адаптивного алгоритма с пороговым приоритетом и возможностью продления обслуживания, исследование свойств модели, получение условий существования в системе стационарного режима.

% 4. Разработка имитационных моделей указанных выше систем управления потоками, определение момента достижения системами квазистационарного режима, поиск квазиоптимальных значений управляющих параметров системы.  


% {\novelty} Основные результаты являются новыми и состоят в следующем:

% 1. Представлена модель потока зависимых неоднородных требований в виде неординарного пуассоновского потока с ограниченным числом требований в пачке, обоснована корректность модели.

% 2. Построена модель системы циклического управления потоками неоднородных требований в виде многомерной цепи Маркова, получен критерий существования ее стационарного распределения.

% 3. Решена задача построения математической модели системы адаптивного управления разнородными потоками в классе алгоритмов с пороговым приоритетом и возможностью продления обслуживания, решена проблема выходного потока, исследованы эргодические свойства такой системы, получены необходимые и достаточные условия существования в ней стационарного режима.

% 4. Разработан алгоритм определения момента окончания квазипереходных процессов в указанных системах.

% 5. Проведено оригинальное исследование по поиску квазиоптимальной стратегии адаптивного управления потоками.


% {\influence} Научная ценность данной работы состоит в расширении класса алгоритмов управления конфликтными потоками. Так, был изучен адаптивный алгоритм с пороговым приоритетом и возможностью продления обслуживания. Проведенные исследования увеличивают базу для дальнейшего сравнения эффективности различных алгоритмов управления конфликтными потоками требований. Практическая значимость работы обусловлена возможностью применения полученных результатов к реальным управляющим системам обслуживания. В частности, разработанные модели являются адекватными для систем управления транспортом, систем обработки запросов клиентов интернет-магазинов, систем обработки почтовых отправлений, телекоммуникационных систем, систем обработки и сборки деталей на промышленных предприятиях и т.~п.

% Полученные в работе результаты используются в учебном процессе при чтении специальных курсов для студентов четвертого курса Института информационных технологий, математики и механики ННГУ им. Н.И.~Лобачевского, специализирующихся на кафедре программной инженерии. 

% {\methods} Методология диссертационной работы базируется на представлении стохастических систем массового обслуживания в виде кибернетических управляющих систем. Применение принципов кибернетического подхода позволяет выделить в изучаемых системах ключевые блоки, структурировать информацию о законах функционирования блоков и основных связях между ними. В работе используется аппарат теории вероятностей, теории массового обслуживания, исследования операций, теории управляемых марковских процессов, теории функций комплексного переменного. Также применяются методы математической статистики, теории систем линейных дифференциальных уравнений и теории систем линейных алгебраических уравнений. При разработке имитационных моделей использовался язык программирования C++ и среда Microsoft Visual Studio. Для визуализации результатов некоторых численных исследований использовался язык R и среда разработки RStudio.

% {\defpositions}

% 1. Способ нелокального описания потоков зависимых неоднородных требований в виде неординарных пуассоновских потоков с максимальным количеством требованием в группе, равным трем.

% 2. Методика нахождения условий существования стационарного режима в системах управления потоками неоднородных требований циклическим алгоритмом и алгоритмом с пороговым приоритетом и возможностью продления обслуживания.

% 3. Метод определения момента достижения управляемой системой обслуживания квазистационарного режима.

% 4. Алгоритм поиска квазиоптимальной стратегии адаптивного управления неоднородными потоками, основанного на пороговом приоритете и возможности продления обслуживания.



% {\probation} Достоверность полученных результатов обеспечивается строгим применением используемого математического аппарата, проведением статистических и численных исследований. Результаты работы находятся в соответствии с результатами, полученными ранее другими авторами при исследовании управляющих систем обслуживания.

% Основные результаты диссертации докладывались и обсуждались на следующих семинарах и конференциях: Международный семинар <<Распределенные компьютерные и телекоммуникационные сети: управление, вычисление, связь>> DCCN-2010 (Москва, 2010 г.), Международная научная конференция <<Современные вероятностные методы анализа и оптимизации информационно-телекоммуникационных сетей>> (Минск, Республика Беларусь, 2011 г.), XVI Международная конференция <<Проблемы теоретической кибернетики>> (Н. Новгород, 2011 г.), Международный семинар <<Прикладные методы статистического анализа. Имитационное моделирование и статистические выводы>> (Новосибирск, 2011 г.), 17-я международная конференция <<Распределенные компьютерные и коммуникационные сети: управление, вычисление, связь>> DCCN-2013 (Москва, 2013 г.), XVII Международная конференция <<Проблемы теоретической кибернетики>> (Казань, 2014 г.), Международная научная конференция <<Теория вероятностей, случайные процессы, математическая статистика и приложения>> (Минск, Республика Беларусь, 2015 г.), Межрегиональная научно-практическая конференция, посвященная 180-летию со времени образования органов государственной статистики Нижегородской области <<Статистика в современном обществе: ее роль и значение в вопросах государственного управления и общественного развития>> (Н. Новгород, 2015 г.), IX Международная конференция <<Дискретные модели в теории управляющих систем>> (Москва и Подмосковье, 2015 г.), 18-я международная научная конференция <<Распределенные компьютерные и коммуникационные сети: управление, вычисление, связь>> DCCN-2015 (Москва, 2015 г.), XII Международный семинар имени академика О.Б. Лупанова <<Дискретная математика и ее приложения>> (Москва, 2016 г.), XVIII Международная конференция <<Проблемы теоретической кибернетики>> (Пенза, 2017 г.), 20-я международная научная конференция <<Распределенные компьютерные и телекоммуникационные сети: управление, вычисление, связь>> DCCN-2017 (Москва, 2017 г.), XVI Международная конференция имени А.Ф. Терпугова <<Информационные технологии и математическое моделирование>> ИТММ-2017 (Казань, 2017 г.), Международная научная конференция <<Аналитические и вычислительные методы в теории вероятностей и ее приложениях>> АВМТВ-2017 (Москва, 2017 г.).

% Отдельные результаты диссертации получены в рамках госбюджетной темы \hbox{№\,01201456585} <<Математическое моделирование и анализ стохастических эволюционных систем и процессов принятия решений>>.

% {\contribution} В совместных публикациях научному руководителю принадлежит постановка задачи и общее редактирование работ. Все исследования выполнены автором диссертации лично, все полученные результаты принадлежат автору. В совместных публикациях трех авторов научному руководителю принадлежит постановка задачи, второму соавтору --- результаты, полученные для транспортного потока при высокой плотности машин с быстрым движением, автору диссертации --- результаты, полученные для транспортного потока при относительно небольшой плотности машин с быстрым движением.


% \ifnumequal{\value{bibliosel}}{0}{% Встроенная реализация с загрузкой файла через движок bibtex8
% 	\publications\ Основные результаты по теме диссертации изложены в XX печатных изданиях, 
% 	X из которых изданы в журналах, рекомендованных ВАК, 
% 	X "--- в тезисах докладов.%
% }{% Реализация пакетом biblatex через движок biber
% 	%Сделана отдельная секция, чтобы не отображались в списке цитированных материалов
% 	\publications\ Основные результаты по теме диссертации изложены в 26 работах, 
% 	2 из них "--- в журналах, рекомендованных ВАК для защиты по специальности 01.01.09, 2 "--- в библиографической базе Scopus, 2 "--- в библиографической базе Web of Science, 2 "--- в журналах, рекомендованных ВАК для защиты по смежной специальности 05.13.01, 13 "--- в библиографической базе РИНЦ,
% 	16 "--- в тезисах докладов. Получено 1 свидетельство о государственной регистрации программы для ЭВМ.
% 	%  \end{refsection}
% }
% %При использовании пакета \verb!biblatex! для автоматического подсчета
% %количества публикаций автора по теме диссертации, необходимо
% %их здесь перечислить с использованием команды \verb!\nocite!.




% %Диссертационная работа была выполнена при поддержке грантов ...

% %\underline{\textbf{Объем и структура работы.}} Диссертация состоит из~введения, четырех глав, заключения и~приложения. Полный объем диссертации \textbf{ХХХ}~страниц текста с~\textbf{ХХ}~рисунками и~5~таблицами. Список литературы содержит \textbf{ХХX}~наименование.

% %\newpage
% \section*{Содержание работы}
% Во \underline{\textbf{введении}} приводится обзор научной литературы по изучаемой проблематике, обосновывается актуальность проводимых исследований и дается краткая характеристика работы, содержащая цели, задачи и основные результаты исследований, сведения об апробации результатов, а также теоретическую и практическую значимость работы.

% \underline{\textbf{Первая глава}} посвящена построению и изучению математической модели потока неоднородных требований. Одной из наиболее распространенных моделей потока случайных требований является рекуррентный поток. Случайные интервалы между поступлениями требований по такому потоку есть независимые и одинаково распределенные величины. Однако, если предположить наличие факторов, обуславливающих зависимость между поступлениями требований, то структура и свойства потока меняются. В подобных случаях необходимо исследовать и формализовать принципы образования взаимосвязей между соседними заявками с целью построения адекватной модели потока. В разделе~1.1 на примере транспортного потока предлагается механизм образования скопления неоднородных автомобилей в группы -- пачки. Каждая пачка состоит из одного медленного автомобиля и, возможно, нескольких быстрых, движущихся за ним и ожидающих возможности совершить обгон. При этом транспортная пачка рассматривается как управляющая система. Для построения ее модели применяется кибернетический подход Ляпунова--Яблонского. Введены в рассмотрение следующие случайные величины, характеризующие изменение количества автомобилей в пачке с течением времени $t \geq 0$ при $\Delta t > 0$: 1)~$\eta_0(t, \Delta t)$ -- число быстрых автомобилей, поступивших в пачку за промежуток времени $[t, t + \Delta t)$; 2)~$\mbox{\ae}_0(t)$ -- число всех автомобилей в пачке в момент времени $t$; 3)~$\xi_0(t, \Delta t)$ -- число автомобилей, покинувших пачку (совершивших обгон) на промежутке времени $[t, t + \Delta t)$. При некоторых предположениях можно считать что $\eta_0(t, \Delta t) \in \{0, 1, \ldots\}$ имеет пуассоновское распределение с параметром $\lambda_0 \Delta t$. В случае, если  интенсивность обгона в транспортном потоке достаточно высока, будет наблюдаться образование пачек, содержащих не более, чем фиксированное количество $N \geq 2$ автомобилей. Тогда величина $\mbox{\ae}_0(t)$ принимает значения из конечного множества $\{1, 2, \ldots, N\}$. На введенные величины естественным образом налагается ограничение $\xi_0(t, \Delta t) \leq \mbox{\ae}_0(t) + \eta_0(t, \Delta t) - 1$.

% Предполагается, что среднее время обгона зависит от количества автомобилей, находящихся в пачке. Пусть $\mu_{1}^{-1}$ и $\mu_{2}^{-1}$ -- средние времена обгона в случаях, когда пачка содержит один и два быстрых автомобиля соответственно. Если пачка содержит три и более быстрых автомобилей, среднее время обгона не меняется и равно $\mu_{3}^{-1}$. Задается условное распределение $\{\mathbf{P}(\xi_0(t, \Delta t) = x \;|\; \mbox{\ae}_0(t) = y, \eta_0(t, \Delta t) = z),\; x \in \{0, 1, \ldots, y + z - 1\}\}$ величины $\xi_0(t, \Delta t)$:
% \begin{equation}
% \label{S:conditional_probability_3} 
% \begin{gathered}
% \mathbf{P}(\xi_0(t, \Delta t) = 0 \;|\; \mbox{\ae}_0(t) = 1, \eta_0(t, \Delta t) = 0) = 1, \\
% \mathbf{P}(\xi_0(t, \Delta t) = 0 \;|\; \mbox{\ae}_0(t) = 1, \eta_0(t, \Delta t) = 1) = 1 - o(\Delta t),\\
% \mathbf{P}(\xi_0(t, \Delta t) = 1 \;|\; \mbox{\ae}_0(t) = 2, \eta_0(t, \Delta t) = 0) = \mu_1 \Delta t - o(\Delta t), \\
% \mathbf{P}(\xi_0(t, \Delta t) = 0 \;|\; \mbox{\ae}_0(t) = 2, \eta_0(t, \Delta t) = 0) = 1 - \mu_1 \Delta t + o(\Delta t), \\
% \mathbf{P}(\xi_0(t, \Delta t) = 1 \;|\; \mbox{\ae}_0(t) = 3, \eta_0(t, \Delta t) = 0) = \mu_2 \Delta t - o(\Delta t), \\
% \mathbf{P}(\xi_0(t, \Delta t) = 0 \;|\; \mbox{\ae}_0(t) = 3, \eta_0(t, \Delta t) = 0) = 1 - \mu_2 \Delta t + o(\Delta t), \\
% \mathbf{P}(\xi_0(t, \Delta t) = 1 \;|\; \mbox{\ae}_0(t) = k, \eta_0(t, \Delta t) = 0) = \mu_3 \Delta t - o(\Delta t), \\
% \mathbf{P}(\xi_0(t, \Delta t) = 0 \;|\; \mbox{\ae}_0(t) = k, \eta_0(t, \Delta t) = 0) = 1 - \mu_3 \Delta t + o(\Delta t),\\
% k \in \{4, 5, \ldots, N\},\\
% \mathbf{P}(\xi_0(t, \Delta t) = 1 \;|\; \mbox{\ae}_0(t) = N, \eta_0(t, \Delta t) = 1) = 1.
% \end{gathered}
% \end{equation}
% Перейдя к обозначениям  $Q(t, k) = \mathbf{P}(\mbox{\ae}_0(t) = k)$ при $k \in \{1, 2, \ldots, N\}$ и $t \geq 0$, из~\eqref{S:conditional_probability_3} получим систему уравнений
% \begin{equation*} 
% \begin{gathered}
% d Q(t, 1)/d t =  - \lambda_0 Q(t,1) + \mu _{1, 0} Q(t, 2),\\
% d Q(t, 2)/d t = \lambda_0 Q(t, 1) - (\lambda_0 + \mu_{1, 0}) Q(t, 2)+ \mu_{2, 0} Q(t, 3),\\
% d Q(t, 3)/d t = \lambda_0 Q(t, 2) - (\lambda_0 + \mu_{2, 0}) Q(t, 3)+ \mu_{3, 0} Q(t, 4),\\
% d Q(t, k)/d t = \lambda_0 Q(t, k - 1) - (\lambda_0 + \mu_{3, 0}) Q(t, k) + \mu_{3, 0} Q(t, k + 1),\\
% k=4, 5, \ldots, N-1, \\
% d Q(t, N)/d t = \lambda_0 Q(t, N - 1) - \mu_{3,0} Q(t, N).
% \end{gathered}
% \end{equation*}
% Система определяет динамику распределения $\{Q(t, k),\; k \in \{1, 2, \ldots, N\}\}$  числа неоднородных автомобилей в пачке. В разделе~1.2 определяется решение указанной системы уравнений в важном для реальных потоков частном случае $N=3$. Для транспортных потоков такая ситуация складывается, как правило, когда интенсивность быстрых автомобилей превышает интенсивность медленных незначительно. Так, при использовании обозначений $\nu_1 = \frac{\lambda_0}{\mu_{1,0}}$, $\nu_2 = \frac{\lambda_0}{\mu_{2,0}}$ величины 
% \begin{equation}
% \label{S:distribution}
% p = \frac 1 {1 + \nu_1 + \nu_1\nu_2},\;\;
% q = \frac {\nu_1} {1 + \nu_1 + \nu_1\nu_2},\;\;
% s = \frac {\nu_1 \nu_2} {1 + \nu_1 + \nu_1\nu_2}
% \end{equation}
% задают предельное установившееся распределение числа требований в пачке при $N=3$: $p = \lim_{t \to \infty}Q(t, 1)$, $q = \lim_{t \to \infty}Q(t, 2)$ и $s = \lim_{t \to \infty}Q(t, 3)$.

% В разделе~1.3 производится переход от модели отдельной транспортной пачки к модели потока неоднородных автомобилей. Пусть для момента $t \geq 0$ случайная величина $\eta(t)$ считает число автомобилей, пересекших некоторую стоп-линию магистрали за промежуток времени $[0, t)$. Разницей между моментами пересечения автомобилями одной пачки стоп-линии можно пренебречь. Считается, что поток медленных автомобилей может быть аппроксимирован пуассоновским потоком с параметром $\lambda$. Тогда модель потока неоднородных автомобилей $\{\eta(t)\colon t \geq 0\}$ есть неординарный пуассоновский поток со следующими параметрами: $\lambda$~-- интенсивность вызывающих моментов, $p$, $q$, $s$ -- вероятности поступления в вызывающий момент пачки из одного, двух, трех автомобилей.

% Пусть $P_k(t) = \mathbf{P}(\eta(t) = k)$ для $k = 0, 1, \ldots$ и $t \geq 0$. 

% \textbf{Теорема 1.} Производящая функция $\Psi(t; z) = \sum_{k=0}^{\infty} P_k(t)z^k$ одномерных распределений процесса $\{\eta(t)\colon t \geq 0\}$ при $t \geq 0$ и $|z| \leq 1$ имеет вид
% 	\[
% 	\Psi(t; z) = e^{-\lambda t} \sum_{k=0}^{\infty}z^k
% 	\sum_{i=0}^{[\frac k 2]} \sum_{j=0}^{[\frac {k-2i} 3]}
% 	p^{k-2i-3j} q^i s^j \frac{(\lambda t)^{k-i-2j}} {i! j! (k-2i-3j)!}.
% 	\]

% В лемме~1 определяются параметры потока, составленного из конечной суммы независимых неординарных пуассоновских потоков исследуемого типа. В лемме~2 определяются выражения для основных числовых характеристик величины $\eta(t)$ и исследуются их экстремальные значения.

% В разделе~1.4 рассматривается метод получения нелокального описания реального потока требований. Пусть имеются данные о некотором потоке в виде последовательности $\{\tau_i^\prime,\; i = 1, 2, \ldots\}$, где $\tau_i^\prime$ есть момент поступления по потоку $i$-го требования. Первоначально с помощью известных статистических критериев проверяется гипотеза о независимости и одинаковом распределении величин $\tau_1^\prime$, $\tau_{2}^\prime - \tau_{1}^\prime$, $\tau_{3}^\prime - \tau_{2}^\prime$, \ldots. Если указанная гипотеза отвергается, предлагается разбить требования исходного потока на пачки по принципу близости требований внутри пачки. Пусть для $i \in \{1, 2, \ldots\}$ величина $\tau_i$ есть момент поступления первого требования пачки с номером~$i$, а $\eta_i$ -- количество требований в $i$-ой пачке. Разбиение потока на пачки считается успешным, если принимается как гипотеза о независимости и одинаковом распределении величин $\tau_1$, $\tau_{2} - \tau_{1}$, $\tau_{3} - \tau_{2}$, \ldots, так и аналогичная гипотеза для величин $\eta_1$, $\eta_2$, \ldots. В таком случае исходный поток можно описать нелокально последовательностью $\{(\tau_i, \eta_i), i = 1, 2, \ldots\}$. 

% Для того, чтобы аппроксимировать поток неоднородных требований неординарным пуассоновским потоком, после разбиения потока на пачки необходимо проверить согласие полученных данных с гипотетическим распределением. Для величин $\tau_1$, $\tau_{2} - \tau_{1}$, $\tau_{3} - \tau_{2}$, \ldots\, гипотетическое распределение является экспоненциальным (или смещенным экспоненциальным), для величин $\eta_1$, $\eta_2$, \ldots\, рассматривается распределение~\eqref{S:distribution}. Для проверки гипотез применяется модифицированный метод минимума $\chi^2$. При этом дается обоснование следующим оценкам неизвестных параметров распределения~\eqref{S:distribution}: $\nu_1^* = \frac{n_2 + 2n_3}{K n_1}$ и $\nu_2^* = \frac{(K n_1 + n_2 + 2n_3) n_3}{(n_2 + 2n_3)(n_1 + n_2)}$. Здесь $n_i$, $i \in \{1, 2, 3\}$, есть количество пачек, полученных в результате разбиения потока и содержащих в точности $i$ заявок. Коэффициент $K$ для транспортных потоков определяет зависимость между интенсивностью обгона $\mu_{1,0}$ и интенсивностью потока пачек, состоящих из одного автомобиля. Значение коэффициента $K$ считается заданным.

% Была разработана и зарегистрирована программа для ЭВМ <<Статистический анализ потока событий>>. Программа реализует следующие процедуры для анализа данных некоторого потока: 1)~проверка гипотезы о независимости и одинаковом распределении для интервалов $\tau_1^\prime$, $\tau_{2}^\prime - \tau_{1}^\prime$, $\tau_{3}^\prime - \tau_{2}^\prime$, \ldots\, между соседними требованиями с применением четырех статистических критериев; 2)~получение описания потока в виде последовательности $\{(\tau_{i}, \eta_i), i = 1, 2, \ldots\}$; 3)~проверка гипотез о независимости и одинаковом распределении для интервалов $\tau_1$, $\tau_{2} - \tau_{1}$, $\tau_{3} - \tau_{2}$, \ldots\, и для величин $\eta_1$, $\eta_2$, \ldots; 4)~проверка гипотез о виде распределения с сопутствующей оценкой неизвестных параметров распределений. На основе разработанной программы представленный метод исследований был апробирован на данных нескольких реальных потоков: поток импульсов вдоль нервного волокна, транспортный поток и др. Полученные результаты позволяют утверждать, что модель неординарного пуассоновского потока согласуется со многими реальными потоками неоднородных требований.

% Во \underline{\textbf{второй главе}} в разделе~2.1 приводится описание класса систем обслуживания требований и алгоритмического управления конфликтными потоками. Общая схема указанного класса изображена на рисунке~\ref{S:img:common_scheme}. На вход поступает $m \geq 2$ независимых конфликтных потоков $\overline{\Pi}_1$, $\overline{\Pi}_2$, \ldots, $\overline{\Pi}_m$. Требования потока $\overline{\Pi}_j$, $j \in J = \{1, 2, \ldots, m\}$, ожидают начала обслуживания в очереди $O_j$. Обслуживающее устройство (ОУ) с множеством состояний $\Gamma$ выполняет операции по обслуживанию требований и управлению потоками. Все состояния разделяются на два типа: состояния обслуживания и состояния переналадки. Во всех состояниях обслуживания одного потока $\overline{\Pi}_j$ интенсивность обслуживания равна $\mu_j$. В каждом из состояний ОУ пребывает в течение промежутка времени с фиксированной длительностью. После завершения такого промежутка ОУ может перейти в другое состояние или остаться в текущем. Переходы на множестве $\Gamma$ осуществляются согласно управляющему алгоритму $s(\Gamma)$. В состояниях обслуживания потока $\overline{\Pi}_j$ действует экстремальная стратегия обслуживания $\delta_j$: из очереди $O_j$ на обслуживание поступает как можно большее количество требований, но не превышающее пропускной способности ОУ. Обслуженные требования потока $\overline{\Pi}_j$ образуют выходной поток $\Pi_j^\prime$.
% \begin{figure}[ht] 
% 	\center
% 	\includegraphics [width=0.6\linewidth] {Common_scheme.pdf}
% 	\caption{Общая схема класса систем обслуживания требований и управления конфликтными потоками} 
% 	\label{S:img:common_scheme}  
% \end{figure}

% При изучении кибернетических систем представленного класса применялся подход Ляпунова--Яблонского. Для этого в схеме выделены структурные блоки: 1)~входные полюса -- входные потоки $\overline{\Pi}_1$, $\overline{\Pi}_2$, \ldots, $\overline{\Pi}_m$; 2)~внешняя память -- накопители $O_1$, $O_2$, \ldots, $O_m$; 3)~блок по переработке информации внешней памяти -- экстремальные стратегии обслуживания $\delta_1$, $\delta_2$, \ldots, $\delta_m$; 4)~внутренняя память -- ОУ с множеством состояний $\Gamma$; 5)~блок по переработке информации внутренней памяти -- управляющий алгоритм $s(\Gamma)$; 6)~выходные полюса -- выходные потоки $\Pi_1^\prime$, $\Pi_2^\prime$, \ldots, $\Pi_m^\prime$. Рассматриваемые кибернетические системы осуществляют функции управления входными потоками и обслуживания требований. Выявление функциональных и статистических связей между выделенными блоками помогает построить математические модели изучаемых систем.

% Системы внутри указанного класса различаются двумя составляющими: 1)~видом входных потоков; 2)~видом множества $\Gamma$ и алгоритма $s(\Gamma)$. При изучении реальных систем вид входных потоков обусловлен различными факторами и не находится в распоряжении исследователя. Однако исследователь может выбирать алгоритм $s(\Gamma)$ и устанавливать различные значения его управляющих параметров. Общая цель исследования подобных систем состоит в определении оптимальной стратегии управления. Оптимальной стратегией будем называть такой выбор алгоритма $s(\Gamma)$ и значений его управляющих параметров, при котором достигается минимальное значение среднего времени ожидания начала обслуживания произвольным требованием системы. 

% В диссертационной работе рассматривается случай, когда входные потоки до поступления в систему были сформированы под воздействием внешних факторов, приводящих к образованию зависимости между требованиями. Тогда поток $\overline{\Pi}_j$ может быть аппроксимирован неординарным пуассоновским потоком $\Pi_j = \{\eta_j(t)\colon t \geq 0\}$ с параметрами $\lambda_j$ -- интенсивность поступления пачек, $p_j$, $q_j$ и $s_j$ -- вероятности поступления пачки из одного, двух и трех требований соответственно. В разделах~2.2\,--\,2.4 второй главы рассмотрена система с циклическим алгоритмом управления такими потоками. Такой алгоритм часто применяется в случае однородных входных потоков. В этом случае $\Gamma = \{\Gamma^{(1)}, \Gamma^{(2)}, \ldots, \Gamma^{(2m)}\}$. Обслуживание потока $\Pi_j$ происходит в состоянии $\Gamma^{(2j-1)}$, а переналадка ОУ после обслуживания потока $\Pi_j$ происходит в состоянии $\Gamma^{(2j)}$. В состоянии $\Gamma^{(k)}$, $k \in M = \{1, 2, \ldots, 2m\}$, ОУ находится в течение промежутка длительностью $T_k > 0$. Тогда пропускная способность ОУ по потоку $\Pi_j$ есть величина $l_j = [\mu_j T_{2j-1}]$. В разделе~2.2 строится математическая модель системы. Пусть последовательность $\{\tau_i, i = 0, 1, 2, \ldots\}$ составлена из случайных последовательных моментов, в которые происходит смена состояний ОУ. Тогда временная ось $0t$ разбивается на промежутки $\Delta_{-1} = [0, \tau_0)$, $\Delta_i = [\tau_i, \tau_{i+1})$, $i \in I = \{0, 1, 2, \ldots\}$. Вводятся следующие случайные величины и элементы: 1)~$\Gamma_i \in \Gamma$ --- состояние ОУ на промежутке $\Delta_i$; 2)~$\eta_{j, i} \in X = \{0, 1, 2, \ldots\}$ --- количество требований потока $\Pi_j$, поступивших в систему на промежутке $\Delta_i$; 3)~$\mbox{\ae}_{j, i} \in X$ --- количество требований потока $\Pi_j$, находящихся в очереди $O_j$ в момент $\tau_i$; 4)~$\xi_{j, i} \in \{0, l_j\}$ --- максимальное число требований потока $\Pi_j$, которое может быть обслужено на промежутке $\Delta_i$;  5)~$\xi_{j, i}^\prime \in Y_j = \{0, 1, \ldots, l_j\}$ --- реальное число требований потока $\Pi_j$, обслуженных на промежутке $\Delta_i$ (в данном случае $i \in I \cup \{-1\}$). Циклический алгоритм смены состояний ОУ функционально выражается зависимостью $\Gamma_{i+1} = u(\Gamma_{i})$, $i \in I$, где $u(\Gamma^{(k)}) = \Gamma^{(k+1)}$ при $k \in \{1, 2, \ldots, 2m-1\}$ и $u(\Gamma^{(2m)}) = \Gamma^{(1)}$. Для величины $\eta_{j, i}$ известны условные вероятности $\mathbf P(\eta_{j, i} = n\;|\; \Gamma_i = \Gamma^{(k)}) = \varphi_j(n; T_k)$, 
% где $n \in X$, $k \in M = \{1, 2, \ldots, 2m\}$ и функции $\varphi_j\colon X \times [0, \infty) \to [0, 1]$ согласно теореме~1 имеют вид $\varphi_j(n; t) = e^{-\lambda_j t} \sum_{u=0}^{[\frac n 2]} \sum_{v=0}^{[\frac {n-2u} 3]}
% p_j^{n-2u-3v} q_j^u s_j^v \frac{(\lambda_j t)^{n-u-2v}} {u! v! (n-2u-3v)!}$. Экстремальная стратегия обслуживания $\delta_j$ формализована равенством $\xi_{j,i}^\prime = \min\{\mbox{\ae}_{j,i} + \eta_{j,i}, \xi_{j,i}\}$. В качестве случайного состояния системы c точки зрения потока $\Pi_j$ в момент $\tau_i$ выбирается вектор $(\Gamma_i, \mbox{\ae}_{j, i}, \xi_{j,i-1}^\prime)$. Функционирование системы определяется многомерной последовательностью
% \begin{equation}
% \label{S:Markov_chain_2}
% \{(\Gamma_i, \mbox{\ae}_{j,i}, \xi_{j,i-1}^\prime),\; i \in I\}.
% \end{equation}
% Обосновываются рекуррентные по $i \in I$ соотношения
% \begin{equation}
% \label{S:reccurent_relation_2}
% (\Gamma_{i+1}, \mbox{\ae}_{j,i+1}, \xi_{j,i}^\prime) = (u(\Gamma_i), \max\{0, \mbox{\ae}_{j,i} + \eta_{j,i} - \xi_{j,i}\}, \min\{\mbox{\ae}_{j,i} + \eta_{j,i}, \xi_{j,i}\}).
% \end{equation}

% \textbf{Теорема 2.} Для каждого $j \in J$ многомерная случайная последовательность~\eqref{S:Markov_chain_2}, определяемая рекуррентным соотношением~\eqref{S:reccurent_relation_2}, с заданным на пространстве $\Gamma \times X \times Y_j$ начальным распределением является однородной цепью Маркова.

% В разделе~2.3 изучаются одномерные распределения и пространство состояний цепи Маркова~\eqref{S:Markov_chain_2}. Для $j \in J$, $i \in I$, $k \in M$, $x \in X$, $y \in Y_j$ введены обозначения $Q_{j, i}(\Gamma^{(k)}, x, y) = \mathbf{P}(\Gamma_{i} = \Gamma^{(k)}, \mbox{\ae}_{j, i} = x, \xi_{j, i-1}^\prime = y)$. Лемма 3 содержит выражения для рекуррентных по $i \in I \setminus \{0\}$ соотношений между указанными одномерными распределениями. Выполнение условия нормировки для распределений, подчиняющихся рекуррентным соотношениям из леммы~3, проверяется в лемме~4. При $k \in M$, $y \in Y_j$, $|z| \leq 1$ рассматриваются производящие функции $\Phi_{j,i}(\Gamma^{(k)}, z, y) = \sum_{x=0}^{\infty}Q_{j,i}(\Gamma^{(k)}, x, y) z^x$. В леммах~5 и 6 выводятся рекуррентные соотношения для производящих функций за один шаг перехода цепи Маркова~\eqref{S:Markov_chain_2} и за $2m$ шагов соотвественно.

% \textbf{Теорема 3.} Пространство $\Gamma \times X \times Y_j$ состояний цепи Маркова~\eqref{S:Markov_chain_2} содержит незамкнутое подмножество $D_j$ несущественных состояний и замкнутое подмножество $E_j$ существенных $2m$-периодических состояний:
% \begin{equation*}
% \begin{gathered}
% D_j = \{(\Gamma^{(k)}, x, y)\colon k \in M\setminus\{2j\}, x \in X, y \in Y_j \setminus\{0\}\} \cup\\
% \cup \{(\Gamma^{(2j)}, x, y)\colon x \in X\setminus\{0\}, y \in Y_j \setminus\{l_j\}\};\\
% E_j(\Gamma^{(2j)}) = \{(\Gamma^{(2j)}, x, l_j),\; x \in X\} \cup \{(\Gamma^{(2j)}, 0, y),\;y \in Y_j \setminus\{l_j\}\}; \\
% E_j(\Gamma^{(k)}) = \{(\Gamma^{(k)}, x, 0), x \in X\},\;\; k \in M\setminus\{2j\}; \\
% E_j = \bigcup_{k=1}^{2m}E_j(\Gamma^{(k)});\;\;
% \Gamma \times X \times Y_j = D_j \cup E_j.
% \end{gathered}
% \end{equation*} 

% В разделе~2.4 определяются условия существования в системе стационарного режима по потоку $\Pi_j$, $j \in J$. Пусть $T = \sum_{k \in M}T_k$ есть длительность цикла смены состояний ОУ.

% \textbf{Теорема 4.} Необходимое и достаточное условие существования стационарного режима в системе по потоку $\Pi_j$ заключается в выполнении неравенства $\lambda_j T (2s_j + q_j + 1) - l_j < 0$.

% В свою очередь, для существования стационарного режима во всей системе необходимо и достаточно одновременного выполнения условий $\lambda_j T (2s_j + q_j + 1) - l_j < 0$ для всех $j = 1, 2, \ldots, m$.


% \underline{\textbf{Третья глава}} посвящена исследованию системы адаптивного управления неоднородными потоками. Входные потоки имеют, как и прежде, одинаковую вероятностную структуру, однако являются неоднородными, т.~е. существенно различаются интенсивностью требований и их приоритетом. Требования потока $\Pi_1$ обладают высоким приоритетом, но интенсивность поступления требований по потоку $\Pi_1$ мала. Поток $\Pi_m$ имеет большую интенсивность поступления требований, но низкий приоритет. Потоки $\Pi_2$, $\Pi_3$, \ldots, $\Pi_{m-1}$ имеют малую интенсивность требований и низкий приоритет. В этом случае циклический алгоритм управления не представляется рациональным и необходим адаптивный алгоритм. Предлагается рассматривать множество состояний ОУ вида $\Gamma = \{\Gamma^{(1)}, \Gamma^{(2)}, \ldots, \Gamma^{(2m+1)}\}$. В состоянии $\Gamma^{(k)}$, $k \in M = \{1, 2, \ldots, 2m+1\}$, ОУ находится в течение промежутка времени длительностью $T_k$. Любое из состояний вида $\Gamma^{(2j-1)}$ по-прежнему выделено для обслуживания потока $\Pi_j$. Введено дополнительное состояние $\Gamma^{(2m)}$ обслуживания потока $\Pi_m$, причем $T_{2m-1} > T_{2m}$. Пропускная способность в дополнительном состоянии обслуживания характеризуется величиной $l_m^\prime = [\mu_m T_{2m}]$, $l_m^\prime \leq l_m$. Состояния $\Gamma^{(2j)}$, $j = 1, 2, \ldots, m-1$, и $\Gamma^{(2m+1)}$ являются состояниями переналадки ОУ после обслуживания соответствующего потока. Решения о смене или продлении текущего состояния ОУ принимаются в моменты $\tau_0$, $\tau_{1}$, $\tau_2$, \ldots смены состояния ОУ. На каждом из промежутков $\Delta_i$, $i \in I$, или на их концах следующие величины и элементы, введенные в главе 2, сохраняют свой смысл и множество значений: $\Gamma_i \in \Gamma$ и $\eta_{j, i} \in X$, $\mbox{\ae}_{j, i} \in X$ при $j \in J$. Кроме того, для любого $i \in I\, \cup\, \{-1\}$ по-прежнему $\xi_{j, i}^\prime \in Y_j$, $j \in J$. Однако теперь $\xi_{j,i} \in B_j = \{0, l_j\}$ для $j \in J \setminus \{m\}$ и $\xi_{m,i} \in B_m = \{0, l_m^\prime, l_m\}$.  

% \begin{figure}[ht] 
% 	\center
% 	\includegraphics [width=0.75\linewidth] {graph_change_state_feedback.pdf}
% 	\caption{Граф алгоритма управления $s(\Gamma)$ с пороговым приоритетом и возможностью продления} 
% 	\label{S:img:graph_feedback}  
% \end{figure}  

% Предлагается алгоритм управления $s(\Gamma)$, граф переходов на множестве состояний ОУ которого представлен на рисунке~\ref{S:img:graph_feedback}. Алгоритм реализует <<пороговый>> приоритет потока $\Pi_1$ с величиной порога $h_1 \in \{0, 1, \ldots, l_1 - 1\}$. Величина $h_1$ является управляющим параметром системы. В момент $\tau_i$ происходит упреждающее вычисление количества $\mbox{\ae}_{j,i+1}$ требований в очереди в момент $\tau_{i+1}$. Если ожидается, что эта величина не достигнет к концу текущего промежутка $\Delta_i$ величины порога $h_1$, то принимается решение о дальнейшем увеличении интервала обслуживания потока с большой интенсивностью. Такое увеличение может происходить либо за счет перехода в более длительное состояние $\Gamma^{(2m-1)}$, либо за счет продления более короткого состояния $\Gamma^{(2m)}$. В противном случае принимается решение о наискорейшем переходе к состоянию обслуживания высокоприоритетного потока $\Pi_1$. Заметим, что в случае $h_1 = 0$ управляющий алгоритм вырождается в циклический алгоритм на множестве состояний $\{\Gamma^{(1)}, \Gamma^{(2)}, \ldots, \Gamma^{(2m-2)}, \Gamma^{(2m)}, \Gamma^{(2m+1)}\}$. 

% В разделе~3.2 строится математическая модель системы. Управляющий алгоритм выражается равенством $\Gamma_{i+1} = u(\Gamma_i, \mbox{\ae}_{1,i}, \eta_{1,i})$, где функция $u\colon \Gamma \times X \times X \to \Gamma$ задана поточечно:
% \begin{equation*}
% \label{3:u_function}
% u(\Gamma^{(k)}, x_1, n_1) = \begin{cases}
% \Gamma^{(k+1)},\; k \in M \setminus \{2m-2, 2m, 2m+1\};\\
% \Gamma^{(2m-1)},\; k = 2m-2,\; x_1 + n_1 < h_1;\\
% \Gamma^{(2m)},\; k = 2m-2,\; x_1 + n_1 \geq h_1; \\
% \Gamma^{(2m)},\; k = 2m,\; x_1 + n_1 < h_1;\\
% \Gamma^{(2m+1)},\; k = 2m,\; x_1 + n_1 \geq h_1;\\
% \Gamma^{(1)},\; k = 2m+1.\end{cases}
% \end{equation*}
% В качестве случайного состояния системы в момент $\tau_i$ предлагается выбирать пятимерный вектор 
% $(\Gamma_i, \mbox{\ae}_{1,i}, \mbox{\ae}_{j,i}, \xi_{1,i-1}^\prime, \xi_{j,i-1}^\prime)$, где $j \in J \setminus \{1\}$. Для компонент вектора справедливы рекуррентные по $i \in I$ соотношения:
% \begin{equation}
% \label{S:reccurent_relation}
% \begin{gathered}
% \Gamma_{i+1} = u(\Gamma_i, \mbox{\ae}_{1,i}, \eta_{1,i}), \\
% \mbox{\ae}_{1,i+1} = \max\{0, \mbox{\ae}_{1,i}+\eta_{1,i}-\xi_{1,i}\},\; \mbox{\ae}_{j,i+1} = \max\{0, \mbox{\ae}_{j,i}+\eta_{j,i}-\xi_{j,i}\},\\
% \xi_{1,i}^\prime = \min\{\mbox{\ae}_{1,i} + \eta_{1,i}, \xi_{1,i}\},\; 
% \xi_{j,i}^\prime = \min\{\mbox{\ae}_{j,i} + \eta_{j,i}, \xi_{j,i}\}.
% \end{gathered}
% \end{equation}

% \textbf{Теорема 5.} Для каждого $j \in J \setminus \{1\}$ случайная последовательность $\{(\Gamma_i, \mbox{\ae}_{1,i}, \mbox{\ae}_{j,i}, \xi_{1,i-1}^\prime, \xi_{j,i-1}^\prime), i \in I\}$, для компонент которой справедливы соотношения~\eqref{S:reccurent_relation}, с заданным на пространстве $\Gamma \times X \times X \times Y_1 \times Y_j$ начальным распределением является однородной многомерной цепью Маркова.

% Далее изучение динамики функционирования системы происходит для высокоприоритетного потока $\Pi_1$ и потока $\Pi_m$ с большой интенсивностью. В качестве основной модели рассматривается цепь Маркова
% \begin{equation}
% \label{S:Markov_chain}
% \{(\Gamma_i, \mbox{\ae}_{1,i}, \mbox{\ae}_{m,i}, \xi_{1,i-1}^\prime, \xi_{m,i-1}^\prime), i \in I\}.
% \end{equation}
% В разделе~3.3 исследуются свойства цепи~\eqref{S:Markov_chain}. Вводятся при $i \in I$, $k \in M$, $x_1, x_m \in X$, $y_1 \in Y_1$, $y_m \in Y_m$ обозначения для вероятностей, порождаемых последовательностью~\eqref{S:Markov_chain}: 
% \begin{equation*}
% \label{3:one_dimensional_notation}
% \begin{gathered}
% Q_i(\Gamma^{(k)}, x_1, x_m, y_1, y_m) = \\
% = \mathbf{P}(\Gamma_i = \Gamma^{(k)}, \mbox{\ae}_{1,i} = x_1, \mbox{\ae}_{m,i} = x_m, \xi_{1,i-1}^\prime = y_1, \xi_{m,i-1}^\prime = y_m).
% \end{gathered}
% \end{equation*}
% В лемме~7 приводится классификация по Колмогорову счетного пространства $S = \Gamma \times X \times X \times Y_1 \times Y_m$ состояний цепи~\eqref{S:Markov_chain}. Выделено множество $D$ несущественных состояний и замкнутое множество $E$ существенных апериодических состояний. 

% \textbf{Лемма 8.} При любом начальном распределении многомерной цепи Маркова~\eqref{S:Markov_chain} либо для любого $(\Gamma^{(k)}, x_1, x_m, y_1, y_m) \in S$ имеет место предельное равенство $\lim_{i\to\infty}Q_i(\Gamma^{(k)}, x_1, x_m, y_1, y_m) = 0$ и стационарного распределения не существует, либо существуют пределы $\lim_{i\to\infty}Q_i(\Gamma^{(k)}, x_1, x_m, y_1, y_m) = Q(\Gamma^{(k)}, x_1, x_m, y_1, y_m)$, причем
% \[
% Q(\Gamma^{(k)}, x_1, x_m, y_1, y_m) > 0 \text{ при }  (\Gamma^{(k)}, x_1, x_m, y_1, y_m) \in E,
% \]
% \[
% Q(\Gamma^{(k)}, x_1, x_m, y_1, y_m) = 0 \text{ при }  (\Gamma^{(k)}, x_1, x_m, y_1, y_m) \in D,
% \]
% имеет место равенство $\sum_{(\Gamma^{(k)}, x_1, x_m, y_1, y_m) \in S}Q(\Gamma^{(k)}, x_1, x_m, y_1, y_m) = 1$ и стационарное распределение существует и единственно.

% В лемме~9 выводятся рекуррентные соотношения для одномерных распределений цепи~\eqref{S:Markov_chain}, в лемме~10 проверяется условие нормировки для одномерных распределений. Леммы~11, 12 и 13 содержат выражения для рекуррентных зависимостей между производящими функциями указанных одномерных распределений.

% Раздел~3.4 посвящен получению условий, при которых в системе существует стационарный режим. Пусть $T = \sum_{k \in M}T_k$ и $T^* = T - T_{2m-1}$.

% \textbf{Теорема 6.} Если параметры системы удовлетворяют условию $\lambda_1 T^* (3 s_1 + 2 q_1 + p_1) - l_1 \geq 0$, то стационарного распределения цепи Маркова \eqref{S:Markov_chain} не существует.

% \textbf{Теорема 7.} Если параметры системы удовлетворяют условиям $\lambda_m T_{2m} (3 s_m + 2 q_m + p_m) - l_m^\prime \geq 0$ и $\lambda_m T_{2m-1} (3 s_m + 2 q_m + p_m) - l_m \geq 0$, то стационарного распределения цепи Маркова~\eqref{S:Markov_chain} не существует.

% \textbf{Теорема 8.} Если параметры системы удовлетворяют условиям $\lambda_m (T-T_{2m}) (3 s_m + 2 q_m + p_m) - l_m < 0$, $\lambda_m T (3 s_m + 2 q_m + p_m) - l_m - l_m^\prime > 0$, то стационарного распределения цепи Маркова~\eqref{S:Markov_chain} не существует.

% \textbf{Теорема 9.} Для существования стационарного распределения цепи Маркова $\{(\Gamma_i, \mbox{\ae}_{1,i}, \xi_{1,i-1}^\prime), i \in I\}$ достаточно выполнения условия $\lambda_1 T^* (3 s_1 + 2 q_1 + p_1) - l_1 < 0$.

% Следует отметить, что условия существования стационарного режима в системе, полученные в теоремах~6\,--\,9, легко проверить для реальных управляющих систем конфликтного обслуживания. Кроме того, полученным условиям можно дать простую интерпретацию для физических систем. Так, теоремы~6 и~9 дают критерий существования стационарного режима в системе по потоку $\Pi_1$ с высоким приоритетом в виде неравенства $\lambda_1 T^* (3 s_1 + 2 q_1 + p_1) < l_1$. Здесь в левой части неравенства стоит величина, оценивающая среднее число заявок потока $\Pi_1$, поступивших в систему за промежуток времени длительностью $T^*$. Такой промежуток времени отвечает циклу смены состояний устройства обслуживания вида $\Gamma^{(1)} \to \Gamma^{(2)} \to \ldots \to \Gamma^{(2m-2)} \to \Gamma^{(2m)} \to \Gamma^{(2m+1)} \to \Gamma^{(1)}$, т.~е. исключая дополнительное состояние обслуживания $\Gamma^{(2m-1)}$. Заметим, что согласно графу, изображенному на рисунке~\ref{S:img:graph_feedback}, ОУ будет менять свое состояние по указанной цепочке в случае, если очередь по высокоприоритетному потоку $\Pi_1$ накапливается и достигает размера $h_1$ довольно быстро. Иными словами, такой цикл реализуется при достаточно высокой интенсивности приоритетного потока, т.~е. в <<худшем>> случае. Следовательно, полученному критерию можно дать следующую интерпретацию: пропускная способность $l_1$ системы по потоку $\Pi_1$ превышает среднее число заявок этого потока, поступивших за один цикл работы системы при достаточно высокой его интенсивности. Если же система успешно справляется с обслуживанием потока $\Pi_1$ в <<худшем>> случае, то следует ожидать, что очередь также не будет накапливаться и при меньших интенсивностях потока $\Pi_1$, т.~е. при различных иных возможных цепочках перехода графа на рисунке~\ref{S:img:graph_feedback}.

% Аналогичные рассуждения можно применить к условиям, полученным в теоремах~7 и~8. Неравенства теоремы~7 указывают на то, что пропускная способность системы по потоку $\Pi_m$ не превышает среднего числа заявок этого потока, поступивших за промежуток пребывания ОУ в состояниях обслуживания $\Gamma^{(2m-1)}$ и $\Gamma^{(2m)}$. В свою очередь, совокупное выполнение условий теоремы~8 означает, с одной стороны, что система справляется с обслуживанием потока $\Pi_m$ за промежуток длительностью $T-T_{2m}$. С другой стороны, обслуживание в состоянии $\Gamma^{(2m)}$ организованно так, что суммарная пропускная способность $l_m + l_m^\prime$ меньше среднего числа заявок, поступающих за полный цикл смены состояний устройства обслуживания вида $\Gamma^{(1)} \to \Gamma^{(2)} \to \ldots \to \Gamma^{(2m-2)} \to \Gamma^{(2m-1)} \to \Gamma^{(2m)} \to \Gamma^{(2m+1)} \to \Gamma^{(1)}$. Таким образом, теоремы~7 и~8 дают представление о неоптимальной настройке параметров системы, которая заведомо не может привести к стационарному режиму функционирования. 

% В \underline{\textbf{четвертой главе}} методами имитационного моделирования проводится исследование систем, изучаемых в главах~2 и~3. Одной из основных задач, решаемых при численном исследовании, является синтез оптимальной в некотором смысле системы. Управляющими параметрами изучаемых систем являются длительности $T_k$, $k \in M$. При изучении системы управления с пороговым приоритетом из главы~3 управляющим параметром также является величина порога $h_1$. В качестве основного показателя эффективности выбирается среднее время ожидания начала обслуживания произвольным требованием. Задача оптимизации в этом случае состоит в определении таких значений управляющих параметров, при которых достигается минимальное значение среднего взвешенного времени ожидания. Получаемые результаты представляют собой не точное решение задачи оптимизации, а некоторую оценку, полученную при численном анализе. В связи с этим поставленная задача сводится к поиску т.~н. \textit{квазиоптимальной} стратегии управления системой. 

% Разработана компьютерная программа, реализующая имитационные модели для системы циклического управления и системы управления с обратной связью согласно алгоритму c графом переходов на рисунке~\ref{S:img:graph_feedback}. При этом предполагается, что в систему поступает $m = 2$ входных потоков. Раздел~4.2 содержит описание разработанных имитационных моделей, построенных на основе метода дискретных событий (Averill M.L., Kelton W.D. Simulation modeling and analysis. McGraw-Hill, 2000. 760~pp.). Для корректного использования некоторой величины как показателя качества функционирования системы необходимо, чтобы она характеризовала стационарный режим работы системы. С момента начала функционирования системы должны пройти переходные процессы, необходимые для ее стабилизации. Для определения момента достижения системой квазистационарного режима в разделах~4.2 и~4.3 предлагается следующий алгоритм. До запуска процесса имитации генерируются входные потоки $\Pi_1$ и $\Pi_2$. Одновременно происходит имитация работы системы при одних и тех же значениях параметров для двух следующих различных начальных условий. \textit{Нулевое начальное условие}: для любого $j \in \{1, 2\}$ первая пачка требований, поступающая по потоку $\Pi_j$, застает очередь $O_j$ пустой. \textit{Смещенное начальное условие}: для любого $j \in \{1, 2\}$ в очереди $O_j$ на обслуживание перед поступлением первой пачки требований потока $\Pi_j$ находится положительное количество $K_j l_j$ требований, где $K_j > 0$ есть параметр смещения для потока $\Pi_j$. Фиксируются параметры сближения <<траекторий>> процесса имитации для нулевых и смещенных начальных условий: $d \in \{1, 2, \ldots\}$ и $\delta \in (0, 1)$ и инициализируется счетчик $i = 0$. Каждый раз после завершения обслуживания заявки с номером $v$ потока $\Pi_j$ при смещенных начальных условиях вычисляются значения величин $\hat{\gamma}_{j, v}^{0} = \frac 1 v \sum_{u=1}^v \gamma_{j, u}^{0}$ и $\hat{\gamma}_{j, v}^{+} = \frac 1 v \sum_{u=1}^v \gamma_{j, u}^{+}$. Указанные величины определяют среднее время ожидания начала обслуживания по первым $v$ заявкам потока $\Pi_j$ в системе с нулевыми и смещенными начальными условиями соответственно. Если условие $|\hat{\gamma}_{j, v}^{+} - \hat{\gamma}_{j, v}^{0}| \leq \delta \hat{\gamma}_{j, v}^{0}$
% выполняется для очередного $v$, счетчик $i$ увеличивается на единицу. Иначе счетчик обнуляется: $i = 0$. Определяется значение $t_j$ текущего времени имитации $t$, при котором впервые выполнится равенство $i = d$, т.~е. траектории процесса обслуживания потока $\Pi_j$ для различных начальных условий устойчиво сблизятся. Момент $t^* = \max_{j \in J} t_j$ является моментом достижения квазистационарного режима во всей системе. Определяется номер $j^*$ потока, для которого квазистационарный режим достигнут позднее остальных, т.~е. $t_{j^*} = t^*$. Пусть $v_{j^*}$ номер первого требования потока $\Pi_j$, обслуженного в квазистационарном режиме. Такая процедура повторяется $N \in \{1, 2, \ldots\}$ раз с независимыми реализациями входных потоков при фиксированных значениях их параметров. Если $t^{n*}$ есть момент достижения квазистационарного режима для реализации с номером $n \in \{1, 2, \ldots, N\}$, то в качестве итоговой оценки момента достижения квазистационарного режима принимается величина $t^{**} = \max_{n \in \{1, 2, \ldots, N\}} t^{n*}$.

% Использование нескольких независимых реализаций обусловлено следующими соображениями. Пусть $\gamma^n_{j, v}$ есть время ожидания начала обслуживания требованием с номером $v$ потока $\Pi_j$ в реализации с номером $n$. С применением статистических критериев показано, что величины $\gamma^n_{j, v_{j^*}}$, $\gamma^n_{j, v_{j^*}+1}$, $\gamma^n_{j, v_{j^*}+2}$, \ldots, полученные в одной реализации, являются зависимыми и имеют разное распределение. В свою очередь, гипотезу о независимости и одинаковом распределении величин $\gamma^1_{j, v_{j^*}}$, $\gamma^2_{j, v_{j^*}}$, \ldots, $\gamma^N_{j, v_{j^*}}$ следует принять. Такая последовательность составлена из времен ожидания первых заявок квазистационарного режима во всех реализациях. Таким образом, целесообразно использовать указанную выборку для построения оценки среднего времени ожидания начала обслуживания первым требованием квазистационарного режима для потока $\Pi_j$: $\hat{\mathbf{M}} \gamma^*_{j} = \frac 1 N \sum_{n = 1}^N \gamma^n_{j, v_{j^*}}$. В силу случайности номера $v_{j^*}$ такую оценку можно считать оценкой среднего времени ожидания для произвольного требования потока $\Pi_j$. Оценка для среднего взвешенного времени ожидания начала обслуживания произвольным требованием системы
% \begin{equation*}
% \hat{\mathbf{M}} \gamma^* = \frac{\sum_{j = 1}^2 \lambda_j (3s_j + 2q_j + p_j) R_j \hat{\mathbf{M}} \gamma^*_{j}}{\sum_{j = 1}^2 \lambda_j (3s_j + 2q_j + p_j) R_j}
% \end{equation*}
% является показателем качества для системы в целом. Здесь для потока $\Pi_j$ введен вес $R_j > 0$, который позволяет учитывать приоритет различных потоков.

% В разделе~4.4 рассматривается алгоритм поиска квазиоптимальных значений управляющих параметров $T_1$ и $T_3$ для случая циклического управления однородными потоками. Область $D^1$ допустимых значений параметров ограничена, во-первых, условиями существования в системе стационарного режима, а во-вторых, безопасными границами $\underline{T} > 0$, $\overline{T} > \underline{T}$ для длительности полного цикла смены состояний ОУ:
% \[
% D^1 = \{(T_1, T_3)\colon \lambda_j T (2s_j + q_j + 1) - l_j < 0, j = 1, 2,\; \underline{T} \leq T \leq \overline{T}\}.
% \]
% В области $D^1$ выделяется т.~н. ломаная равных квазизагрузок, задаваемая уравнением $\frac {\lambda_1 T (2s_1 + q_1 + 1)}{[\mu_1 T_1]} = \frac {\lambda_2 T (2s_2 + q_2 + 1)}{[\mu_2 T_3]}$. На первом этапе работы алгоритма производится имитация работы системы в точках $(T_1, T_3)$, лежащих на ломаной равных квазизагрузок и отстоящих друг от друга на фиксированном расстоянии по длительности $T_1$. Пусть среди просмотренных точек минимальное значение оценки $\hat{\mathbf{M}} \gamma^*$ было достигнуто в точке $(\hat{T}_1, \hat{T}_3)$. На втором этапе алгоритма функционирование системы имитируется в точках $(T_1, T_3)$, лежащих в области $D^1$ на прямой $T_1 + T_3 = \hat{T}_1 + \hat{T}_3$ и отстоящих друг от друга на фиксированном расстоянии по длительности $T_1$. Определяется точка $(T_1^*, T_3^*)$, на которой было достигнуто минимальное значение оценки $\hat{\mathbf{M}} \gamma^*$. Значения $T_1^*$ и $T_3^*$ длительностей состояний обслуживания для потоков $\Pi_1$ и $\Pi_2$ считаются квазиоптимальными.

% В разделе~4.5 предлагается алгоритм поиска квазиоптимальных значений управляющих параметров $T_1$, $T_3$, $T_4$ и $h_1$ для случая управления неоднородными потоками по алгоритму, рассмотренному в главе~3. Исследуется область изменения значения параметров
% \begin{equation*}
% 	\begin{gathered}
% 		D^2 = \{(T_1, T_3, T_4, h_1) \colon \lambda_1 T^* (3 s_1 + 2 q_1 + p_1) - [\mu_1 T_1] < 0, \\
% 		\lambda_2 T_{4} (3 s_2 + 2 q_2 + p_2) - [\mu_2 T_4] < 0,\;\;
% 		\lambda_2 (T-T_{4}) (3 s_2 + 2 q_2 + p_2) - [\mu_2 T_3] < 0,\\
% 		\underline{T} \leq T \leq \overline{T}, \;\; h_1 \in \{0, 1, \ldots, [\mu_1 T_1] - 1\}\}.
% 	\end{gathered}
% \end{equation*} 
% Рассматривается случай, когда значение $T_4$ фиксировано и равно $T_4 = t_4$. Тогда показатель $\hat{\mathbf{M}} \gamma^*$ является функцией точки вида $x = (T_1, T_3, t_4, h_1)$, т.~е. $\hat{\mathbf{M}} \gamma^* = \hat{\mathbf{M}} \gamma^*(x)$. Устанавливаются значения для параметров алгоритма: $t_3 > 0$, $h_4 > 0$, $h_5 > 0$, $h_6 \in \{1, 2, \ldots\}$. Алгоритм поиска квазиоптимальных значений параметров состоит в следующем:

% 1. Положить $i = 1$, $T = \underline{T}$.

% 2. Положить $T_3^i = t_3$. 

% 3. Положить $h_1^i = 1$, определить $T_1^i = T - (2t_0 + T_3^i + t_4)$.

% 4. Если точка $x_i = (T_1^i, T_3^i, t_4, h_1^i)$ принадлежит области $D^2$, запустить процесс имитации, получить значение оценки $\hat{\mathbf{M}} \gamma^*(x_i)$, перейти в шагу 5. В противном случае положить $\hat{\mathbf{M}} \gamma^*(x_i) = \infty$ и перейти к шагу 5.

% 5. Если $h_1^i + h_6 \leq [\mu_1 T_1^i] - 1$, то положить $i = i+1$, $T_1^i = T_1^{i-1}$, $T_3^i = T_3^{i-1}$, $h_1^i = h_1^{i-1} + h_6$, перейти к шагу 4. В противном случае перейти к шагу~6.

% 6. Если $T_3^i + h_4 < T - 2t_0 - t_4$, то положить $i = i+1$, $T_3^i = T_3^{i-1} + h_4$, перейти к шагу 3. В противном случае перейти к шагу 7. 

% 7. Если $T + h_5 \leq \overline{T}$, то положить $T = T + h_5$, $i = i + 1$ и перейти к шагу 2. В противном случае перейти к шагу 8.

% 8. Среди всех просмотренных точек определить номер $i^*$ точки, для которой было достигнуто минимальное значение оценки $\hat{\mathbf{M}} \gamma^* (x_i)$, т.~е. $\hat{\mathbf{M}} \gamma^* (x_{i^*}) = \min_{i}\hat{\mathbf{M}} \gamma^* (x_i)$.

% Квазиоптимальные значения управляющих параметров системы есть $T_1^* = T_1^{i^*}$, $T_3^* = T_3^{i^*}$ и $h_1^* = h_1^{i^*}$ с наилучшим показателем среднего взвешенного времени ожидания начала обслуживания $\hat{\mathbf{M}} \gamma^* (x_{i^*})$. 

% В \underline{\textbf{заключении}} приведены основные результаты работы, которые состоят в следующем:
% %%% Согласно ГОСТ Р 7.0.11-2011:
%% 5.3.3 В заключении диссертации излагают итоги выполненного исследования, рекомендации, перспективы дальнейшей разработки темы.
%% 9.2.3 В заключении автореферата диссертации излагают итоги данного исследования, рекомендации и перспективы дальнейшей разработки темы.
\begin{enumerate}
  \item Построена и исследована математическая модель потока неоднородных требований в виде неординарного пуассоновского потока с ограниченным количеством требований в группе. Найдены основные вероятностные характеристики такого потока. 
  \item Для обоснования корректности построенной модели потока разработана компьютерная программа. Такая программа позволяет 1)~получить нелокальное описание реальных потоков требований, 2)~проверить, может ли построенная математическая модель быть использована при описании реальных потоков, 3)~получить оценку неизвестных параметров распределений, возникающих в указанной модели.
  \item Построена математическая модель для двух систем обслуживания требований и управления потоками: 1)~циклическое управление потоками неоднородных заявок, 2)~управление разнородными потоками с помощью адаптивного алгоритма с пороговым приоритетом и возможностью продления обслуживания. В обоих случаях модель представляет собой многомерную однородную управляемую цепь Маркова. 
  \item Произведена классификация состояний указанных цепей Маркова, получены рекуррентные соотношения для одномерных распределений таких марковских процессов и их производящих функций. Итеративно-мажорантным методом получены условия существования в системах стационарного режима. Найденные условия являются легко проверяемыми ограничениями на значения параметров и характеристик системы. 
  \item Для указанных систем построены компьютерные имитационные модели. Разработан алгоритм определения момента достижения системами квазистационарного режима. Предложен способ получения численных оценок основных показателей качества функционирования системы.
  \item Для рассмотренных систем управления предложены алгоритмы поиска квазиоптимальных значений управляющих параметров, при которых достигается минимальное значение оценки среднего времени ожидания начала обслуживания произвольной заявкой в квазистационарном режиме.
\end{enumerate}


% 1. Построена и исследована математическая модель потока неоднородных требований в виде неординарного пуассоновского потока с ограниченным количеством требований в группе. Найдены основные вероятностные характеристики такого потока. 

% 2. Для обоснования корректности построенной модели потока разработана компьютерная программа. Такая программа позволяет 1)~получить нелокальное описание реальных потоков требований, 2)~проверить, может ли построенная математическая модель быть использована при описании реальных потоков, 3)~получить оценку неизвестных параметров распределений, возникающих в указанной модели.

% 3. Построена математическая модель для двух систем обслуживания требований и управления потоками: 1)~циклическое управление потоками неоднородных заявок, 2)~управление разнородными потоками с помощью адаптивного алгоритма с пороговым приоритетом и возможностью продления обслуживания. В обоих случаях модель представляет собой многомерную однородную управляемую цепь Маркова. 

% 4. Произведена классификация состояний указанных цепей Маркова, получены рекуррентные соотношения для одномерных распределений таких марковских процессов и их производящих функций. Итеративно-мажорантным методом получены условия существования в системах стационарного режима. Найденные условия являются легко проверяемыми ограничениями на значения параметров и характеристик системы. 

% 5. Для указанных систем построены компьютерные имитационные модели. Разработан алгоритм определения момента достижения системами квазистационарного режима. Предложен способ получения численных оценок основных показателей качества функционирования системы.

% 6. Для рассмотренных систем управления предложены алгоритмы поиска квазиоптимальных значений управляющих параметров, при которых достигается минимальное значение оценки среднего времени ожидания начала обслуживания произвольной заявкой в квазистационарном режиме.

% Также в заключении формулируются возможные направления дальнейших исследований.


% \section*{Список публикаций по теме диссертации}

% \textbf{Публикации в изданиях, рекомендованных ВАК Российской Федерации и включенных в перечень международных баз цитирования (Web of Science, Scopus):}

% \begin{enumerate}
% 	\item Рачинская М.А., Федоткин М.А. Построение и исследование вероятностной модели циклического управления потоками малой интенсивности //	Вестник Нижегородского университета им. Н.И.~Лобачевского. — 2014.	— №~4(1). — С.~370–376.
% 	\item Рачинская М.А., Федоткин М.А. Исследование условий существования стационарного режима в системе конфликтного обслуживания неоднородных требований // Вестник Томского государственного университета. Математика и механика. — 2018. — №~51. — С.~33–47. (Rachinskaya M.A., Fedotkin M.A. Investigation of the stationary mode existence in a system of conflict service of non-homogeneous demands // Tomsk State University Journal of Mathematics and Mechanics. — 2018. — V.~51. — Pp.~33-47.)
% 	\item Fedotkin M.A., Rachinskaya M.A. Parameters estimator of the probabilistic model of moving batches traffic flow // Distributed Computer and Communication Networks, Ser. Communications in Computer and Information Science. — 2014. — V.~279. — Pp.~154–169.
% 	\item Rachinskaya M., Fedotkin M. Stationarity conditions for the control systems that provide service to the conflicting batch Poisson flows // Lecture Notes
% 	in Computer Science (including subseries Lecture Notes in Artificial Intelligence and Lecture Notes in Bioinformatics). — 2017. — V.~10684 LNCS. —
% 	Pp.~43–53.
% \end{enumerate}	


% \textbf{Свидетельство о государственной регистрации программы для ЭВМ:}

% \begin{enumerate}
% 	\setcounter{enumi}{4}
% 	\item Рачинская М.А. Статистический анализ потока событий: А.~с.~№~2016616411, дата государственной регистрации в Реестре программ для ЭВМ 10 июня 2016 г. — 2016.
% \end{enumerate}		

% \textbf{Публикации в изданиях, рекомендованных ВАК Российской Федерации для защиты по смежным специальностям:}

% \begin{enumerate}\setcounter{enumi}{5}
% 	\item Федоткин М.А., Рачинская М.А. Имитационная модель циклического управления конфликтными неординарными пуассоновскими потоками // Вестник Волжской государственной академии водного транспорта. — 	2016. — №~47. — С.~43–51.
% 	\item Федоткин М.А., Рачинская М.А. Модель функционирования системы управления и обслуживания потоков разной интенсивности и приоритетности // Вестник Волжской государственной академии водного транспорта. — 2016. — №~48. — С.~62–69.
% \end{enumerate}	

\textbf{Публикации в иных научных изданиях:}

\begin{enumerate}
	\setcounter{enumi}{7}
	\item Fedotkin M.A., Kudryavtsev E.V., Rachinskaya M.A. About correctness of probabilistic models of traffic flows dynamics on a motorway // Proceedings of International Workshop «Distributed computer and communication networks»
	(DCCN-2010). — Moscow: 2010. — Pp.~86–93.
	\item Fedotkin M.A., Kudryavtsev E.V., Rachinskaya M.A. Simulation and research of probabilistic regularities in motion of traffic flows // Proceedings of the International conference «Applied Methods of Statistical Analysis. Simulations and Statistical Inference». — Novosibirsk: Novosibirsk State Technical University, 2011. — Pp.~117–124.
	\item Fedotkin M., Rachinskaya M. Investigation of traffic flows characteristics in case of the small density // Collection «Queues: Flows, Systems, Networks».
	Proceedings of the International Conference «Modern Probabilistic Methods for Analysis and Optimization of Information Telecommunication Networks». — No.~21. — Minsk: BSU, 2011. — Pp.~82–87.
	\item Федоткин М.А., Рачинская М.А. Изучение математической модели трафика
	автомобилей на основе подхода Ляпунова-Яблонского // Сборник
	научных статей XVI Международной конференции «Проблемы теоретической кибернетики». — Н. Новгород: ННГУ, 2011. — С.~508–512.
	\item Рачинская М.А., Федоткин М.А. Изучение характеристик потока машин в условиях малой плотности. — Нижегородский государственный университет им. Н.И.~Лобачевского, Нижний Новгород, 2012. — 36~с. — Деп. в ВИНИТИ 26.01.12, №~27 — В2012.
	\item Fedotkin M.A., Rachinskaya M.A. Parameters estimator of the probabilistic model of batches traffic flow with the non-intensive movement // Proceedings of International Workshop «Distributed computer and communication networks»
	(DCCN-2013). — Moscow: 2013. — Pp.~357–364.
	\item Рачинская М.А., Федоткин М.А. Построение вероятностной модели процесса циклического управления конфликтными потоками пачек в условиях малой плотности. — Нижегородский государственный университет им.
	Н.И.~Лобачевского, Нижний Новгород, 2014. — 30~с. — Деп. в ВИНИТИ 14.01.2014., №~13 — В2014.
	\item Рачинская М.А., Федоткин М.А. Предельные свойства распределений выходных процессов циклического управления конфликтными потоками пачек в условиях малой плотности. — Нижегородский государственный университет им. Н.И.~Лобачевского, Нижний Новгород, 2014. — 38~с. —
	Деп. в ВИНИТИ 23.04.2014., №~111 — В2014.
	\item Федоткин М.А., Рачинская М.А. Подход Ляпунова-Яблонского при построении
	и исследовании модели управляющих систем обслуживания конфликтных	потоков // Сборник научных статей XVII Международной конференции «Проблемы теоретической кибернетики». — Казань: КФУ, 2014. — С.~280–282.
	\item Рачинская М.А., Федоткин М.А. Численное исследование и синтез дискретных управляющих систем обслуживания // IX Международная конференция <<Дискретные модели в теории управляющих систем>>: Москва и Подмосковье, 20-22 мая 2015 г.: Труды / Отв. ред. В.Б.~Алексеев, Д.С.~Романов, Б.Р.~Данилов. — М.: МАКС Пресс, 2015. — С.~200–202.
	\item Федоткин М.А., Рачинская М.А. Свойства стационарного режима в модели управления конфликтными потоками // Теория вероятностей, случайные процессы, математическая статистика и приложения. Материалы Международной научной конференции. — Минск: РИВШ, 2015. —	С.~262–267.
	\item Федоткин М.А., Рачинская М.А. Статистический анализ потока импульсов вдоль нервного волокна // Статистика в современном обществе: ее роль и значение в вопросах государственного управления и общественного
	развития: Материалы Межрегиональной научно-практической конференции, посвященной 180-летию со времени образования органов государственной статистики Нижегородской области (г. Н. Новгород, 28 мая	2015 г.). — Н. Новгород: Нижегородстат – Нижегородский госуниверситет, 2015. — С.~457–464.
	\item Rachinskaya M.A., Fedotkin M.A. Research of the process of traffic flows control by means of simulation // «Distributed computer and communication	networks: control, computation, communications» (DCCN-2015): материалы
	Восемнадцатой междунар. Научн. Конфер, 19-22 окт. 2015 г., Москва: / Ин-т проблем упр. им. В.А. Трапезникова Рос. Акад. Наук. — М.: ИПУ РАН, 2015. — С.~136–143.
	\item Рачинская М.А., Федоткин М.А. Построение модели и анализ управляющих систем обслуживания // Материалы XII Международного семинара <<Дискретная математика и ее приложения>>, им. академика О.Б.~Лупанова
	(Москва, МГУ, 20-25 июня 2016 г.). — М.: Изд-во механико-математического факультета МГУ, 2016. — С.~156–158.
	\item Рачинская М.А., Федоткин М.А. Исследование операций по управлению конфликтными потоками неоднородных требований // Проблемы теоретической кибернетики: XVIII международная конференция (Пенза, 19-23 июня 2017г.): Материалы: Под редакцией Ю.И. Журавлева. — М.: МАКС Пресс, 2017. — С.~203–205.
	\item Rachinskaya M.A., Fedotkin M.A. Probabilistic and simulation model of the queuing system with non-homogeneous input flows and feedback control algorithm with prolongations // Distibuted computer and communication
	networks: control, computation, communications (DCCN-2017): материалы Двадцатой междунар. науч. конфер., 25-29 сент. 2017 г., Москва: / Ин-т проблем упр. им. В.А.~Трапезникова Рос. акад. наук; под общ. ред. В.М.~Вишневского. — М.: ТЕХНОСФЕРА, 2017. — С.~510–516.
	\item Рачинская М.А., Федоткин М.А. Квазиоптимальное управление неординарными
	пуассоновскими потоками // Информационные технологии и
	математическое моделирование (ИТММ-2017): Материалы XVI Международной конференции имени А.Ф.~Терпугова (29 сентября - 3 октября	2017 г.). — Томск: Изд-во НТЛ, 2017. — С.~164–171.
	\item Rachinskaya M.A., Fedotkin M.A. Stationarity conditions for the control systems that provide service to the conflicting non-ordinary Poisson flows // Аналитические и вычислительные методы в теории вероятностей и ее
	приложениях (АВМТВ-2017): материалы Международной научной конференции. Россия, Москва, 23–27 октября 2017 г. / под общ. ред. А.В.~Лебедева. — Москва: РУДН, 2017. — С.~629–633.
	\item Rachinskaya M.A., Fedotkin M.A. Research of a multidimensional Markov chain as a model for the class of queueing systems controlled by a threshold priority algorithm // Reliability: Theory \& Applications. — 2018. — №1(48). V.13. — Pp.~47-58.
	\item Рачинская М.А., Федоткин М.А. Оптимизация дискретных управляющих систем многопоточного обслуживания // Дискретные модели в теории управляющих систем: Х Международная конференция, Москва и Подмосковье, 23–25 мая 2018 г. : Труды / Отв. ред. В.Б. Алексеев, Д.С. Романов, Б.Р. Данилов. — М.: МАКС Пресс, 2018. — С.~228-231.	
\end{enumerate}	


%\ifdefmacro{\microtypesetup}{\microtypesetup{protrusion=false}}{} % не рекомендуется применять пакет микротипографики к автоматически генерируемому списку литературы
%\ifnumequal{\value{bibliosel}}{1}{% Встроенная реализация с загрузкой файла через движок bibtex8
%  \renewcommand{\refname}{\large \authorbibtitle}
%  \nocite{*}
%  \insertbiblioauthor                          % Подключаем Bib-базы
  %\insertbiblioother   % !!! bibtex не умеет работать с несколькими библиографиями !!!
%}{% Реализация пакетом biblatex через движок biber
%  \insertbiblioauthor                          % Подключаем Bib-базы
%  \insertbiblioother
%}
%\ifdefmacro{\microtypesetup}{\microtypesetup{protrusion=true}}{}

